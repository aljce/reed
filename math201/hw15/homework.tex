\documentclass[fleqn]{article}

\usepackage[a4paper, margin=2cm]{geometry}
\usepackage{amsmath}
\usepackage{amssymb}
\usepackage{gauss}
\usepackage[inline]{enumitem}

\setlength{\parindent}{0pt}
\setlength{\mathindent}{0pt}

% Allow for Augmented Matricies
\usepackage{etoolbox}
\makeatletter
\patchcmd\g@matrix
 {\vbox\bgroup}
 {\vbox\bgroup\normalbaselines}% restore the standard baselineskip
 {}{}
\makeatother

\newcommand{\BAR}{%
  \hspace{-\arraycolsep}%
  \strut\vrule % the `\vrule` is as high and deep as a strut
  \hspace{-\arraycolsep}%
}

\newcommand{\squig}[0]{\ensuremath{\rightsquigarrow}}

\newcommand{\problem}[1]{\large\textbf{Problem #1}\normalsize}

\newcommand{\evidence}[1]{\ensuremath{(\hspace{0.2em} \text{#1} \hspace{0.2em})}}
\newcommand{\relation}[1]{\ensuremath{\hspace{0.2em} {{} #1 {}} \hspace{0.2em}}}
\newcommand{\equal}{\relation{=}}
\newcommand{\qed}{\hfill\ensuremath{\square}}

\newcommand{\idF}[1]{\ensuremath{\text{id}(#1)}}
\newcommand{\coordsF}[2]{\ensuremath{[ \: #1 \: ]_{\mathcal{#2}}}}
\newcommand{\rankF}[1]{\ensuremath{\text{rank}(#1)}}
\newcommand{\nullityF}[1]{\ensuremath{\text{nullity}(#1)}}
\newcommand{\matrixRep}[3]{\ensuremath{{\left [ \: #1 \: \right ]}_{\mathcal{#2}}^{\mathcal{#3}}}}
\newcommand{\signF}[1]{\ensuremath{\text{sign}(#1)}}
\usepackage{listofitems}
% Cycle Notation
\newcommand\cycle[2][\:]{
  \readlist\thecycle{#2}
  #1\foreachitem\i\in\thecycle{\ifnum\icnt=1\else#1\fi\i}#1
}
\begin{document}

\noindent\Large\textbf{Problem Set Week 10 Tuesday} \\
\normalsize
Alice McKean \\
\today \\

\problem{1a}

The eigenvalues of $A$ are $\lambda = -1, 3$ as shown by the following equation:
\begin{align*}
  \rowarrowsep=-2pt
  \det
  \begin{gmatrix}[p]
    1 - \lambda & 1 \\
    4 & 1 - \lambda
  \end{gmatrix}
  = (1 - \lambda)^2 - 4 = (\lambda + 1)(\lambda - 3)
\end{align*}

The eigenvectors for $\lambda = -1$ are the nonzero solutions to:
\begin{align*}
  \rowarrowsep=-2pt
  \begin{gmatrix}[p]
    1 + 1 & 1 \\
    4 & 1 + 1
  \end{gmatrix} 
  \begin{gmatrix}[p]
    x \\
    y
  \end{gmatrix}
  =
  \begin{gmatrix}[p]
    2 & 1 \\
    4 & 2
  \end{gmatrix} 
  \begin{gmatrix}[p]
    x \\
    y
  \end{gmatrix}
  =
  \begin{gmatrix}[p]
    0 \\
    0
  \end{gmatrix}
\end{align*}
Thus the set of eigenvectors for $\lambda = -1$ can be described as:
\begin{align*}
\rowarrowsep=-2pt
\left \{ \:
  t \cdot
  \begin{gmatrix}[p]
    1 \\
    -2
  \end{gmatrix}
  \: \biggr | \: t \in \mathbb{R}, \: t \neq 0 \:
\right \}
\end{align*}

The eigenvectors for $\lambda = 3$ are the nonzero solutions to:
\begin{align*}
  \rowarrowsep=-2pt
  \begin{gmatrix}[p]
    1 - 3 & 1 \\
    4 & 1 - 3
  \end{gmatrix} 
  \begin{gmatrix}[p]
    x \\
    y
  \end{gmatrix}
  =
  \begin{gmatrix}[p]
    -2 & 1 \\
    4 & -2
  \end{gmatrix} 
  \begin{gmatrix}[p]
    x \\
    y
  \end{gmatrix}
  =
  \begin{gmatrix}[p]
    0 \\
    0
  \end{gmatrix}
\end{align*}
Thus the set of eigenvectors for $\lambda = 3$ can be described as:
\begin{align*}
\rowarrowsep=-2pt
\left \{ \:
  t \cdot
  \begin{gmatrix}[p]
    1 \\
    2
  \end{gmatrix}
  \: \biggr | \: t \in \mathbb{R}, \: t \neq 0 \:
\right \}
\end{align*}
The set $\mathcal{B}$ given below clearly forms a basis for $\mathbb{R}^2$.
\begin{align*}
  \rowarrowsep=-2pt
  \mathcal{B} =
  \left \{
  \begin{gmatrix}[p]
    1 \\
    -2
  \end{gmatrix}
  ,
  \begin{gmatrix}[p]
    1 \\
    2
  \end{gmatrix}
  \right \}
\end{align*}
Now all the pieces put together $A = PDP^{-1}$:
\begin{align*}
  \rowarrowsep=-2pt
  \begin{gmatrix}[p]
    1 & 1 \\
    4 & 1
  \end{gmatrix}
  =
  \begin{gmatrix}[p]
    1 & 1 \\
    -2 & 2
  \end{gmatrix}
  \begin{gmatrix}[p]
    -1 & 0 \\
     0 & 3
  \end{gmatrix}
  \begin{gmatrix}[p]
    1/2 & -1/4 \\
    1/2 & 1/4
  \end{gmatrix}
\end{align*}

\problem{1b}

The eigenvalues of $A$ are $\lambda = 2, 1, -1$ as shown by the following equation:
\begin{align*}
  \rowarrowsep=-2pt
  \det
  \begin{gmatrix}[p]
    7 - \lambda & 4 & 10 \\
    4 & 3 - \lambda & 8 \\
    -2 & 1 & -2 - \lambda
  \end{gmatrix}
  = - \lambda^3 + 2\lambda^2 + \lambda - 2
  = -(\lambda - 2)(\lambda - 1)(\lambda + 1)
\end{align*}

The eigenvectors for $\lambda = 2$ are the nonzero solutions to:
\begin{align*}
  \rowarrowsep=-2pt
  \begin{gmatrix}[p]
    7 - 2 & -4 & 10 \\
    4 & -3 - 2 & 8 \\
    -2 & 1 & -2 - 2
  \end{gmatrix} 
  \begin{gmatrix}[p]
    x \\
    y \\
    z
  \end{gmatrix}
  =
  \begin{gmatrix}[p]
    5 & -4 & 10 \\
    4 & -5 & 8 \\
    -2 & 1 & -4
  \end{gmatrix} 
  \begin{gmatrix}[p]
    x \\
    y \\
    z
  \end{gmatrix}
  =
  \begin{gmatrix}[p]
    0 \\
    0 \\
    0
  \end{gmatrix}
\end{align*}

After solving the linear equations above the eigenvectors for $\lambda = 2$ are:
\begin{align*}
\rowarrowsep=-2pt
\left \{ \:
  t \cdot
  \begin{gmatrix}[p]
    -2 \\
    0 \\
    1
  \end{gmatrix}
  \: \biggr | \: t \in \mathbb{R}, \: t \neq 0 \:
\right \}
\end{align*}

The eigenvectors for $\lambda = 1$ are the nonzero solutions to:
\begin{align*}
  \rowarrowsep=-2pt
  \begin{gmatrix}[p]
    7 - 1 & -4 & 10 \\
    4 & -3 - 1 & 8 \\
    -2 & 1 & -2 - 1
  \end{gmatrix} 
  \begin{gmatrix}[p]
    x \\
    y \\
    z
  \end{gmatrix}
  =
  \begin{gmatrix}[p]
    6 & -4 & 10 \\
    4 & -4 & 8 \\
    -2 & 1 & -3
  \end{gmatrix} 
  \begin{gmatrix}[p]
    x \\
    y \\
    z
  \end{gmatrix}
  =
  \begin{gmatrix}[p]
    0 \\
    0 \\
    0
  \end{gmatrix}
\end{align*}

After solving the linear equations above the eigenvectors for $\lambda = 1$ are:
\begin{align*}
\rowarrowsep=-2pt
\left \{ \:
  t \cdot
  \begin{gmatrix}[p]
    1 \\
    -1 \\
    -1
  \end{gmatrix}
  \: \biggr | \: t \in \mathbb{R}, \: t \neq 0 \:
\right \}
\end{align*}

The eigenvectors for $\lambda = -1$ are the nonzero solutions to:
\begin{align*}
  \rowarrowsep=-2pt
  \begin{gmatrix}[p]
    7 + 1 & -4 & 10 \\
    4 & -3 + 1 & 8 \\
    -2 & 1 & -2 + 1
  \end{gmatrix} 
  \begin{gmatrix}[p]
    x \\
    y \\
    z
  \end{gmatrix}
  =
  \begin{gmatrix}[p]
    8 & -4 & 10 \\
    4 & -2 & 8 \\
    -2 & 1 & -1
  \end{gmatrix} 
  \begin{gmatrix}[p]
    x \\
    y \\
    z
  \end{gmatrix}
  =
  \begin{gmatrix}[p]
    0 \\
    0 \\
    0
  \end{gmatrix}
\end{align*}

After solving the linear equations above the eigenvectors for $\lambda = -1$ are:
\begin{align*}
\rowarrowsep=-2pt
\left \{ \:
  t \cdot
  \begin{gmatrix}[p]
    1 \\
    2 \\
    0
  \end{gmatrix}
  \: \biggr | \: t \in \mathbb{R}, \: t \neq 0 \:
\right \}
\end{align*}

The set $\mathcal{B}$ forms a basis for $\mathbb{R}^3$ as shown below:
\begin{align*}
  \mathcal{B} =
  \left \{
  \rowarrowsep=-2pt
  \begin{gmatrix}[p]
    -2 \\
    0 \\
    1
  \end{gmatrix}
  ,
  \begin{gmatrix}[p]
    1 \\
    -1 \\
    -1
  \end{gmatrix}
  ,
  \begin{gmatrix}[p]
    1 \\
    2 \\
    0
  \end{gmatrix}
  \right \}
\end{align*}
The set is linearly independent:
\begin{align*}
  \begin{cases}
  \rowarrowsep=-2pt
  a \cdot
  \begin{gmatrix}[p]
    -2 \\
    0 \\
    1
  \end{gmatrix}
  + b \cdot
  \begin{gmatrix}[p]
    1 \\
    -1 \\
    -1
  \end{gmatrix}
  + c \cdot
  \begin{gmatrix}[p]
    1 \\
    2 \\
    0
  \end{gmatrix}
  = 0
  \end{cases}
  \squig
  \begin{cases}
    0 = -2a + b + c \\
    0 = -b + 2c \\
    0 = a - b
  \end{cases}
  \squig
  \begin{cases}
    0 = a \\
    0 = b \\
    0 = c
  \end{cases}
\end{align*}
and it spans $\mathbb{R}^3$:
\begin{align*}
  \rowarrowsep=-2pt
  \begin{gmatrix}[p]
    a \\
    b \\
    c
  \end{gmatrix}
  =
  (-2a + b - 3c) \cdot
  \begin{gmatrix}[p]
    -2 \\
    0 \\
    1
  \end{gmatrix}
  + (-2a + b - 4c) \cdot
  \begin{gmatrix}[p]
    1 \\
    -1 \\
    -1
  \end{gmatrix}
  + (-a + b - 2c) \cdot
  \begin{gmatrix}[p]
    1 \\
    2 \\
    0
  \end{gmatrix}
\end{align*}
Now all the pieces put together $A = PDP^{-1}$:
\begin{align*}
  \rowarrowsep=-2pt
  \begin{gmatrix}[p]
    7 & -4 & 10 \\
    4 & -3 & 8 \\
    -2 & 1 & -2
  \end{gmatrix} 
  =
  \begin{gmatrix}[p]
    -2 & 1 & 1 \\
    0 & -1 & 2 \\
    1 & -1 & 0
  \end{gmatrix}
  \begin{gmatrix}[p]
    2 & 0 & 0 \\
    0 & 1 & 0 \\
    0 & 0 & -1
  \end{gmatrix}
  \begin{gmatrix}[p]
    -2 & 1 & -3 \\
    -2 & 1 &  -4 \\
    -1 & 1 & -2
  \end{gmatrix}
\end{align*}

\problem{1c}

The matrix $A$ has no real eigenvalues as demonstrated by the following equation: 
\begin{align*}
  \rowarrowsep=-2pt
  \det
  \begin{gmatrix}[p]
    7 - \lambda & -5 \\
    10 & -7 - \lambda
  \end{gmatrix}
  = \lambda^2 + 1
\end{align*}

\problem{2}

Assume $f : V \to V$ is a linear transformation with eigenvalue $\lambda$ such
that $f(v) = \lambda v$ for some $v$. Then $\lambda^m$ is an eigenvalue for
$f^m$ as shown below:
\begin{enumerate}[label=\roman*:]
  \item When $m = 1$ the base case is trival $f^1(v) = f(v) = \lambda v = \lambda^1 v$
  \item When $m \geq 1$ assume that for any positive $k \leq m$ the equality $f^k(v) =
    \lambda^kv$ holds. The inductive step: \\
    $f^{m + 1}(v) = (f \circ f^m)(v) = f(f^m(v)) = f(\lambda^mv) = \lambda^mf(v)
    = \lambda^m \lambda v = \lambda^{m + 1}v$
\end{enumerate}
Thus by the principle of induction $f^m(v) = \lambda^mv$ for any positive $m$.
$\qed$ \\

\problem{3a}

An eigenvalue of $T$ is $1$ as $T(I_n) = I_n = 1 \cdot I_n$. Another eigenvalue
of $T$ is $-1$ as shown by the following equality:
\begin{align*}
  \rowarrowsep=-2pt
  T
  \begin{gmatrix}[p]
    0 & -1 \\
    1 & 0
  \end{gmatrix}
  =
  \begin{gmatrix}[p]
    0 & 1 \\
    -1 & 0
  \end{gmatrix}
  = -1 \cdot
  \begin{gmatrix}[p]
    0 & -1 \\
    1 & 0
  \end{gmatrix}
\end{align*}
Now assume $T$ has some other eigenvalue $\lambda$ such that $T(v) = \lambda v$
for some $v$. $T$ is a linear transformation so by the previous problem
$\lambda^2v = T^2(v) = (T \circ T)(v) = T(T(v)) = v$. Therefore $\lambda = \pm
1$. $\qed$ \\

\problem{3b}

The eigenvectors for the eigenvalue $1$ are all the matrices $A \in M_{2\times 2}(\mathbb{R})$
such that $A_{ij} = A_{ji}$.

Similarly the eigenvectors for the eigenvalue $-1$ are all the matrices
$A \in M_{2 \times 2}(\mathbb{R})$ such that $A_{ij} = -A_{ji}$. \\

\problem{3c}
\begin{align*}
  \rowarrowsep=-2pt
  \mathcal{B} = \left \{
  \begin{gmatrix}[p]
    1 & 0 \\
    0 & 0
  \end{gmatrix} 
  ,
  \begin{gmatrix}[p]
    0 & 1 \\
    1 & 0
  \end{gmatrix} 
  ,
  \begin{gmatrix}[p]
    0 & 0 \\
    0 & 1
  \end{gmatrix} 
  ,
  \begin{gmatrix}[p]
    0 & 1 \\
    -1 & 0
  \end{gmatrix} 
  \right \}
\end{align*}

This set is clearly linearly independent and it spans
$M_{2\times 2}(\mathbb{R})$ as show in the following equation:
\begin{align*}
  \rowarrowsep=-2pt
  \begin{gmatrix}[p]
    a & b \\
    c & d
  \end{gmatrix}
  =
  a \cdot
  \begin{gmatrix}[p]
    1 & 0 \\
    0 & 0
  \end{gmatrix} 
  + \frac{b + c}{2} \cdot
  \begin{gmatrix}[p]
    0 & 1 \\
    1 & 0
  \end{gmatrix} 
  + d \cdot
  \begin{gmatrix}[p]
    0 & 0 \\
    0 & 1
  \end{gmatrix} 
  + \frac{b - c}{2} \cdot
  \begin{gmatrix}[p]
    0 & 1 \\
    -1 & 0
  \end{gmatrix} 
\end{align*}
Now we can find the matrix that represents $T$ with respect to $B$ and $B$:
\begin{align*}
  \rowarrowsep=-2pt
  \matrixRep{T}{B}{B} =
  \begin{gmatrix}[p]
    \BAR      & \BAR      & \BAR & \BAR \\
    \coordsF{T
      \begin{gmatrix}[p]
        1 & 0 \\
        0 & 0
      \end{gmatrix} 
    }{B} &
    \coordsF{T
      \begin{gmatrix}[p]
        0 & 1 \\
        1 & 0
      \end{gmatrix} 
    }{B} &
    \coordsF{T
      \begin{gmatrix}[p]
        0 & 0 \\
        0 & 1
      \end{gmatrix} 
    }{B} &
    \coordsF{T
      \begin{gmatrix}[p]
        0 & 1 \\
        -1 & 0
      \end{gmatrix} 
    }{B} \\
    \BAR      & \BAR      & \BAR & \BAR
  \end{gmatrix} 
  =
  \begin{gmatrix}[p]
    1 & 0 & 0 & 0 \\
    0 & 1 & 0 & 0 \\
    0 & 0 & 1 & 0 \\
    0 & 0 & 0 & -1
  \end{gmatrix}
\end{align*}

\problem{3d}

The eigenvectors for the eigenvalue $1$ are all the matrices $A \in M_{n\times n}(\mathbb{R})$
such that $A_{ij} = A_{ji}$.

Similarly the eigenvectors for the eigenvalue $-1$ are all the matrices
$A \in M_{n \times n}(\mathbb{R})$ such that $A_{ij} = -A_{ji}$. \\

\problem{3e}
\begin{align*}
  \mathcal{B} &=
  \left \{
  \rowarrowsep=-2pt
  \begin{gmatrix}[p]
    1 & 0 & \dots & 0 \\
    0 & 0 & \dots & 0 \\
    \vdots & \vdots & \ddots & \vdots \\
    0 & 0 & \dots & 0 
  \end{gmatrix}
  ,
  \begin{gmatrix}[p]
    0 & 0 & \dots & 0 \\
    0 & 1 & \dots & 0 \\
    \vdots & \vdots & \ddots & \vdots \\
    0 & 0 & \dots & 0 
  \end{gmatrix}
  , \: \dots \: ,
  \begin{gmatrix}[p]
    0 & 0 & \dots & 0 \\
    0 & 0 & \dots & 0 \\
    \vdots & \vdots & \ddots & \vdots \\
    0 & 0 & \dots & 1
  \end{gmatrix}
  \right \} \\
 {} & \relation{\cup} 
      \left \{
      \rowarrowsep=-2pt
      \begin{gmatrix}[p]
        0 & 1 & \dots & 0 \\
        1 & 0 & \dots & 0 \\
        \vdots & \vdots & \ddots & \vdots \\
        0 & 0 & \dots & 0
      \end{gmatrix}
      , \: \dots \: ,
      \begin{gmatrix}[p]
        0 & \dots & 1 & 0 \\
        \vdots & \ddots & \vdots & \vdots \\
        1 & \dots & 0 & 0 \\
        0 & \dots & 0 & 0
      \end{gmatrix}
      , \: \dots \: ,
      \begin{gmatrix}[p]
        0 & \dots & 0 & 0 \\
        \vdots & \ddots & \vdots & \vdots \\
        0 & \dots & 0 & 1 \\
        0 & \dots & 1 & 0
      \end{gmatrix}
      \right \} \\
 {} & \relation{\cup}
      \left \{
      \rowarrowsep=-2pt
      \begin{gmatrix}[p]
        0 & 1 & \dots & 0 \\
        -1 & 0 & \dots & 0 \\
        \vdots & \vdots & \ddots & \vdots \\
        0 & 0 & \dots & 0
      \end{gmatrix}
      , \: \dots \: ,
      \begin{gmatrix}[p]
        0 & \dots & 1 & 0 \\
        \vdots & \ddots & \vdots & \vdots \\
        -1 & \dots & 0 & 0 \\
        0 & \dots & 0 & 0
      \end{gmatrix}
      , \: \dots \: ,
      \begin{gmatrix}[p]
        0 & \dots & 0 & 0 \\
        \vdots & \ddots & \vdots & \vdots \\
        0 & \dots & 0 & 1 \\
        0 & \dots & -1 & 0
      \end{gmatrix}
      \right \}
\end{align*}
This set is clearly a basis for $M_{n\times n}(\mathbb{R})$ and the proof is
just an expanded version of the proof given in 3c.
\begin{align*}
  \matrixRep{T}{B}{B} =
  \begin{gmatrix}[p]
    1 & 0 & 0 & \dots & 0 & 0 & 0 \\
    0 & 1 & 0 & \dots & 0 & 0 & 0 \\
    0 & 0 & 1 & \dots & 0 & 0 & 0 \\
    \vdots & \vdots & \vdots & \ddots & \vdots & \vdots & \vdots \\
    0 & 0 & 0 & \dots & -1 & 0 & 0 \\
    0 & 0 & 0 & \dots & 0 & -1 & 0 \\
    0 & 0 & 0 & \dots & 0 & 0 & -1
  \end{gmatrix}
\end{align*}
\end{document}