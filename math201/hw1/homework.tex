\documentclass{article}

\usepackage[a4paper, margin=2cm]{geometry}
\usepackage{amsmath}
\usepackage{amssymb}
\usepackage{gauss}

\setlength\parindent{0pt}

% Allow for Augmented Matricies
\usepackage{etoolbox}
\makeatletter
\patchcmd\g@matrix
 {\vbox\bgroup}
 {\vbox\bgroup\normalbaselines}% restore the standard baselineskip
 {}{}
\makeatother

\newcommand{\BAR}{%
  \hspace{-\arraycolsep}%
  \strut\vrule % the `\vrule` is as high and deep as a strut
  \hspace{-\arraycolsep}%
}

\newenvironment{augmented}
  {\left[\array{@{}rrr|r@{}}}
  {\endarray\right]}
\newenvironment{operations}
  {\array{@{}c@{}}}
  {\endarray}
\newcommand{\rowAdd}[3]{#3 r_#1 + r_#2 \rightarrow r_#2}
\newcommand{\rowScale}[2]{#2 r_#1 \rightarrow r_#1}
\newcommand{\rowSwap}[2]{r_#1 \leftrightarrow r_#2}
\newcommand{\rowEquiv}[0]{\ensuremath{\rightsquigarrow}}
\newcommand{\problem}[1]{\large\textbf{Problem #1}\normalsize}

\begin{document}

\noindent\Large\textbf{Problem Set 1: Friday Week One} \\
\normalsize
Alice McKean \\
\today \\

\problem{1a}
\begin{align*}
  M_1 &= 
  \begin{gmatrix}[b]
     1 & -2 &  1 & \BAR &  1 \\
    -4 &  2 & -1 & \BAR &  0 \\
     3 &  3 & -1 & \BAR &  1
     \rowops
     \add[4]{0}{1}
     \add[-3]{0}{2}
  \end{gmatrix}
  \rowEquiv
  \begin{gmatrix}[b]
     1 & -2 &  1 & \BAR &  1 \\
     0 & -6 &  3 & \BAR &  4 \\
     0 &  9 & -4 & \BAR & -2
     \rowops
     \add[\frac{3}{2}]{1}{2}
     \add[-\frac{1}{3}]{1}{0}
     \mult{2}{\cdot2}
  \end{gmatrix}
  \rowEquiv
  \begin{gmatrix}[b]
     1 &  0 &  0 & \BAR & -\frac{1}{3} \\
     0 & -6 &  3 & \BAR &  4 \\
     0 &  0 &  1 & \BAR &  8
     \rowops
     \add[-3]{2}{1}
     \mult{1}{\cdot-\frac{1}{6}}
  \end{gmatrix}
  \\
  &\rowEquiv
  \begin{gmatrix}[b]
     1 &  0 &  0 & \BAR & -\frac{1}{3} \\
     0 &  1 &  0 & \BAR &  \frac{10}{3} \\
     0 &  0 &  1 & \BAR &  8
  \end{gmatrix}
  = E_1
\end{align*}
The reduced echelon form shows that the first system of equations has one solution
with the following values.
\begin{align*}
  &x = -\frac{1}{3}&  
  &y = \frac{10}{3}&
  &z = 8&
\end{align*}
\problem{1b}
\begin{align*}
  M_2 &= 
  \begin{gmatrix}[b]
     1 &  1 &  3 & \BAR &  3 \\
    -1 &  1 &  1 & \BAR & -1 \\
     2 &  3 &  8 & \BAR &  4
     \rowops
     \add[]{0}{1}
     \add[-2]{0}{2}
  \end{gmatrix}
  &\rowEquiv&
  \begin{gmatrix}[b]
     1 &  1 &  3 & \BAR &  3 \\
     0 &  2 &  4 & \BAR &  2 \\
     0 &  1 &  2 & \BAR & -2
     \rowops
     \mult{1}{\cdot\frac{1}{2}}
     \add[-1]{1}{2}
  \end{gmatrix}
  \rowEquiv
  \begin{gmatrix}[b]
     1 &  1 &  3 & \BAR &  3 \\
     0 &  1 &  2 & \BAR &  1 \\
     0 &  0 &  0 & \BAR & -3
     \rowops
     \add[-1]{1}{0}
     \mult{2}{\cdot-\frac{1}{3}}
  \end{gmatrix}
  \\
  &\rowEquiv
  \begin{gmatrix}[b]
     1 &  0 &  1 & \BAR &  2 \\
     0 &  1 &  2 & \BAR &  1 \\
     0 &  0 &  0 & \BAR &  1
     \rowops
     \add[-1]{2}{1}
     \add[-2]{2}{0}
  \end{gmatrix}
  &\rowEquiv&
  \begin{gmatrix}[b]
     1 &  0 &  1 & \BAR &  0 \\
     0 &  1 &  2 & \BAR &  0 \\
     0 &  0 &  0 & \BAR &  1
  \end{gmatrix}
  = E_2
\end{align*}
The reduced echelon form shows that the second system of equations is inconsistent and
thus has no solutions. \\

\problem{1c}
\begin{align*}
  M_3 &= 
  \begin{gmatrix}[b]
     1 &  1 &  3 & \BAR &  3 \\
    -1 &  1 &  1 & \BAR & -1 \\
     2 &  3 &  8 & \BAR &  7
     \rowops
     \add[]{0}{1}
     \add[-2]{0}{2}
  \end{gmatrix}
  &\rowEquiv&
  \begin{gmatrix}[b]
     1 &  1 &  3 & \BAR &  3 \\
     0 &  2 &  4 & \BAR &  2 \\
     0 &  1 &  2 & \BAR &  1
     \rowops
     \mult{1}{\cdot\frac{1}{2}}
     \add[-1]{1}{2}
     \add[-1]{1}{0}
  \end{gmatrix}
  \rowEquiv
  \begin{gmatrix}[b]
     1 &  0 &  1 & \BAR &  2 \\
     0 &  1 &  2 & \BAR &  1 \\
     0 &  0 &  0 & \BAR &  0
  \end{gmatrix}
  = E_3
\end{align*}
The reduced echelon form shows that the third system of equations has an infinite
number of solutions. 
\begin{align*}
  \{ (5 - z, -2 - 2z, z) \: | \: z \in \mathbb{R} \}&&
  \begin{gmatrix}[p]
    x \\
    y \\
    z
  \end{gmatrix}    
  =
  \begin{gmatrix}[p]
    5 \\
    -2 \\
    0
  \end{gmatrix}    
  +
  z
  \begin{gmatrix}[p]
    -1 \\
    -2 \\
    1
  \end{gmatrix}    
\end{align*}
\problem{1d}
\begin{align*}
  M_4 &= 
  \begin{gmatrix}[b]
     2 & -2 & -3 &  0 & \BAR & -2 \\
     3 & -3 & -2 &  5 & \BAR &  7 \\
     1 & -1 & -2 & -1 & \BAR & -3
     \rowops
     \swap{0}{1}
     \swap{0}{2}
  \end{gmatrix}
  &\rowEquiv&
  \begin{gmatrix}[b]
     1 & -1 & -2 & -1 & \BAR & -3 \\
     2 & -2 & -3 &  0 & \BAR & -2 \\
     3 & -3 & -2 &  5 & \BAR &  7
     \rowops
     \add[-2]{0}{1}
     \add[-3]{0}{2}
  \end{gmatrix}
  \\
  &\rowEquiv
  \begin{gmatrix}[b]
     1 & -1 & -2 & -1 & \BAR & -3 \\
     0 &  0 &  1 &  2 & \BAR &  4 \\
     0 &  0 &  4 &  8 & \BAR &  16
     \rowops
     \add[-4]{1}{2}
     \add[2]{1}{0}
  \end{gmatrix}
  &\rowEquiv&
  \begin{gmatrix}[b]
     1 & -1 &  0 &  3 & \BAR &  5 \\
     0 &  0 &  1 &  2 & \BAR &  4 \\
     0 &  0 &  0 &  0 & \BAR &  0
  \end{gmatrix}
  = E_4
\end{align*}
The reduced echelon form shows that the fourth system of equations has an infinite
number of solutions. 
\begin{align*}
  \{ (x, y, \frac{2x - 2y + 2}{3}, \frac{-x + y + 5}{3}) \: | \: (x, y) \in \mathbb{R}^2 \}&&
  \begin{gmatrix}[p]
    x \\
    y \\
    z \\
    w
  \end{gmatrix}    
  =
  \begin{gmatrix}[p]
    0 \\
    0 \\
    2/3 \\
    5/3
  \end{gmatrix}    
  +
  x
  \begin{gmatrix}[p]
    1  \\
    0  \\ 
    2/3 \\
    -1/3 
  \end{gmatrix}    
  +
  y
  \begin{gmatrix}[p]
    0 \\
    1 \\
    -2/3 \\
    1/3
  \end{gmatrix}
\end{align*}

\newpage
\problem{2a}
\begin{align*}
  M_c &= 
  \begin{gmatrix}[b]
      1 &  1 &  1 & \BAR &  3 \\
      4 & -2 &  1 & \BAR & 15 \\
      4 &  2 &  1 & \BAR & 11
     \rowops
     \add[-1]{2}{1}
     \mult{1}{\cdot-4}
  \end{gmatrix}
  \rowEquiv
  \begin{gmatrix}[b]
      1 &  1 &  1 & \BAR &  3 \\
      0 &  1 &  0 & \BAR & -1 \\
      4 &  2 &  1 & \BAR & 11
     \rowops
     \add[-4]{0}{2}
     \add[2]{1}{2}
     \mult{2}{\cdot-3}
  \end{gmatrix}
  \rowEquiv
  \begin{gmatrix}[b]
      1 &  1 &  1 & \BAR &  3 \\
      0 &  1 &  0 & \BAR & -1 \\
      0 &  0 &  1 & \BAR &  1
     \rowops
     \add[-1]{1}{0}
     \add[-1]{2}{0}
  \end{gmatrix}
  \\
  &\rowEquiv
  \begin{gmatrix}[b]
      1 &  0 &  0 & \BAR &  3 \\
      0 &  1 &  0 & \BAR & -1 \\
      0 &  0 &  1 & \BAR &  1
  \end{gmatrix}
  = E_c
\end{align*}
The reduced echelon form gives us the coefficients of the parabola that passes
through points (−2, 15), (1, 3), and (2, 11). The full parabola is
$y = 3x^2 - x+ 1$. \\

\problem{2b} \\
Only five points in general are needed to specify a conic. This is because
conics are not unique up-to scaling and because at least one coefficient is
nonzero. This means you can divide out the nonzero coefficient from the whole
equation and reduce the number of unknowns by one. For the sake of argument let
$f$ be nonzero in the following equation: $ax^2 + bxy + cy^2 + dx + ey + f = 0$.
After diving by f the equation looks like:
$\frac{a}{f}x^2 + \frac{b}{f}xy + \frac{c}{f}y^2 + \frac{d}{f}x + \frac{e}{f}y +
1 = 0$ which can be alpha renamed to the following equation: $a'x^2 + b'xy +
c'y^2 + d'x + e'y + 1 = 0$. This equation has five unknowns so five points are
required to generate enough equations to solve for each unknown.

\end{document}