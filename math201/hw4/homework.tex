\documentclass{article}

\usepackage[a4paper, margin=2cm]{geometry}
\usepackage{amsmath}
\usepackage{amssymb}
\usepackage{gauss}

\setlength\parindent{0pt}

% Allow for Augmented Matricies
\usepackage{etoolbox}
\makeatletter
\patchcmd\g@matrix
 {\vbox\bgroup}
 {\vbox\bgroup\normalbaselines}% restore the standard baselineskip
 {}{}
\makeatother

\newcommand{\BAR}{%
  \hspace{-\arraycolsep}%
  \strut\vrule % the `\vrule` is as high and deep as a strut
  \hspace{-\arraycolsep}%
}

\newcommand{\rowEquiv}[0]{\ensuremath{\rightsquigarrow}}
\newcommand{\problem}[1]{\large\textbf{Problem #1}\normalsize}

\newcommand{\evidence}[1]{\ensuremath{(\hspace{0.2em} \text{#1} \hspace{0.2em})}}
\newcommand{\relation}[1]{\ensuremath{\hspace{0.2em} {{} #1 {}} \hspace{0.2em}}}
\newcommand{\equal}{\relation{=}}
\newcommand{\qed}{\hfill\ensuremath{\square}}

\begin{document}

\noindent\Large\textbf{Problem Set Week 3 Tuesday} \\
\normalsize
Alice McKean \\
\today \\

\problem{1a}

If $v \in \text{span}(S)$ then $v$ can be written as the following linear combination: 
\begin{alignat*}{3}
  & x^3 - 8x^2 + 7x - 8 & \equal & a(x^3 + 3x - 2) + b(x^3 + 4x^2 + x + 1) + c(x^2 + 2x + 4) \\
  & {}                  & \equal & ax^3 + 3ax - 2a + bx^3 + 4bx^2 + bx + b + cx^2 + 2cx + 4c \\
  & {}                  & \equal & (a + b)x^3 + (4b + c)x^2 + (3a + b + 2c)x + (-2a + b + 4c)
\end{alignat*}
Which generates the following system of equations:
\begin{align*}
  M &= 
  \rowarrowsep=-2pt
  \begin{gmatrix}[b]
     1 &  1 &  0 & \BAR &  1 \\
     0 &  4 &  1 & \BAR & -8 \\
     3 &  1 &  2 & \BAR &  7 \\
    -2 &  1 &  4 & \BAR & -8
    \rowops
    \add[-3]{0}{2}
    \add[2]{0}{3}
  \end{gmatrix}
  \rowEquiv
  \begin{gmatrix}[b]
     1 &  1 &  0 & \BAR &  1 \\
     0 &  4 &  1 & \BAR & -8 \\
     0 & -2 &  2 & \BAR &  4 \\
     0 &  3 &  4 & \BAR & -6
    \rowops
    \add[-1]{3}{1}
  \end{gmatrix}
  \rowEquiv                         
  \begin{gmatrix}[b]
     1 &  1 &  0 & \BAR &  1 \\
     0 &  1 & -3 & \BAR & -2 \\
     0 & -2 &  2 & \BAR &  4 \\
     0 &  3 &  4 & \BAR & -6
    \rowops
    \add[-1]{1}{0}
    \add[2]{1}{2}
    \add[-3]{1}{3}
  \end{gmatrix}
  \\
  &\rowEquiv
  \begin{gmatrix}[b]
     1 &  0 &  3 & \BAR &  3 \\
     0 &  1 & -3 & \BAR & -2 \\
     0 &  0 & -4 & \BAR &  0 \\
     0 &  0 & -5 & \BAR &  0
    \rowops
    \add[-1]{3}{2}
  \end{gmatrix}
  \rowEquiv
  \begin{gmatrix}[b]
     1 &  0 &  3 & \BAR &  3 \\
     0 &  1 & -3 & \BAR & -2 \\
     0 &  0 &  1 & \BAR &  0 \\
     0 &  0 & -5 & \BAR &  0
    \rowops
    \add[5]{2}{3}
    \add[3]{2}{1}
    \add[-3]{2}{0}
  \end{gmatrix}
  \rowEquiv
  \begin{gmatrix}[b]
     1 &  0 &  0 & \BAR &  3 \\
     0 &  1 &  0 & \BAR & -2 \\
     0 &  0 &  1 & \BAR &  0 \\
     0 &  0 &  0 & \BAR &  0
  \end{gmatrix}
  = E
\end{align*}
So $v \in \text{span}(S)$ because $v = 3(x^3 + 3x - 2) - 2(x^3 + 4x^2 + x + 1) +
0(x^2 + 2x + 4)$. \qed \\

\problem{1b}

If $v \in \text{span}(S)$ then $v$ can be written as the following linear combination: 
\begin{alignat*}{3}
  & \rowarrowsep=-2pt
    \begin{gmatrix}[b]
      9  &  10 \\
      10 &  9
    \end{gmatrix} 
  & \equal
  & \rowarrowsep=-2pt
    a \cdot
    \begin{gmatrix}[b]
      1 &  1 \\
      1 & -2
    \end{gmatrix} 
    + b \cdot
    \begin{gmatrix}[b]
     -1 &  2 \\
      1 &  2
    \end{gmatrix} 
    + c \cdot
    \begin{gmatrix}[b]
      2 &  4 \\
      3 &  5
    \end{gmatrix} 
  \\
  & {}
  & \equal
  & \rowarrowsep=-2pt
    \begin{gmatrix}[b]
      a &  a \\
      a & -2a
    \end{gmatrix} 
    +
    \begin{gmatrix}[b]
     -b &  2b \\
      b &  2b
    \end{gmatrix} 
    +
    \begin{gmatrix}[b]
      2c &  4c \\
      3c &  5c
    \end{gmatrix} 
  \\
  & {}
  & \equal
  & \rowarrowsep=-2pt
    \begin{gmatrix}[b]
      a - b + 2c &  a + 2b + 4c \\
      a + b + 3c & -2a + 2b + 5c
    \end{gmatrix} 
\end{alignat*}
Which generates the following system of equations:
\begin{align*}
  M &= 
  \rowarrowsep=-2pt
  \begin{gmatrix}[b]
     1 & -1 &  0 & \BAR &  1 \\
     0 &  4 &  1 & \BAR & -8 \\
     3 &  1 &  2 & \BAR &  7 \\
    -2 &  1 &  4 & \BAR & -8
  \end{gmatrix}
  \rowEquiv
  \begin{gmatrix}[b]
     1 &  0 &  0 & \BAR &  0 \\
     0 &  1 &  0 & \BAR &  0 \\
     0 &  0 &  1 & \BAR &  0 \\
     0 &  0 &  0 & \BAR &  1
  \end{gmatrix}
  = E
\end{align*}
The reduced row echelon form above shows that the system of equations is inconsistent which means that $v$ can't be
written as a linear combination of $S$ so $v \notin \text{span}(S)$. \qed \\

\problem{2a}

The set $S' = \{ (1, 0, 0), (0, 1, 0), (0, 0, 1) \}$ clearly generates
$\mathbb{Q}^3$ as $(x, y, z) = x\cdot(1, 0, 0) + y\cdot(0, 1, 0) + z\cdot(0, 0,
1)$. You can write every element of $S'$ as a linear combination of elements in
$S$. 
\begin{align*}
% = {(1, 1, 0),(1, 0, 1),(0, 1, 1)} 
  (1, 0, 0) &= \frac{1}{2}\cdot((1, 1, 0) + (1, 0, 1) - (0, 1, 1)) \\
  (0, 1, 0) &= \frac{1}{2}\cdot((1, 1, 0) + (0, 1, 1) - (1, 0, 1)) \\
  (0, 0, 1) &= \frac{1}{2}\cdot((1, 0, 1) + (0, 1, 1) - (1, 1, 0)) 
\end{align*}
These facts and the fact that linear combinations compose imply that $S$ generates $\mathbb{Q}^3$. \qed \\

\problem{2b}

$\mathbb{F}_2$ does not generate $\mathbb{F}_2^3$ because
$(1, 1, 1) \in \mathbb{F}_2^3$ but $(1, 1, 1) \notin \text{span}(S)$.  
If $(1, 1, 1) \in \text{span}(S)$ then $(1, 1, 1)$ can be rewritten as the
following linear combination:
\begin{alignat*}{3}
  & (1, 1, 1) & \equal & a\cdot(1, 1, 0) + b\cdot(1, 0, 1) + c\cdot(0, 1, 1) \\
  & {}        & \equal & (a, a, 0) + (b, 0, b) + (0, c, c) \\
  & {}        & \equal & (a + b, a + c, b + c)
\end{alignat*}
Which generates the following system of equations:
\begin{align*}
  M &= 
  \rowarrowsep=-2pt
  \begin{gmatrix}[b]
     1 & 1 & 0 & \BAR & 1 \\
     1 & 0 & 1 & \BAR & 1 \\
     0 & 1 & 1 & \BAR & 1
     \rowops
     \add[]{0}{1}
  \end{gmatrix}
  \rowEquiv
  \begin{gmatrix}[b]
     1 & 1 & 0 & \BAR & 1 \\
     0 & 1 & 1 & \BAR & 0 \\
     0 & 1 & 1 & \BAR & 1
     \rowops
     \add[]{1}{0}
     \add[]{1}{2}
  \end{gmatrix}
  \rowEquiv                       
  \begin{gmatrix}[b]
     1 & 0 & 1 & \BAR & 1 \\
     0 & 1 & 1 & \BAR & 0 \\
     0 & 0 & 0 & \BAR & 1
     \rowops
     \add[]{2}{0}
  \end{gmatrix}
  \rowEquiv                       
  \begin{gmatrix}[b]
     1 & 0 & 1 & \BAR & 0 \\
     0 & 1 & 1 & \BAR & 0 \\
     0 & 0 & 0 & \BAR & 1
  \end{gmatrix}
  = E
\end{align*}
The reduced row echelon form above shows that the system of equations is
inconsistent which means that $(1, 1, 1)$ can't be
written as a linear combination of $S$. This means $\text{span}(S) \neq \mathbb{F}_2^3$ so $S$
does not generate $\mathbb{F}_2^3$. \qed \\

\problem{3}

This proof trivally follows from the definitions and the associative property of
the vector space.

The forward case of the biconditional
$\text{span}(S_1 \cup S_2) \subseteq \text{span}(S_1)
+ \text{span}(S_2)$. Proof let $v \in \text{span}(S_1 \cup S_2)$:
\begin{alignat*}{3}
  & v  & \equal & c_1s_1 + c_2s_2 + \dots + c_ns_n \\
  & {} & {}     & \evidence{$v \in \text{span}(S_1 \cup S_2)$ where $c_i \in F, \: s_i \in (S_1 \cup S_2)$} \\  
  & {} & \equal & (c_1s_1 + c_2s_2 + \dots + c_ks_k) +
                  (c_{k + 1}t_{k + 1} + c_{k + 2}t_{k+2} + \dots + c_nt_n) \\
  & {} & {}     & \evidence{group like terms where $s_i \in S_1, \: t_i \in S_2$} 
\end{alignat*} 
The backward case of the biconditional
$\text{span}(S_1) + \text{span}(S_2) \subseteq \text{span}(S_1 \cup S_2)$.
Proof trival. \qed
\end{document}