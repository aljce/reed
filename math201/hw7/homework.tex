\documentclass{article}

\usepackage[a4paper, margin=2cm]{geometry}
\usepackage{amsmath}
\usepackage{amssymb}
\usepackage{gauss}
\usepackage[inline]{enumitem}

\setlength\parindent{0pt}

\newcommand{\problem}[1]{\large\textbf{Problem #1}\normalsize}

\newcommand{\evidence}[1]{\ensuremath{(\hspace{0.2em} \text{#1} \hspace{0.2em})}}
\newcommand{\relation}[1]{\ensuremath{\hspace{0.2em} {{} #1 {}} \hspace{0.2em}}}
\newcommand{\equal}{\relation{=}}
\newcommand{\qed}{\hfill\ensuremath{\square}}

\begin{document}

\noindent\Large\textbf{Problem Set Week 5 Tuesday} \\
\normalsize
Alice McKean \\
\today \\

\problem{1a}

$W$ is a subspace because $\vec{0}_{3\times3}$ is clearly symmetric and because
symmetry is preserved under addition and scalar multiplication. This fact is
demonstrated below.
\begin{equation*}
  \rowarrowsep=-2pt
  \begin{gmatrix}[p]
    a_1 & b_1 & c_1 \\
    b_1 & d_1 & e_1 \\
    c_1 & e_1 & f_1
  \end{gmatrix} 
  + r \cdot
  \begin{gmatrix}[p]
    a_2 & b_2 & c_2 \\
    b_2 & d_2 & e_2 \\
    c_2 & e_2 & f_2
  \end{gmatrix} 
  =
  \begin{gmatrix}[p]
    a_1 + r a_2 & b_1 + r b_2 & c_1 + r c_2 \\
    b_1 + r b_2 & d_1 + r d_2 & e_1 + r e_2 \\
    c_1 + r c_2 & e_1 + r e_2 & f_1 + r f_2
  \end{gmatrix} 
\end{equation*}

\problem{1b}

The following equation shows that $B$ spans $W$. It is also clearly
linearly independent. These facts imply $B$ is a basis for $W$.
\begin{equation*}
  \rowarrowsep=-2pt
  \begin{gmatrix}[p]
    a & b & c \\
    b & d & e \\
    c & e & f
  \end{gmatrix} 
  = a
  \begin{gmatrix}[p]
    1 & 0 & 0 \\
    0 & 0 & 0 \\
    0 & 0 & 0
  \end{gmatrix} 
  + d
  \begin{gmatrix}[p]
    0 & 0 & 0 \\
    0 & 1 & 0 \\
    0 & 0 & 0
  \end{gmatrix} 
  + f
  \begin{gmatrix}[p]
    0 & 0 & 0 \\
    0 & 0 & 0 \\
    0 & 0 & 1
  \end{gmatrix} 
  + b
  \begin{gmatrix}[p]
    0 & 1 & 0 \\
    1 & 0 & 0 \\
    0 & 0 & 0
  \end{gmatrix} 
  + c
  \begin{gmatrix}[p]
    0 & 0 & 1 \\
    0 & 0 & 0 \\
    1 & 0 & 0
  \end{gmatrix} 
  + e
  \begin{gmatrix}[p]
    0 & 0 & 0 \\
    0 & 0 & 1 \\
    0 & 1 & 0
  \end{gmatrix} 
\end{equation*} 

\begin{equation*}
  \rowarrowsep=-2pt
  B = \left \{ \:
  \begin{gmatrix}[p]
    1 & 0 & 0 \\
    0 & 0 & 0 \\
    0 & 0 & 0
  \end{gmatrix} 
  , \:
  \begin{gmatrix}[p]
    0 & 0 & 0 \\
    0 & 1 & 0 \\
    0 & 0 & 0
  \end{gmatrix} 
  , \:
  \begin{gmatrix}[p]
    0 & 0 & 0 \\
    0 & 0 & 0 \\
    0 & 0 & 1
  \end{gmatrix} 
  , \:
  \begin{gmatrix}[p]
    0 & 1 & 0 \\
    1 & 0 & 0 \\
    0 & 0 & 0
  \end{gmatrix} 
  , \:
  \begin{gmatrix}[p]
    0 & 0 & 1 \\
    0 & 0 & 0 \\
    1 & 0 & 0
  \end{gmatrix} 
  , \:
  \begin{gmatrix}[p]
    0 & 0 & 0 \\
    0 & 0 & 1 \\
    0 & 1 & 0
  \end{gmatrix} 
  \: \right \}
\end{equation*} 

\problem{1c}

The cardinally of the basis $B$ given in 1b is 6 so $\text{dim}(W) = 6$. \\

\problem{2a}

The reduced echelon form of $M$ is:
\begin{equation*}
  \rowarrowsep=-2pt
  M =
  \begin{gmatrix}[p]
      8 & -53 & -32 & -23 \\
    -14 &  92 &  56 &  40 \\
      6 & -39 & -24 & -17 \\
     -1 &   7 &   4 &   3     
  \end{gmatrix} 
  \rightsquigarrow
  \begin{gmatrix}[p]
    1 & 0 & -4 & -2/3 \\
    0 & 1 &  0 & 1/3  \\
    0 & 0 &  0 & 0 \\
    0 & 0 &  0 & 0
  \end{gmatrix} 
  = E
\end{equation*} 

\problem{2b}

The basis for the row space of $A$ is $\{ \: (1, 0, -4, -2/3), \: (0, 1, 0, 1/3)
\: \}$ \\

The basis for the column space of $A$ is
$
\left \{
\rowarrowsep=-2pt
\begin{gmatrix}[p]
  8 \\
  -14 \\
  6 \\
  -1
\end{gmatrix} 
,
\begin{gmatrix}[p]
  -53 \\
  92 \\
  -39 \\
  7
\end{gmatrix} 
\right \}
$

\problem{3a}

In order to determine $f$ we first try to write $f$ as a general linear
transformation $f(x,y) = x \cdot (a, b, c) + y \cdot (d, e, f)$. We know
what $f$ should evaluate to at $(2, 1)$ and $(-1, 2)$ so we can work backwards to
solve for the unknowns.
$
\begin{cases}
  (0, -1, 3) = 2 \cdot (a, b, c) + 1 \cdot (d, e, f), \\
  (1, 0, 4)  = -1 \cdot (a, b, c) + 2 \cdot (d, e, f) 
\end{cases}
\rightsquigarrow
\begin{cases}
  a = -\frac{1}{5}, \:\: b = -\frac{2}{5}, \:\: c = \frac{2}{5}, \\
  d = \frac{2}{5}, \:\:\:\:\:  e = -\frac{1}{5}, \:\: f = \frac{11}{5} \\
\end{cases} \\
$ \\
Therefore
$f(x,y) = x \cdot (-\frac{1}{5}, -\frac{2}{5}, \frac{2}{5}) +
y \cdot (\frac{2}{5},  -\frac{1}{5}, \frac{11}{5} )$
extends the basis $\{ (2, 1), (-1, 2) \}$ lineraly. \\

\problem{3b}

In order to determine $f$ we first try to write $f$ as a general linear
transformation $f(x,y,z) = x \cdot (a, b) + y \cdot (c, d) + z \cdot (e, f)$. We know
what $f$ should evaluate to at $(2, 1, 1)$, $(1, 3, 2)$ and $(-1, 7, 4)$
so we can work backwards to solve for the unknowns. \vspace{0.2em} \\
$
\begin{cases}
  (2, 3) = 2 \cdot (a, b) + 1 \cdot (c, d) + 1 \cdot (e, f), \\
  (6, 2) = 1 \cdot (a, b) + 3 \cdot (c, d) + 2 \cdot (e, f), \\
  (14, 0) = -1 \cdot (a, b) + 7 \cdot (c, d) + 4 \cdot (e, f)
\end{cases}
\rightsquigarrow
\begin{cases}
  a = -\frac{s}{5}, \:\:\:\:\:\:\ b = \frac{7}{5} - \frac{t}{5} \\
  c = 2 - \frac{3s}{5}, \:\: d = \frac{1}{5} - \frac{3t}{5}  \\
  e = s, \:\:\:\:\:\:\:\:\:\:\:\: f = t \\
\end{cases} \vspace{0.2em}\\
$
This implies there are infinite linear transformations that satisfy the
constraints. This agrees with the fact that the set
$\{ (2, 1, 1), (1, 3, 2), (-1, 7, 4)\}$ is linearly dependent. \\ One such linear
transforamtion is:
$f(x,y,z) = x \cdot (0, \frac{7}{5}) + y \cdot (2, \frac{1}{5}) + z \cdot (0, 0)$
\end{document}