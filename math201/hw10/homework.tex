\documentclass{article}

\usepackage[a4paper, margin=2cm]{geometry}
\usepackage{amsmath}
\usepackage{amssymb}
\usepackage{gauss}
\usepackage[inline]{enumitem}

\setlength\parindent{0pt}

\newcommand{\problem}[1]{\large\textbf{Problem #1}\normalsize}

\newcommand{\evidence}[1]{\ensuremath{(\hspace{0.2em} \text{#1} \hspace{0.2em})}}
\newcommand{\relation}[1]{\ensuremath{\hspace{0.2em} {{} #1 {}} \hspace{0.2em}}}
\newcommand{\equal}{\relation{=}}
\newcommand{\qed}{\hfill\ensuremath{\square}}


\begin{document}

\noindent\Large\textbf{Problem Set Week 7 Tuesday} \\
\normalsize
Alice McKean \\
\today \\

\problem{1a}
\begin{align*}
  \displaystyle
  \text{tr}(AB)
     & \equal \sum_{i=1}^{n} (AB)_{ii}                  && \evidence{Definition of tr} \\
  {} & \equal \sum_{i=1}^{n} \sum_{k=1}^{n} A_{ik}B_{ki} && \evidence{Definition of matrix multiplication} \\
  {} & \equal \sum_{i=1}^{n} \sum_{k=1}^{n} B_{ki}A_{ik} && \evidence{Commutativity of multiplication} \\
  {} & \equal \sum_{k=1}^{n} \sum_{i=1}^{n} B_{ki}A_{ik} && \evidence{Commutativity of vector addition} \\
  {} & \equal \sum_{i=1}^{n} \sum_{k=1}^{n} B_{ik}A_{ki} && \evidence{Rename indices} \\
  {} & \equal \sum_{i=1}^{n} (BA)_{ii}                  && \evidence{Definition of matrix multiplication} \\
  {} & \equal \text{tr}(BA)                           && \evidence{Definition of tr} \\
\end{align*}

\problem{1b}

$
\begin{gmatrix}[b]
  1 & 0 & 0 & 1
\end{gmatrix} 
$ \\

\problem{2}

\begin{enumerate}[label=(\alph*)]
\item
  $
  \begin{gmatrix}[p]
    13 & 3  & 3 \\
    -2 & -1 & -8
  \end{gmatrix} 
  $
\item
  $
  \begin{gmatrix}[p]
    20 & 27 & 18 \\
    5 & -2  & 8
  \end{gmatrix} 
  $
\item Not possible $C$ has more columns than $A$.
\item
  $
  \begin{gmatrix}[p]
    29 \\
    -26
  \end{gmatrix} 
  $
\item The same as (d) by associativity.
\item Not possible the number of columns in $A$ doesn't equal the number of rows
  in $D$.
\end{enumerate}

\problem{3}


Assume that $J \in \mathcal{M}_{n \times n}$ such that for any $A \in
\mathcal{M}_{m \times n}$ the following equation holds: $AJ = A$. Let $A = I_n$
so by assumption $I_nJ = I_n$ but $I_n$ is a left unit for matrix
multiplication so
$J = I_n$. $\qed$

\end{document}