\documentclass{article}

\usepackage[a4paper, margin=2cm]{geometry}
\usepackage{amsmath}
\usepackage{amssymb}
\usepackage{gauss}
\usepackage[inline]{enumitem}

\setlength\parindent{0pt}

\newcommand{\problem}[1]{\large\textbf{Problem #1}\normalsize}

\newcommand{\evidence}[1]{\ensuremath{(\hspace{0.2em} \text{#1} \hspace{0.2em})}}
\newcommand{\relation}[1]{\ensuremath{\hspace{0.2em} {{} #1 {}} \hspace{0.2em}}}
\newcommand{\equal}{\relation{=}}
\newcommand{\qed}{\hfill\ensuremath{\square}}

\begin{document}

\noindent\Large\textbf{Problem Set Week 4 Tuesday} \\
\normalsize
Alice McKean \\
\today \\

\problem{1a}

$[ \: (11, 1) \: ]_{B}$ evaluates to $\langle c_1, c_2 \rangle$ such that
$(11, 1) = c_1\cdot(1, 2) + c_2\cdot(-2, 3)$. This generates a system of
equations: \\
\begin{cases}
  11 = c_1 - 2c_2  \\
  1 \:\: = 2c_1 + 3c_2 
\end{cases}
$\rightarrow\:\:\:\:$
\begin{cases}
  c_1 = 5 \\
  c_2 = 3
\end{cases} \\

This means that $[ \: (11, 1) \: ]_{B}$ evaluates to
$\langle 5, 3 \rangle$. \\

\problem{1b}

In this case $B$ is the standard basis for $\mathbb{R}^2$ so
$[ \: (11, 1) \: ]_{B}$ evaluates to
$\langle 11, 1 \rangle$. \\

\problem{1c}

$[ \: x^2 + 2 \: ]_{B}$ evaluates to $\langle c_1, c_2, c_3 \rangle$ such that
$x^2 + 2 = c_1 \cdot 1 + c_2 \cdot (x - 1) + c_3 \cdot (x - 1)^2$. This generates a system of
equations: \\
\begin{cases}
  2 = c_1 - c_2 + c_3 \\
  0 = c_2 - 2c_3 \\
  1 = c_3
\end{cases}
$\rightarrow\:\:\:\:$
\begin{cases}
  c_1 = 3 \\
  c_2 = 2 \\
  c_3 = 1
\end{cases} \\

This means that $[ \: x^2 + 2 \: ]_{B}$ evaluates to
$\langle 3, 2, 1 \rangle$. \\

\problem{1d}

In this case $B$ is the standard basis for $\mathcal{P}_3(\mathbb{R})$ so
$[ \: x^2 + 2 \: ]_{B}$ evaluates to
$\langle 2, 0, 1, 0 \rangle$. \\

\problem{1e} \\

$
\left [
\rowarrowsep=-2pt
\begin{gmatrix}[p]
  5 & 1 \\
  8 & -3
\end{gmatrix}
\right ]_B
$
evaluates to $\langle c_1, c_2 \rangle$ such that
$
\rowarrowsep=-2pt
\begin{gmatrix}[p]
  5 & 1 \\
  8 & -3
\end{gmatrix}
= c_1 \cdot
\begin{gmatrix}[p]
  1 & 1 \\
  2 & 1
\end{gmatrix}
+ c_2 \cdot
\begin{gmatrix}[p]
  -1 & 1 \\
  -1 & 3
\end{gmatrix}
$ \\
This generates a system of equations: \\
\begin{cases}
  \:\:\: 5 = c_1 - c_2 \\
  \:\:\: 1 = c_1 + c_2 \\
  \:\:\: 8 = 2c_1 - c_2 \\
  -3 = c_1 + 3c_2
\end{cases}
$\rightarrow\:\:\:\:$
\begin{cases}
  c_1 = 3 \\
  c_2 = -2 \\
\end{cases} \\
This means that 
$
\left [
\rowarrowsep=-2pt
\begin{gmatrix}[p]
  5 & 1 \\
  8 & -3
\end{gmatrix}
\right ]_B
$
evaluates to
$\langle 3, -2 \rangle$. \\

\problem{2}

\begin{enumerate*}[label=(\alph*)]
  \item False
  \item False
  \item True
  \item False
  \item True
\end{enumerate*} \\

\problem{3}

A good way to conceptualize this problem is to realize that $\mathbb{R}^x$ is
isomorphic to $\mathbb{R}^3$. This is a trivial result of the curry-howard
correspondence. In laymen terms you can think of any value in $\mathbb{R}^x$ as
a coordinate in $\mathbb{R}^3$ where you're waiting pick which values you need
from x, y, and z. \\
The following functions are clearly equal at every possible value in their
domain so they are equal.
\begin{enumerate}[label=(\alph*),leftmargin=*]
  \item $f = 5 \cdot \chi_1 + \pi \cdot \chi_2 + -7 \cdot \chi_3$ 
  \item $g = g(1) \cdot \chi_1 + g(2) \cdot \chi_2 + g(3) \cdot \chi_3$
  \item Assume $a\chi_1 + b\chi_2 + c\chi_3 = z$ and apply $1$ to both sides to get
        $(a\chi_1 + b\chi_2 + c\chi_3)(1) = z(1)$. After simplification based on the
        definitions of $+$, $\chi_i$, and $z$ you get $a = 0$. Repeat this process for $2$ and $3$
        to generate $b = 0$ and $c = 0$ respectively. \qed
\end{enumerate}

\end{document}