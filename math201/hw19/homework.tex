\documentclass[fleqn]{article}

\usepackage[a4paper, margin=2cm]{geometry}
\usepackage{amsmath}
\usepackage{amssymb}
\usepackage{gauss}
\usepackage[inline]{enumitem}
\usepackage{tikz}

\setlength{\parindent}{0pt}
\setlength{\mathindent}{0pt}

% Allow for Augmented Matricies
\usepackage{etoolbox}
\makeatletter
\patchcmd\g@matrix
 {\vbox\bgroup}
 {\vbox\bgroup\normalbaselines}% restore the standard baselineskip
 {}{}
\makeatother

\newcommand{\BAR}{%
  \hspace{-\arraycolsep}%
  \strut\vrule % the `\vrule` is as high and deep as a strut
  \hspace{-\arraycolsep}%
}

\newcommand{\squig}[0]{\ensuremath{\rightsquigarrow}}

\newcommand{\problem}[1]{\large\textbf{Problem #1}\normalsize}

\newcommand{\evidence}[1]{\ensuremath{(\hspace{0.2em} \text{#1} \hspace{0.2em})}}
\newcommand{\relation}[1]{\ensuremath{\hspace{0.2em} {{} #1 {}} \hspace{0.2em}}}
\newcommand{\equal}{\relation{=}}
\newcommand{\qed}{\hfill\ensuremath{\square}}

\newcommand{\idF}[1]{\ensuremath{\text{id}(#1)}}
\newcommand{\coordsF}[2]{\ensuremath{[ \: #1 \: ]_{\mathcal{#2}}}}
\newcommand{\rankF}[1]{\ensuremath{\text{rank}(#1)}}
\newcommand{\nullityF}[1]{\ensuremath{\text{nullity}(#1)}}
\newcommand{\matrixRep}[3]{\ensuremath{{\left [ \: #1 \: \right ]}_{\mathcal{#2}}^{\mathcal{#3}}}}
\newcommand{\signF}[1]{\ensuremath{\text{sign}(#1)}}
\usepackage{listofitems}
% Cycle Notation
\newcommand\cycleF[2][\:]{
  \readlist\thecycle{#2}
  #1\foreachitem\i\in\thecycle{\ifnum\icnt=1\else#1\fi\i}#1
}
\newcommand{\normF}[1]{\left\lVert#1\right\rVert}

\begin{document}

\noindent\Large\textbf{Problem Set Week 12 Friday} \\
\normalsize
Alice McKean \\
\today \\

\problem{1}

The problem simplifies to:
\begin{align*}
  S^\bot =
  \left \{ \, x \in \mathbb{C}^3 \, | \, \left \langle \, x, (1, 0, i) \, \right \rangle , \,
                                         \left \langle \, x, (1, 2, 1) \, \right \rangle \, \right \}
\end{align*}
Which generates the following system of equations:
\begin{align*}
\begin{cases}
  0 = \left \langle \, (a, b, c), (1, 0, i) \, \right \rangle \\
  0 = \left \langle \, (a, b, c), (1, 2, 1) \, \right \rangle
\end{cases}
\squig
\begin{cases}
  0 = a - i c \\
  0 = a + 2b + c
\end{cases}
\squig
\begin{cases}
  a = i c \\
  b = \dfrac{-1 - i}{2} c
\end{cases}
\end{align*}
Therefore
$S^\bot =
\text{span}\left (\left \{ \left (i, \dfrac{-1 - i}{2}, 1 \right ) \right \} \right)$. \\

\problem{2}

Let $x \in S$ and $y \in S^\bot$ so $\langle y, x \rangle = 0$. Take the
conjugate of both sides to get $\langle x, y \rangle = 0$ therefore $x \in
(S^\bot)^\bot$. \\ 

% For any $x \in S$ the equality $\langle x, y \rangle = 0$ holds for any $y \in
% S^\bot$ as all the elements of $S^\bot$ are perpendicular to $S$. Therefore $x \in (S^\bot)^\bot$. \\

Assume $V$ is a finite dimensional inner product space and that $W$ is a
subspace of $V$.
\begin{align*}
  \text{dim} V &= \text{dim} W + \text{dim} W^\bot 
               && \evidence{$W$ is a subspace of $V$} \\
               &\relation{\leq} \text{dim} (W^\bot)^\bot + \text{dim} W^\bot
               && \evidence{By the previous question $W \subseteq (W^\bot)^\bot$ as $W \subseteq V$} \\
               &= \text{dim} W^\bot + \text{dim} (W^\bot)^\bot
               && \evidence{Communitivity} \\
               &= \text{dim} V
               && \evidence{$W^\bot$ is a subspace of $V$}
\end{align*}
The facts above imply
$\text{dim} W = \text{dim} (W^\bot)^\bot$ and by the previous question $W
\subseteq (W^\bot)^\bot$ therefore $W = (W^\bot)^\bot$. \\

\problem{3}

Apply Gram-Schmidt orthogonalization to the basis $\{ 1, x \kern0.05em \}$:
\begin{align*}
  v_1 &= 1 \\
  v_2 &= x - \dfrac{\langle \kern0.1em x, v_1 \rangle \cdot v_1}{\normF{\kern0.05em v_1}}
       = x - \dfrac{\langle x, 1 \rangle \cdot 1}{\normF{\kern0.1em 1}} 
       = x - \dfrac{1}{2}
\end{align*}
Now normalize the basis above:
\begin{align*}
  \displaystyle
  u_1 &= \dfrac{1}{\normF{1}} = 1 \\
  u_2 &= \dfrac{x - \dfrac{1}{2}}{\kern0.05em \normF{x - \dfrac{1}{2} \kern0.05em}}
       = \dfrac{x - \dfrac{1}{2}}{\sqrt{\displaystyle\int_0^1 \left (x - \dfrac{1}{2} \right )^2 dx}}
       = 2\sqrt{3}x - \sqrt{3}
\end{align*}
This vector is the closest vector to $h(x)$ under the given dot product
as $\{ u_1, u_2 \}$ is an othornormal basis:
\begin{align*}
  w &= \langle \, h(x), u_1 \kern0.1em \rangle \cdot u_1 + \langle \, h(x), u_2 \kern0.1em \rangle \cdot u_2 \\
    &= \int_0^1 \left (4 + 3x - 2x^2 \right )1 \, dx \cdot u_1 +
       \int_0^1 \left (4 + 3x - 2x^2 \right ) \left (2\sqrt{3}x - \sqrt{3} \kern0.15em \right ) \, dx \cdot u_2 \\
    &= \dfrac{29}{6} \cdot u_1 +
       \dfrac{1}{2\sqrt{3}} \cdot u_2
\end{align*}
The vector given above can be written using the standard basis:
\begin{align*}
  w &= \dfrac{29}{6} \cdot u_1 +
       \dfrac{1}{2\sqrt{3}} \cdot u_2 \\
    &= \dfrac{29}{6} \cdot 1 +
       \dfrac{1}{2\sqrt{3}} \cdot \left (2\sqrt{3}x - \sqrt{3} \kern0.15em \right ) \\
    &= \dfrac{29}{6} +
       x - \dfrac{1}{2} \\
    &= \dfrac{13}{3} \cdot 1 +
       1 \cdot x
\end{align*}

\end{document}