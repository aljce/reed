\documentclass[fleqn]{article}

\usepackage[a4paper, margin=2cm]{geometry}
\usepackage{amsmath}
\usepackage{amssymb}
\usepackage{gauss}
\usepackage[inline]{enumitem}

\setlength{\parindent}{0pt}
\setlength{\mathindent}{0pt}

% Allow for Augmented Matricies
\usepackage{etoolbox}
\makeatletter
\patchcmd\g@matrix
 {\vbox\bgroup}
 {\vbox\bgroup\normalbaselines}% restore the standard baselineskip
 {}{}
\makeatother

\newcommand{\BAR}{%
  \hspace{-\arraycolsep}%
  \strut\vrule % the `\vrule` is as high and deep as a strut
  \hspace{-\arraycolsep}%
}

\newcommand{\squig}[0]{\ensuremath{\rightsquigarrow}}

\newcommand{\problem}[1]{\large\textbf{Problem #1}\normalsize}

\newcommand{\evidence}[1]{\ensuremath{(\hspace{0.2em} \text{#1} \hspace{0.2em})}}
\newcommand{\relation}[1]{\ensuremath{\hspace{0.2em} {{} #1 {}} \hspace{0.2em}}}
\newcommand{\equal}{\relation{=}}
\newcommand{\qed}{\hfill\ensuremath{\square}}

\newcommand{\idF}[1]{\ensuremath{\text{id}(#1)}}
\newcommand{\coordsF}[2]{\ensuremath{[ \: #1 \: ]_{\mathcal{#2}}}}
\newcommand{\rankF}[1]{\ensuremath{\text{rank}(#1)}}
\newcommand{\nullityF}[1]{\ensuremath{\text{nullity}(#1)}}
\newcommand{\matrixRep}[3]{\ensuremath{[ \: #1 \: ]_{\mathcal{#2}}^{\mathcal{#3}}}}
\newcommand{\signF}[1]{\ensuremath{\text{sign}(#1)}}
\usepackage{listofitems}
% Cycle Notation
\newcommand\cycle[2][\:]{
  \readlist\thecycle{#2}
  #1\foreachitem\i\in\thecycle{\ifnum\icnt=1\else#1\fi\i}#1
}
\begin{document}

\noindent\Large\textbf{Problem Set Week 9 Tuesday} \\
\normalsize
Alice McKean \\
\today \\

\problem{1a}

The following uses Cauchy one line permutation notation so the permutation $(\cycle{3,2,1})$
represents the following function:
$\pi(1) = 3 \:\:\: \pi(2) = 2 \:\:\: \pi(3) = 1$.
\begin{align*}
  \rowarrowsep=-2pt
  \det{
  \begin{gmatrix}[p]
    0 & 2 & -3 \\
    -2 & 0 & 1 \\
    3 & -1 & 0
  \end{gmatrix} 
  }
  & \equal
    0 \cdot 0 \cdot 0 \cdot \signF{\cycle{1,2,3}} \: + \:
    0 \cdot 1 \cdot -1 \cdot \signF{\cycle{1,3,2}} \\
 {} & \: + \:
    2 \cdot -2 \cdot 0 \cdot \signF{\cycle{2,1,3}} \: + \:
    2 \cdot 1 \cdot 3 \cdot \signF{\cycle{2,3,1}} \\
 {} & \: + \:
    -3 \cdot -2 \cdot -1 \cdot \signF{\cycle{3,1,2}} \: + \:
    -3 \cdot 0 \cdot 3 \cdot \signF{\cycle{3,2,1}} \\
 {} & \equal
    2 \cdot 1 \cdot 3 \cdot \signF{\cycle{2,3,1}} \: - \:
    3 \cdot 2 \cdot 1 \cdot \signF{\cycle{3,1,2}} \equal 0
\end{align*}
Note that the permutation matrices of the following two permutations $(\cycle{2,3,1})$ and $(\cycle{3,1,2})$
have even parity. \\

\problem{1b}

There is only one nonzero term of the permutation expansion sum. This is because
in each row there is only one nonzero element to choose.
\begin{align*}
  \rowarrowsep=-2pt
  \det{
  \begin{gmatrix}[p]
    0 & 0 & 0 & 0 & 1 \\
    0 & 0 & 0 & 2 & 0 \\
    0 & 0 & 3 & 0 & 0 \\
    0 & 4 & 0 & 0 & 0 \\
    5 & 0 & 0 & 0 & 0
  \end{gmatrix}
  }
  = 1 \cdot 2 \cdot 3 \cdot 4 \cdot 5 \cdot \signF{\cycle{5,4,3,2,1}} = 120
\end{align*}
Note that the permutation matrix of the permutation $(\cycle{5,4,3,2,1})$ has
even parity. \\

\problem{2a}

Here we perform a Laplace expansion arround the first row so $k=1$.
\begin{align*}
  \rowarrowsep=-2pt
  \det{
  \begin{gmatrix}[p]
    0 & 2 & -3 \\
    -2 & 0 & 1 \\
    3 & -1 & 0
  \end{gmatrix} 
  }
  & \equal 0 \cdot \det{
    \rowarrowsep=-2pt
    \begin{gmatrix}[p]
      0  & 1 \\
      -1 & 0
    \end{gmatrix}
    }
  - 2 \cdot \det{
    \rowarrowsep=-2pt
    \begin{gmatrix}[p]
      -2 & 1 \\
       3 & 0
    \end{gmatrix}
    }
  - 3 \cdot \det{
    \rowarrowsep=-2pt
    \begin{gmatrix}[p]
      -2 & 0 \\
       3 & -1
    \end{gmatrix}
  } = 0        
\end{align*}

\problem{2b}

Here we perform a Laplace expansion arround the first row so $k=1$.
\begin{align*}
  \det{
  \rowarrowsep=-2pt
  \begin{gmatrix}[p]
    0 & 0 & 0 & 0 & 1 \\
    0 & 0 & 0 & 2 & 0 \\
    0 & 0 & 3 & 0 & 0 \\
    0 & 4 & 0 & 0 & 0 \\
    5 & 0 & 0 & 0 & 0
  \end{gmatrix}
  }
  & \equal 0 \cdot (\dots) - 0 \cdot (\dots) + 0 \cdot (\dots) - 0 \cdot (\dots) + 1 \cdot
    \det{
    \begin{gmatrix}[p]
      0 & 0 & 0 & 2 \\
      0 & 0 & 3 & 0 \\
      0 & 4 & 0 & 0 \\
      5 & 0 & 0 & 0
    \end{gmatrix}
    } \\
  & \equal 1 \cdot \left ( 0 \cdot (\dots) - 0 \cdot (\dots) + 0 \cdot (\dots) - 2 \cdot
    \det{
    \rowarrowsep=-2pt
    \begin{gmatrix}[p]
      0 & 0 & 3 \\
      0 & 4 & 0 \\
      5 & 0 & 0
    \end{gmatrix}
    }
    \right ) \\
  & \equal 1 \cdot -2 \cdot \left ( 0 \cdot (\dots) - 0 \cdot (\dots) + 3 \cdot
    \det{
    \rowarrowsep=-2pt
    \begin{gmatrix}[p]
      0 & 4 \\
      5 & 0
    \end{gmatrix}
    }
    \right ) \\
  & \equal 1 \cdot -2 \cdot 3 \cdot (0 \cdot 0 - 4 \cdot 5) = 120
\end{align*}

\problem{3a}

Assume without loss of generality that $x_1 = x_2 = x_*$.
\begin{align*}
  \det{
  \rowarrowsep=-2pt
  \begin{gmatrix}[p]
    1 & 1 & 1 & \dots & 1 \\
    x_* & x_* & x_3 & \dots & x_n \\
    x_*^2 & x_*^2 & x_3^2 & \dots & x_n^2 \\
    \vdots & \vdots & \vdots & \ddots & \vdots \\
    x_*^{n-1} & x_*^{n-1} & x_3^{n-1} & \dots & x_n^{n-1}
  \end{gmatrix} 
  }
  & \equal \det{
    \rowarrowsep=-2pt
    \begin{gmatrix}[p]
      1 & x_* & x_*^2 & \dots & x_*^{n-1} \\
      1 & x_* & x_*^2 & \dots & x_*^{n-1} \\
      1 & x_3 & x_3^2 & \dots & x_3^{n-1} \\
      \vdots & \vdots & \vdots & \ddots & \vdots \\
      1 & x_n & x_n^2 & \dots & x_n^{n-1}
    \end{gmatrix}
    }
   = 0
\end{align*}
The proof above depends on the fact that $\det(A) = \det(A^T)$. $\qed$

\newpage

\problem{3b}
\begin{align*}
  \displaystyle
  \det(V(x_1,\dots,x_n))
     & \equal \sum_{\sigma \in S_n} \signF{\sigma} \cdot \prod_{i=1}^n V(x_1,\dots,x_n)_{i \, \sigma(i)}
     && \evidence{The permutation expansion} \\
  {} & \equal \sum_{\sigma \in S_n} \signF{\sigma} \cdot \prod_{i=1}^n x_{\sigma(i)}^{i-1}
     && \evidence{The definition of the Vandermonde matrix} \\
     & \equal \sum_{\sigma \in S_n} \signF{\sigma} \cdot \prod_{i=2}^{n} x_{\sigma(i)}^{i-1}
     && \evidence{$z^{1-1}=1$} \\
     & \equal \sum_{\sigma \in S_n} \signF{\sigma} \cdot \prod_{i=1}^{n-1} x_{\sigma(i)}^{i}
     && \evidence{Shift indices}
\end{align*}

Note that the degree of
$\displaystyle \prod_{i=1}^{n - 1} x_{\sigma(i)}^{i}$ is
$\displaystyle \sum_{i=1}^{n-1} i = \frac{n(n - 1)}{2}$ and that
$\signF{\sigma} \in F$. These facts imply that every term of the permutation
expansion has degree
$\displaystyle \frac{n(n - 1)}{2}.$ $\qed$ \\

\problem{3c}

The problem above tells us that every term of $f$ is of the form
$\displaystyle \signF{\sigma} \cdot \prod_{i=1}^n x_{\sigma(i)}^{i-1}$. We know
that the coefficent $k$ is equal to the coefficent of
$\displaystyle \signF{\sigma} \cdot \prod_{i=1}^n x_{\sigma(i)}^{i-1}$ when
$\sigma = (\cycle{1,2,\dots,n})$. That permutation clearly has even parity so
$k = 1$. $\qed$

\end{document}








