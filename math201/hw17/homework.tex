\documentclass[fleqn]{article}

\usepackage[a4paper, margin=2cm]{geometry}
\usepackage{amsmath}
\usepackage{amssymb}
\usepackage{gauss}
\usepackage[inline]{enumitem}

\setlength{\parindent}{0pt}
\setlength{\mathindent}{0pt}

% Allow for Augmented Matricies
\usepackage{etoolbox}
\makeatletter
\patchcmd\g@matrix
 {\vbox\bgroup}
 {\vbox\bgroup\normalbaselines}% restore the standard baselineskip
 {}{}
\makeatother

\newcommand{\BAR}{%
  \hspace{-\arraycolsep}%
  \strut\vrule % the `\vrule` is as high and deep as a strut
  \hspace{-\arraycolsep}%
}

\newcommand{\squig}[0]{\ensuremath{\rightsquigarrow}}

\newcommand{\problem}[1]{\large\textbf{Problem #1}\normalsize}

\newcommand{\evidence}[1]{\ensuremath{(\hspace{0.2em} \text{#1} \hspace{0.2em})}}
\newcommand{\relation}[1]{\ensuremath{\hspace{0.2em} {{} #1 {}} \hspace{0.2em}}}
\newcommand{\equal}{\relation{=}}
\newcommand{\qed}{\hfill\ensuremath{\square}}

\newcommand{\idF}[1]{\ensuremath{\text{id}(#1)}}
\newcommand{\coordsF}[2]{\ensuremath{[ \: #1 \: ]_{\mathcal{#2}}}}
\newcommand{\rankF}[1]{\ensuremath{\text{rank}(#1)}}
\newcommand{\nullityF}[1]{\ensuremath{\text{nullity}(#1)}}
\newcommand{\matrixRep}[3]{\ensuremath{{\left [ \: #1 \: \right ]}_{\mathcal{#2}}^{\mathcal{#3}}}}
\newcommand{\signF}[1]{\ensuremath{\text{sign}(#1)}}
\usepackage{listofitems}
% Cycle Notation
\newcommand\cycle[2][\:]{
  \readlist\thecycle{#2}
  #1\foreachitem\i\in\thecycle{\ifnum\icnt=1\else#1\fi\i}#1
}

\begin{document}

\noindent\Large\textbf{Problem Set Week 11 Friday} \\
\normalsize
Alice McKean \\
\today \\

\problem{1a}

Here we can see that $\langle \, , \rangle$ is linear on its first argument:
\begin{align*}
  \left \langle x + r \cdot z, y \right \rangle
  = y^* A (x + r \cdot z)
  = y^* A x + y^* A (r \cdot z)
  = y^* A x + r \cdot (y^* A z)
  = \langle x, y \rangle + r \langle z, y \rangle
\end{align*}
Note that $A = \overline{A^T}$ as $A$ is anti-symmetric and purely imaginary on
the antidiagonal. \\
This fact allows $\langle \, , \rangle$ to be conjugate symmetric:
\begin{align*}
  \langle x, y \rangle
  &\equal y^* A x
  &&\evidence{Definition of $\langle \, , \rangle$} \\
  &\equal (y^* A x)^T
  &&\evidence{When $Z \in M_{1\times 1}$ then $Z = Z^T$} \\
  &\equal x^T A^T \overline{y}
  &&\evidence{$(Z_1Z_2\dots Z_n)^T = Z_n^TZ_{n-1}^T\dots Z_1^T$} \\
  &\equal \overline{\overline{x^T A^T \overline{y}}}
  &&\evidence{Conjugation is an ivolution} \\
  &\equal \overline{\overline{x^T} \: \overline{A^T} \: \overline{\overline{y}}} 
  &&\evidence{Conjugation respects matrix multiplication} \\
  &\equal \overline{x^* A y}
  &&\evidence{Lemma given above} \\
  &\equal \overline{\langle y, x \rangle}
  &&\evidence{Definition of $\langle \, , \rangle$}
\end{align*}
Lastly $\langle \, , \rangle$ is positive definite:
\begin{align*}
  \rowarrowsep=-2pt
  \left \langle
  \begin{gmatrix}[p]
    a + bi \\
    c + di
  \end{gmatrix}
  ,
  \begin{gmatrix}[p]
    a + bi \\
    c + di
  \end{gmatrix}
  \right \rangle
  {} &\equal
  \rowarrowsep=-2pt
  {\begin{gmatrix}[p]
    a + bi \\
    c + di
  \end{gmatrix}}^*
  \begin{gmatrix}[p]
    1 & i \\
    -i & 2
  \end{gmatrix}
  \begin{gmatrix}[p]
    a + bi \\
    c + di
  \end{gmatrix}
  &&\evidence{Definition of $\langle \, , \rangle$} \\
  {} &\equal
  \rowarrowsep=-2pt
  \begin{gmatrix}[p]
    a - bi & c - di
  \end{gmatrix}
  \begin{gmatrix}[p]
    1 & i \\
    -i & 2
  \end{gmatrix}
  \begin{gmatrix}[p]
    a + bi \\
    c + di
  \end{gmatrix}
  &&\evidence{Definition of transpose and conjugate} \\
  {} &\equal a^2 - 2ad + 2d^2 + b^2 + 2bc + 2c^2
  &&\evidence{Complex multiplication} \\
  {} &\equal (a - d)^2 + d^2 + (b + c)^2 + c^2
  &&\evidence{Complete the square} \\
  {} &\relation{\geq} 0
  &&\evidence{Sum of squares}
\end{align*}
Now assume to the contrary that the above equation equals zero when $a \neq 0$
The first term requires $d = a$ but the next term makes the whole equation
non zero. All the other variables follow similar logic so the equation is zero
if and only if $a = b = c = d = 0$. $\qed$ \\

\problem{1b}
\begin{align*}
  \rowarrowsep=-2pt
  \left \langle
  \begin{gmatrix}[p]
    1 - i \\
    2 + 3i
  \end{gmatrix}
  ,
  \begin{gmatrix}[p]
    2 + i \\
    3 - 2i
  \end{gmatrix}
  \right \rangle
  =
  {\begin{gmatrix}[p]
    2 + i \\
    3 - 2i
  \end{gmatrix}}^*
  \begin{gmatrix}[p]
    1 & i \\
    -i & 2
  \end{gmatrix}
  \begin{gmatrix}[p]
    1 - i \\
    2 + 3i
  \end{gmatrix}
  = 
  \begin{gmatrix}[p]
    2 - i & 3 + 2i
  \end{gmatrix}
  \begin{gmatrix}[p]
    1 & i \\
    -i & 2
  \end{gmatrix}
  \begin{gmatrix}[p]
    1 - i \\
    2 + 3i
  \end{gmatrix}
  = - 4 + 25i
\end{align*} \\

\newpage

\problem{2}

Here we can see that $\langle \, , \rangle$ is linear on its first argument:
\begin{align*}
  \langle A + rC, B \rangle
  = \text{tr}(B^*(A + rC))
  = \text{tr}(B^*A) + r \, \text{tr}(B^*C)
  =  \langle A, B \rangle + r \langle C, B \rangle
\end{align*}
Note that $\langle \, , \rangle$ is conjugate symmetric:
\begin{align*}
  \langle A, B \rangle
  &= \text{tr}\left ( B^*A \right )
  && \evidence{Definition of $\langle \, , \rangle$} \\
  &= \text{tr}\left ( \overline{B^T}A \right )
  && \evidence{Definition of $X^*$} \\
  &= \text{tr}\left ( A^T \, \overline{B} \, \right )
  && \evidence{$\text{tr}(A) = \text{tr}(A^T)$} \\
  &= \text{tr}\left ( \overline{\overline{A^T\overline{B}}} \right )
  && \evidence{Conjugation is an involution} \\
  &= \text{tr}\left ( \overline{A^*B} \, \right )
  && \evidence{Simplification} \\
  &= \sum_{i=1}^n {\left ( \overline{A^*B} \right )}_{ii}
  && \evidence{Definition of tr} \\
  &= \sum_{i=1}^n \overline{\left ( A^*B \right )_{ii}}
  && \evidence{Definition of matrix conjugate} \\
  &= \overline{\sum_{i=1}^n \left ( A^*B \right )_{ii}}
  && \evidence{Conjugation respects addition} \\
  &= \overline{\text{tr}\left ( A^*B \right )}
  && \evidence{Defintion of tr} \\
  &= \overline{\langle B, A \rangle}
  && \evidence{Definition of $\langle \, , \rangle$}
\end{align*}
Lastly $\langle \, , \rangle$ is positive definite:
\begin{align*}
  \langle A, A \rangle
  = \text{tr}(A^*A)
  = \sum_{i=1}^n (A^*A)_{ii}
  = \sum_{i=1}^n\sum_{k=1}^m \overline{(A^T)_{ik}} A_{ki}
  = \sum_{i=1}^n\sum_{k=1}^m \overline{A_{ki}} A_{ki}
  \geq 0
\end{align*}
Note that the last sum covers every element of $A$ so the equation above is zero
when every element in $A$ is zero. Therefore the equation above is zero when
$A = 0_{n\times m} = 0$. $\qed$

\end{document}