\documentclass[fleqn]{article}

\usepackage[a4paper, margin=2cm]{geometry}
\usepackage{amsmath}
\usepackage{amssymb}
\usepackage{gauss}
\usepackage{cancel}
\usepackage[inline]{enumitem}

\setlength{\parindent}{0pt}
\setlength{\mathindent}{0pt}

% Allow for Augmented Matricies
\usepackage{etoolbox}
\makeatletter
\patchcmd\g@matrix
 {\vbox\bgroup}
 {\vbox\bgroup\normalbaselines}% restore the standard baselineskip
 {}{}
\makeatother

\newcommand{\BAR}{%
  \hspace{-\arraycolsep}%
  \strut\vrule % the `\vrule` is as high and deep as a strut
  \hspace{-\arraycolsep}%
}

\newcommand{\squig}[0]{\ensuremath{\rightsquigarrow}}

\newcommand{\problem}[1]{\large\textbf{Problem #1}\normalsize}

\newcommand{\evidence}[1]{\ensuremath{(\hspace{0.2em} \text{#1} \hspace{0.2em})}}
\newcommand{\relation}[1]{\ensuremath{\hspace{0.2em} {{} #1 {}} \hspace{0.2em}}}
\newcommand{\equal}{\relation{=}}
\newcommand{\qed}{\hfill\ensuremath{\square}}

\newcommand{\idF}[1]{\ensuremath{\text{id}(#1)}}
\newcommand{\coordsF}[2]{\ensuremath{[ \: #1 \: ]_{\mathcal{#2}}}}
\newcommand{\matrixRep}[3]{\ensuremath{[ \: #1 \: ]_{\mathcal{#2}}^{\mathcal{#3}}}}

\begin{document}

\noindent\Large\textbf{Problem Set Week 8 Tuesday} \\
\normalsize
Alice McKean \\
\today \\

\problem{1a}
\begin{align*}
  \rowarrowsep=1pt
  \det{
    \begin{gmatrix}[p]
      1 & 4 & 7 \\
      2 & 5 & 8 \\
      3 & 6 & 9
      \rowops
      \add[-2]{0}{1}
      \add[-3]{0}{2}
    \end{gmatrix}
  } 
  = \det{
      \begin{gmatrix}[p]
        1 & 4  & 7 \\
        0 & -3 & -6 \\
        0 & -6 & -12
        \rowops
        \add[-1]{1}{2}
      \end{gmatrix}
  }
  = \det{
      \begin{gmatrix}[p]
        1 & 4  & 7 \\
        0 & -3 & -6 \\
        0 & -3 & -6
        \rowops
        \add[-1]{1}{2}
      \end{gmatrix}
  }
  = 0
\end{align*} 
The last matrix shown above has a repeated row so its determinate is zero. \\

\problem{1b}
\begin{align*}
  \rowarrowsep=1pt
  \det{
    \begin{gmatrix}[p]
      1 & 3 & -1 & 2 \\
      2 & 4 & 7 & -3 \\
      0 & 0 & 2 & 0 \\
      0 & 0 & 0 & 6
      \rowops
      \add[-2]{0}{1}
    \end{gmatrix}
  }
  &= \det{
     \begin{gmatrix}[p]
       1 & 3 & -1 & 2 \\
       0 & -2 & 9 & -7 \\
       0 & 0 & 2 & 0 \\
       0 & 0 & 0 & 6
       \rowops
       \mult{1}{\cdot-1/2}
       \mult{2}{\cdot1/2}
       \mult{3}{\cdot1/6}
     \end{gmatrix}
     } \\
  {} &= -2 \cdot 2 \cdot 6 \cdot \det{
     \begin{gmatrix}[p]
       1 & 3 & -1 & 2 \\
       0 & 1 & -9/2 & 7/2 \\
       0 & 0 & 1 & 0 \\
       0 & 0 & 0 & 1
     \end{gmatrix}
     } \\
  {} &= -24 \cdot 1 = -24
\end{align*}
The last matrix shown above is upper triangular with a diagonal of ones so it's determinate will always
be one. \\

\problem{1c}
\begin{align*}
  \rowarrowsep=1pt
  \det{
    \begin{gmatrix}[p]
       4 & -1 & -1 & -1 \\
      -1 &  4 & -1 & -1 \\
      -1 & -1 &  4 & -1 \\
      -1 & -1 & -1 & 4
      \rowops
      \mult{0}{\cdot 1/4}
      \add[]{0}{1}
      \add[]{0}{2}
      \add[]{0}{3}
    \end{gmatrix}
  }
    &= 4 \cdot \det{
     \begin{gmatrix}[p]
       1 & -1/4 & -1/4 & -1/4 \\
       0 &  15/4 & -5/4 & -5/4 \\
       0 & -5/4 &  15/4 & -5/4 \\
       0 & -5/4 & -5/4 & 15/4
       \rowops
       \mult{1}{\cdot 4/15}
       \add[5/4]{1}{2}
       \add[5/4]{1}{3}
     \end{gmatrix}
     } \\
 {} &= 15 \cdot \det{
     \begin{gmatrix}[p]
       1 & -1/4 & -1/4 & -1/4 \\
       0 & 1    & -1/3 & -1/3 \\
       0 & 0    &  10/3 & -5/3 \\
       0 & 0    & -5/3 & 10/3
       \rowops
       \mult{2}{\cdot 3/10}
       \add[5/3]{2}{3}
     \end{gmatrix}
     } \\
 {} &= 50 \cdot \det{
     \begin{gmatrix}[p]
       1 & -1/4 & -1/4 & -1/4 \\
       0 & 1    & -1/3 & -1/3 \\
       0 & 0    &  1   & -1/2 \\
       0 & 0    &  0   & 5/2
       \rowops
       \mult{3}{\cdot 2/5}
     \end{gmatrix}
     } \\
 {} &= 125 \cdot \det{
     \begin{gmatrix}[p]
       1 & -1/4 & -1/4 & -1/4 \\
       0 & 1    & -1/3 & -1/3 \\
       0 & 0    &  1   & -1/2 \\
       0 & 0    &  0   & 1
     \end{gmatrix}
     } \\
    &= 125 \cdot 1 = 125
\end{align*}
The last matrix shown above is upper triangular with a diagonal of ones so it's determinate will always
be one. \\

\problem{2a}
\begin{align*}
  u \wedge v
     &= (a e_1 + b e_2) \wedge (c e_1 + d e_2)
     && \evidence{Standard Basis for $\mathbb{R}^2$} \\
  {} &= a e_1 \wedge (c e_1 + d e_2) + be_2 \wedge (c e_1 + d e_2)
     && \evidence{Multilinearity of $\wedge$} \\
  {} &= a e_1 \wedge c e_1 + ae_1 \wedge d e_2 + be_2 \wedge c e_1 + be_2 \wedge d e_2
     && \evidence{Multilinearity of $\wedge$} \\
  {} &= ac(e_1 \wedge e_1) + ad(e_1 \wedge e_2) + bc(e_2 \wedge e_1) + bd(e_2 \wedge e_2)
     && \evidence{Multilinearity of $\wedge$} \\
  {} &= ad(e_1 \wedge e_2) + bc(e_2 \wedge e_1)
     && \evidence{Alternation of $\wedge$} \\
  {} &= ad(e_1 \wedge e_2) - bc(e_1 \wedge e_2)
     && \evidence{Anticommutativity of $\wedge$} \\
  {} &= (ad - bc)(e_1 \wedge e_2)
     && \evidence{Simplify}
\end{align*}
Therefore
$k = ad - bc =
\det{
  \rowarrowsep=-2pt
  \begin{gmatrix}[p]
    a & b \\
    c & d
  \end{gmatrix} 
}
$.
$\qed$ \\

\problem{2b}
\begin{align*}
  u \wedge v
     &= (u_1 e_1 + u_2 e_2 + u_3 e_3) \wedge (v_1 e_1 + v_2 e_2 + v_3 e_3)
     && \evidence{Standard Basis for $\mathbb{R}^3$} \\
  {} &= u_1 e_1 \wedge (v_1 e_1 + v_2 e_2 + v_3 e_3) +
        (u_2 e_2 + u_3 e_3) \wedge (v_1 e_1 + v_2 e_2 + v_3 e_3)
     && \evidence{Multilinearity of $\wedge$} \\
  {} &= u_1 e_1 \wedge v_1 e_1 + u_1 e_1 \wedge (v_2 e_2 + v_3 e_3) +
        (u_2 e_2 + u_3 e_3) \wedge (v_1 e_1 + v_2 e_2 + v_3 e_3)
     && \evidence{Multilinearity of $\wedge$} \\
  {} &= u_1 e_1 \wedge (v_2 e_2 + v_3 e_3) +
        (u_2 e_2 + u_3 e_3) \wedge (v_1 e_1 + v_2 e_2 + v_3 e_3)
     && \evidence{Alternation of $\wedge$} \\
  {} &= u_1 e_1 \wedge v_2 e_2 + u_1 e_1 \wedge v_3 e_3 +
        (u_2 e_2 + u_3 e_3) \wedge (v_1 e_1 + v_2 e_2 + v_3 e_3)
     && \evidence{Multilinearity of $\wedge$} \\
  {} &= u_1 e_1 \wedge v_2 e_2 + u_1 e_1 \wedge v_3 e_3 +
        (u_2 e_2 + u_3 e_3) \wedge (v_1 e_1 + v_2 e_2 + v_3 e_3)
     && \evidence{Multilinearity of $\wedge$} \\
  {} &= u_1v_2(e_1 \wedge e_2) + u_1v_3(e_1 \wedge e_3) +
        (u_2 e_2 + u_3 e_3) \wedge (v_1 e_1 + v_2 e_2 + v_3 e_3)
     && \evidence{Multilinearity of $\wedge$} \\
  {} &= \{ \dots \} + 
       (u_2 e_2) \wedge (v_1 e_1 + v_2 e_2 + v_3 e_3) +
       (u_3 e_3) \wedge (v_1 e_1 + v_2 e_2 + v_3 e_3)
     && \evidence{Multilinearity of $\wedge$} \\
  {} &= \{ \dots \} + 
       u_2 e_2 \wedge v_1 e_1 + u_2 e_2 \wedge v_2 e_2 + u_2 e_2 \wedge v_3 e_3 +
       \{ \dots \}
     && \evidence{Multilinearity of $\wedge$} \\
  {} &= \{ \dots \} + 
       u_2 e_2 \wedge v_1 e_1 + u_2 e_2 \wedge v_3 e_3 +
       (u_3 e_3) \wedge (v_1 e_1 + v_2 e_2 + v_3 e_3)
     && \evidence{Alternation of $\wedge$} \\
  {} &= \{ \dots \} + \{ \dots \} +
       u_3 e_3 \wedge v_1 e_1 + u_3 e_3 \wedge v_2 e_2 + u_3 e_3 \wedge v_3 e_3
     && \evidence{Multilinearity of $\wedge$} \\
  {} &= \{ \dots \} +
       u_2 e_2 \wedge v_1 e_1 + u_2 e_2 \wedge v_3 e_3 +
       u_3 e_3 \wedge v_1 e_1 + u_3 e_3 \wedge v_2 e_2
     && \evidence{Alternation of $\wedge$} \\
  {} &= \{ \dots \} +
       u_2v_1(e_2 \wedge e_1) + u_2v_3(e_2 \wedge e_3) +
       u_3v_1(e_3 \wedge e_1) + u_3v_2(e_3 \wedge e_2)
     && \evidence{Multilinearity of $\wedge$} \\
  {} &= \{ \dots \} -
       u_2v_1(e_1 \wedge e_2) + u_2v_3(e_2 \wedge e_3) -
       u_3v_1(e_1 \wedge e_3) - u_3v_2(e_2 \wedge e_3)
     && \evidence{Anticommutativity of $\wedge$} \\
  {} &= \{ \dots \} -
       u_2v_1(e_1 \wedge e_2) + (u_2v_3 - u_3v_2)(e_2 \wedge e_3) -
       u_3v_1(e_1 \wedge e_3)
     && \evidence{Simplify} \\
  {} &= u_1v_2(e_1 \wedge e_2) + u_1v_3(e_1 \wedge e_3) -
       u_2v_1(e_1 \wedge e_2) + \{ \dots \} -
       u_3v_1(e_1 \wedge e_3)
     && \evidence{Simplify} \\
  {} &= (u_1v_2 - u_2v_1)(e_1 \wedge e_2) - (u_3v_1 - u_1v_3)(e_1 \wedge e_3) + \{ \dots \}
     && \evidence{Simplify} \\
  {} &= (u_2v_3 - u_3v_2)(e_2 \wedge e_3) - (u_3v_1 - u_1v_3)(e_1 \wedge e_3) +
        (u_1v_2 - u_2v_1)(e_1 \wedge e_2)
     && \evidence{Simplify}
\end{align*}
Therefore
$p = u_2v_3 - u_3v_2$,
$q = u_3v_1 - u_1v_3$, and
$r = u_1v_2 - u_2v_1$. $\qed$

\end{document}