\documentclass{article}

\usepackage[a4paper, margin=2cm]{geometry}
\usepackage{amsmath}
\usepackage{amssymb}
\usepackage{gauss}
\usepackage[inline]{enumitem}

\setlength\parindent{0pt}

\newcommand{\problem}[1]{\large\textbf{Problem #1}\normalsize}

\newcommand{\evidence}[1]{\ensuremath{(\hspace{0.2em} \text{#1} \hspace{0.2em})}}
\newcommand{\relation}[1]{\ensuremath{\hspace{0.2em} {{} #1 {}} \hspace{0.2em}}}
\newcommand{\equal}{\relation{=}}
\newcommand{\qed}{\hfill\ensuremath{\square}}


\begin{document}

\noindent\Large\textbf{Problem Set Week 6 Tuesday} \\
\normalsize
Alice McKean \\
\today \\

\problem{1a}

$
{\left [ f \right ]}_{\mathcal{B}}^{\mathcal{D}} = 
\rowarrowsep=-2pt
\begin{gmatrix}[b]
  \: [ f(1) ]_{\mathcal{D}} &
  [ f(x) ]_{\mathcal{D}} &
  [ f(x^2) ]_{\mathcal{D}} \:
\end{gmatrix} 
=
\begin{gmatrix}[b]
  \: [ x ]_{\mathcal{D}} &
  [ \frac{x^2}{2} ]_{\mathcal{D}} &
  [ \frac{x^3}{3} ]_{\mathcal{D}} \:
\end{gmatrix} 
=
\begin{gmatrix}[b]
  0 & 0           & 0 \\
  1 & 0           & 0 \\
  0 & 1/2 & 0 \\
  0 & 0           & 1/3  
\end{gmatrix} 
$

\problem{1b}

$
{\left [ f \right ]}_{\mathcal{B}}^{\mathcal{D}} = 
\rowarrowsep=-2pt
\begin{gmatrix}[b]
  \: [ f(1) ]_{\mathcal{D}} &
  [ f(1 + x) ]_{\mathcal{D}} &
  [ f(1 + x + x^2) ]_{\mathcal{D}} \:
\end{gmatrix} 
=
\begin{gmatrix}[b]
  \: [ x ]_{\mathcal{D}} &
  [ x + \frac{x^2}{2} ]_{\mathcal{D}} &
  [ x + \frac{x^2}{2} + \frac{x^3}{3} ]_{\mathcal{D}} \:
\end{gmatrix} 
=
\begin{gmatrix}[b]
  -1 & -1 & -1 \\
  1 & 1/2 & 1/2 \\
  0 & 1/2 & 1/6 \\
  0 & 0   & 1/3  
\end{gmatrix} 
$

\problem{2a}

To show $f$ is injective assume $f(p(x)) = f(q(x))$ which means
$\int_{0}^xp(t)dt = \int_{0}^xq(t)dt$. Differentiate both sides to
get $p(x) = q(x)$. Therefore $f$ is injective. $\qed$ \\

Assume to the contrary that $f$ is surjective. Therefore there exists some
$p(x) \in \mathcal{P}(\mathbb{R})$ such that $f(p(x)) = 1$ and $\int_0^xp(t)dt =
1$. Differentiate both sides to get $p(x) = 0$ but $\int_0^x0 dt = 0$ so
$0 = 1$. $\rightarrow\leftarrow$ $\qed$ \\

\problem{2b}

To show $g$ is surjective let $q(x) \in \mathcal{P}(\mathbb{R})$ and note that
when $p(x) = \int_0^xq(t)dt$ the following equation holds $f(p(x)) = q(x)$.
Therefore $g$ is surjective. $\qed$ \\

Assume to the contrary that $g$ is injective. Note that $g(1) = g(2)$ because
so by assumption $1 = 2$. $\rightarrow\leftarrow$ $\qed$ \\

\problem{2c}

$f \circ g$ is not injective because $f(g(1)) = f(g(2))$.

$f \circ g$ is not surjective because there isn't a
$p(x) \in \mathcal{P}(\mathbb{R})$ such that
$(f \circ g)(p(x)) = 1$ by a similar argument to 2b. \\

$g \circ f$ is injective and surjective because $g \circ f = id$.

\end{document}