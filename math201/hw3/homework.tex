\documentclass{article}

\usepackage[a4paper, margin=2cm]{geometry}
\usepackage{amsmath}
\usepackage{amssymb}

\setlength\parindent{0pt}

\newcommand{\problem}[1]{\large\textbf{Problem #1}\normalsize}
\newcommand{\qed}{\hfill\ensuremath{\blacksquare}}

\begin{document}

\noindent\Large\textbf{Problem Set 3} \\
\normalsize
Alice McKean \\
\today \\

\problem{1a}

Those operations do not form a vector space over $\mathbb{R}$. They do not
satisfy axiom (5) distributivity. After simplifying both sides of the following equation: 
$(0 + 0) \cdot (2, 0) = 0 \cdot (2, 0) + 0 \cdot (2, 0)$ you get
$(2, 0) = (4, 0)$. This counter-example proves the claim. \qed \\

\problem{1b}

Those operations do not form a vector space over $\mathbb{R}$. They do not
satisfy axiom (2) additive associativity. After simplifying both sides of
the following equation: $((1, 0) + (2, 0)) + (3, 0) = (1, 0) + ((2, 0) + (3, 0))$
you get $(64, 0) = (84, 0)$. This counter-example proves the claim. \qed \\

\problem{2a}

First note that $\vec{0} \in W$ since $3 \cdot 0 - 0 - 2 \cdot 0 = 0$. Hence, $W \neq \emptyset$. Next,
suppose that $r \in F$ and $u, v \in W$. Note that $u = 3a - b - 2c = 0$ and
$v = 3x - y - 2z = 0$. Multiply v by r, add the equations, and rearange terms so that:
$3(a + rx) - (b + ry) - 2(c + rz) = 0$. Hence, $u + r \cdot v \in W$.
Therefore W is a subspace of V. \qed \\

\problem{2b}

First note that $\vec{0} \in W$ since $0$ is a polynomial with degree $0$. Hence, $W \neq \emptyset$. Next,
suppose that $r \in F$ and $u, v \in W$. Note that $\forall n. \: 1^n = 1$ so any
element of $W$ simplifies to zero. Hence, $u + r \cdot v = 0 + r \cdot 0 = 0 \in W$.
Therefore W is a subspace of V. \qed


\end{document}