\documentclass[fleqn]{article}

\usepackage[a4paper, margin=2cm]{geometry}
\usepackage{amsmath}
\usepackage{amssymb}
\usepackage{gauss}
\usepackage[inline]{enumitem}
\usepackage{tikz}

\setlength{\parindent}{0pt}
\setlength{\mathindent}{0pt}

% Allow for Augmented Matricies
\usepackage{etoolbox}
\makeatletter
\patchcmd\g@matrix
 {\vbox\bgroup}
 {\vbox\bgroup\normalbaselines}% restore the standard baselineskip
 {}{}
\makeatother

\newcommand{\BAR}{%
  \hspace{-\arraycolsep}%
  \strut\vrule % the `\vrule` is as high and deep as a strut
  \hspace{-\arraycolsep}%
}

\newcommand{\squig}[0]{\ensuremath{\rightsquigarrow}}

\newcommand{\problem}[1]{\large\textbf{Problem #1}\normalsize}

\newcommand{\evidence}[1]{\ensuremath{(\hspace{0.2em} \text{#1} \hspace{0.2em})}}
\newcommand{\relation}[1]{\ensuremath{\hspace{0.2em} {{} #1 {}} \hspace{0.2em}}}
\newcommand{\equal}{\relation{=}}
\newcommand{\qed}{\hfill\ensuremath{\square}}

\newcommand{\idF}[1]{\ensuremath{\text{id}(#1)}}
\newcommand{\coordsF}[2]{\ensuremath{[ \: #1 \: ]_{\mathcal{#2}}}}
\newcommand{\rankF}[1]{\ensuremath{\text{rank}(#1)}}
\newcommand{\nullityF}[1]{\ensuremath{\text{nullity}(#1)}}
\newcommand{\matrixRep}[3]{\ensuremath{{\left [ \: #1 \: \right ]}_{\mathcal{#2}}^{\mathcal{#3}}}}
\newcommand{\signF}[1]{\ensuremath{\text{sign}(#1)}}
\usepackage{listofitems}
% Cycle Notation
\newcommand\cycleF[2][\:]{
  \readlist\thecycle{#2}
  #1\foreachitem\i\in\thecycle{\ifnum\icnt=1\else#1\fi\i}#1
}
\newcommand{\normF}[1]{\left\lVert#1\right\rVert}

\begin{document}

\noindent\Large\textbf{Problem Set Week 12 Tuesday} \\
\normalsize
Kyle McKean \\
\today \\

\problem{1}

In this problem we will be working with an arbitrary parrallelogram with sides
$\vec{x}$ and $\vec{y}$. As you can see in the following diagram one diagonal is
$\vec{x} + \vec{y}$ and the other diagonal is $\vec{x} - \vec{y}$. \\
\begin{tikzpicture}
  \draw[line width=0.5pt,-stealth,shorten >=0.01cm](0,0)--(3,3) node[midway, left]{$\vec{x} \:$};
  \draw[line width=0.5pt,-stealth,shorten >=0.01cm](0,0)--(4,0) node[midway, below]{$\vec{y}$};
  \draw[line width=0.5pt,-stealth,shorten >=0.1cm](3,3)--(7,3) node[midway, above]{$\vec{y}$};
  \draw[line width=0.5pt,-stealth,shorten >=0.1cm](4,0)--(7,3) node[midway, right]{$\vec{x}$};
  \draw[line width=0.5pt,-stealth,shorten >=0.01cm](4,0)--(3,3) node[near end, right]{$\vec{x} - \vec{y}$};
  \draw[line width=0.5pt,-stealth,shorten >=0.1cm](0,0)--(7,3) node[near start, right]{$\:\:\:\: \vec{x} + \vec{y}$};
\end{tikzpicture}

$[\Leftarrow]$: Assume that the diagonals are perpendicular so
$\langle \, \vec{x} + \vec{y}, \, \vec{x} - \vec{y} \: \rangle = 0$:
\begin{align*}
  \normF{\vec{x}}^2
  &= \langle \, \vec{x}, \vec{x} \, \rangle \\
  &= \langle \, \vec{x}, \vec{x} \, \rangle
   - \langle \, \vec{y}, \vec{y} \, \rangle
   + \langle \, \vec{y}, \vec{y} \, \rangle \\
  &= \langle \, \vec{x}, \vec{x} \, \rangle
   - \langle \, \vec{x}, \vec{y} \, \rangle
   + \langle \, \vec{x}, \vec{y} \, \rangle
   - \langle \, \vec{y}, \vec{y} \, \rangle
   + \langle \, \vec{y}, \vec{y} \, \rangle \\
  &= \langle \, \vec{x}, \vec{x} \, \rangle
   - \langle \, \vec{y}, \vec{x} \, \rangle
   + \langle \, \vec{x}, \vec{y} \, \rangle
   - \langle \, \vec{y}, \vec{y} \, \rangle
   + \langle \, \vec{y}, \vec{y} \, \rangle \\
  &= \langle \, \vec{x} - \vec{y}, \vec{x} \, \rangle
   + \langle \, \vec{x} - \vec{y}, \vec{y} \, \rangle
   + \langle \, \vec{y}, \vec{y} \, \rangle \\
  &= \langle \, \vec{x} - \vec{y}, \vec{x} + \vec{y} \, \rangle
   + \langle \, \vec{y}, \vec{y} \, \rangle \\
  &= \langle \, \vec{x} + \vec{y}, \vec{x} - \vec{y} \, \rangle
   + \langle \, \vec{y}, \vec{y} \, \rangle \\
  &= \langle \, \vec{y}, \vec{y} \, \rangle = \normF{\vec{y} }^2
\end{align*}
Now take the square root of both sides of the equation above so
$\normF{\vec{x}} = \normF{\vec{y}}$. \\

$[\Rightarrow]$: This direction is basically the reverse of above. $\qed$ \\

\problem{2}

The length of $(1,0)$ is:
\begin{align*}
  \normF{(1,0)}
  = \sqrt{\langle (1,0), (1,0) \rangle}
  = \sqrt{3 \cdot 1 \cdot 1 + 2 \cdot 1 \cdot 0 + 2 \cdot 0 \cdot 1 + 4 \cdot 0 \cdot 0}
  = \sqrt{3}
\end{align*}
The length of $(0,1)$ is:
\begin{align*}
  \normF{(0,1)}
  = \sqrt{\langle (0,1), (0,1) \rangle}
  = \sqrt{3 \cdot 0 \cdot 0 + 2 \cdot 0 \cdot 1 + 2 \cdot 1 \cdot 0 + 4 \cdot 1 \cdot 1}
  = \sqrt{4}
\end{align*}
The inner product of $(1, 0)$ and $(0, 1)$ is:
\begin{align*}
  \langle (1, 0), (0, 1) \rangle
  = 3 \cdot 1 \cdot 0 + 2 \cdot 1 \cdot 1 + 2 \cdot 0 \cdot 0 + 4 \cdot 0 \cdot 1
  = 2
\end{align*}
Now we can put all these pieces together to find the cosine of the angle between
the two vectors:
\begin{align*}
  \cos{\left ( \arccos{\left ( \dfrac{\langle (1, 0), (0,1) \rangle}{\normF{(1, 0)}\normF{(0,1)}}\right ) }\right )}
  = \dfrac{2}{\sqrt{3}\sqrt{4}}
  = \dfrac{\sqrt{3}}{3}
\end{align*}
Any point $(y_1, y_2)$ perpendicular to $(1, 0)$ would have to satisfy the
following equation:
\begin{align*}
  0 = \langle (1, 0), (y_1, y_2) \rangle
  = 3 \cdot 1 \cdot y_1 + 2 \cdot 1 \cdot y_2 + 2 \cdot 0 \cdot y_1 + 4 \cdot 0 \cdot y_2 
  = 3 y_1 + 2 y_2
\end{align*}
A point that does is $(-2, 3)$ so that point is perpendicular to $(1, 0)$. \\

\newpage

\problem{3}

Note that $B$ is a basis for $V$ so $x$ and $y$ can be written as linear
combinations of $\{ v_1, \dots, v_n \}$:
\begin{align*}
  \langle x, y \rangle
  &= \left \langle \sum_{j=1}^n a_j v_j, y \right \rangle
  && \evidence{$\mathcal{B}$ is a basis for $V$} \\
  &= \sum_{j=1}^n a_j \left \langle v_j, y \right \rangle
  && \evidence{$\langle \, , \rangle$ is linear in the first argument} \\
  &= \sum_{j=1}^n a_j \overline{\left \langle y, v_j \right \rangle}
  && \evidence{$\langle \, , \rangle$ is conjugate symmetric} \\
  &= \sum_{j=1}^n a_j \overline{\left \langle \sum_{i=1}^n b_i v_i, v_j \right \rangle}
  && \evidence{$\mathcal{B}$ is a basis for $V$} \\
  &= \sum_{j=1}^n a_j \overline{\sum_{i=1}^n b_i \left \langle  v_i, v_j \right \rangle}
  && \evidence{$\langle \, , \rangle$ is linear in the first argument} \\
  &= \sum_{j=1}^n a_j \sum_{i=1}^n \overline{b_i} \, \overline{\left \langle  v_i, v_j \right \rangle}
  && \evidence{Conjugation respects field addition and multiplication} \\
  &= \sum_{j=1}^n a_j \sum_{i=1}^n \overline{b_i} \left \langle  v_j, v_i \right \rangle
  && \evidence{$\langle \, , \rangle$ is conjugate symmetric} \\
  &= \sum_{i=1}^n \overline{b_i} \sum_{j=1}^n \left \langle  v_j, v_i \right \rangle a_j
  && \evidence{Field axioms} \\
  &= \sum_{i=1}^n \overline{b_i} \sum_{j=1}^n A_{ij} a_j
  && \evidence{Definition of $A$} \\
  &= \sum_{i=1}^n \overline{b_i} \sum_{j=1}^n A_{ij} (\coordsF{x}{B})_{j1}
  && \evidence{Definition of $\coordsF{x}{B}$} \\
  &= \sum_{i=1}^n \overline{b_i} (A \coordsF{x}{B})_{i1}
  && \evidence{Definition of matrix multiplication} \\
  &= \sum_{i=1}^n \overline{(\coordsF{y}{B})_{i1}} (A \coordsF{x}{B})_{i1}
  && \evidence{Definition of $\coordsF{y}{B}$} \\
  &= \sum_{i=1}^n \overline{((\coordsF{y}{B})^T)_{1i}} (A \coordsF{x}{B})_{i1}
  && \evidence{Definition of the transpose} \\
  &= \sum_{i=1}^n \left (\overline{(\coordsF{y}{B})^T} \right )_{1i} (A \coordsF{x}{B})_{i1}
  && \evidence{Definition of the matrix conjugate} \\
  &= \sum_{i=1}^n \left ((\coordsF{y}{B})^* \right )_{1i} (A \coordsF{x}{B})_{i1}
  && \evidence{Definition of the conjugate transpose} \\
  &= \left ((\coordsF{y}{B})^* A \coordsF{x}{B}\right )_{11}
  && \evidence{Definition of matrix multiplication} \\
  &= (\coordsF{y}{B})^* A \coordsF{x}{B}
  && \evidence{A matrix of size $1\times 1$ is just a value}
\end{align*}
Note that two matrices $X, Y \in M_{n\times n}$ are equal if forall $i,j$ the
following equality holds $X_{ij} = Y_{ij}$.
\begin{align*}
  A_{ij}
  = \langle v_j, v_i \rangle
  = \overline{\langle v_i, v_j \rangle}
  = \overline{A_{ji}}
  = \overline{(A^T)_{ij}}
  = \left (\overline{A^T}\right )_{ij}
  = (A^*)_{ij}
\end{align*}
Therefore $A$ equals $A^*$ as the two matrices are equal for every element. $\qed$ \\

If the basis $B$ is orthonormal then $A = I_n$. Note the proof follows from cases.
Let $i,j$ be arbitrary indices of $A$:
\begin{enumerate}[label=\roman*:]
\item $i=j$ so
  $A_{ii}
  = \langle v_i, v_i \rangle
  = (\sqrt{\langle v_i, v_i \rangle})^2
  = (\normF{v_i})^2
  = 1^2 = 1 = I_{ii}$
\item $i \neq j$ so
  $A_{ij} = \langle v_j, v_i \rangle = 0 = I_{ij}$
\end{enumerate}
Therefore $A$ equals $I_n$ as the two matrices are equal for every element. $\qed$ \\
\end{document}