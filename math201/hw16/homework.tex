\documentclass[fleqn]{article}

\usepackage[a4paper, margin=2cm]{geometry}
\usepackage{amsmath}
\usepackage{amssymb}
\usepackage{gauss}
\usepackage[inline]{enumitem}

\setlength{\parindent}{0pt}
\setlength{\mathindent}{0pt}

% Allow for Augmented Matricies
\usepackage{etoolbox}
\makeatletter
\patchcmd\g@matrix
 {\vbox\bgroup}
 {\vbox\bgroup\normalbaselines}% restore the standard baselineskip
 {}{}
\makeatother

\newcommand{\BAR}{%
  \hspace{-\arraycolsep}%
  \strut\vrule % the `\vrule` is as high and deep as a strut
  \hspace{-\arraycolsep}%
}

\newcommand{\squig}[0]{\ensuremath{\rightsquigarrow}}

\newcommand{\problem}[1]{\large\textbf{Problem #1}\normalsize}

\newcommand{\evidence}[1]{\ensuremath{(\hspace{0.2em} \text{#1} \hspace{0.2em})}}
\newcommand{\relation}[1]{\ensuremath{\hspace{0.2em} {{} #1 {}} \hspace{0.2em}}}
\newcommand{\equal}{\relation{=}}
\newcommand{\qed}{\hfill\ensuremath{\square}}

\newcommand{\idF}[1]{\ensuremath{\text{id}(#1)}}
\newcommand{\coordsF}[2]{\ensuremath{[ \: #1 \: ]_{\mathcal{#2}}}}
\newcommand{\rankF}[1]{\ensuremath{\text{rank}(#1)}}
\newcommand{\nullityF}[1]{\ensuremath{\text{nullity}(#1)}}
\newcommand{\matrixRep}[3]{\ensuremath{{\left [ \: #1 \: \right ]}_{\mathcal{#2}}^{\mathcal{#3}}}}
\newcommand{\signF}[1]{\ensuremath{\text{sign}(#1)}}
\usepackage{listofitems}
% Cycle Notation
\newcommand\cycle[2][\:]{
  \readlist\thecycle{#2}
  #1\foreachitem\i\in\thecycle{\ifnum\icnt=1\else#1\fi\i}#1
}
\begin{document}

\noindent\Large\textbf{Problem Set Week 11 Tuesday} \\
\normalsize
Alice McKean \\
\today \\

\problem{1}

The adjacency matrix for $G$ is:
\begin{align*}
  A = A(G) =
  \begin{gmatrix}[p]
    1 & 1 \\
    1 & 0
  \end{gmatrix}
\end{align*}
Note that $\displaystyle {(A^4)}_{12} = 3$ so there are $3$ distinct walks from $v_1$ to $v_2$
of length $4$. \\
The possible walks are $v_1v_2v_1v_1v_2$, $v_1v_1v_2v_1v_2$, and $v_1v_1v_1v_1v_2$.
\begin{align*}
  A^4 =
  {\rowarrowsep=-2pt
   \begin{gmatrix}[p]
    1 & 1 \\
    1 & 0
   \end{gmatrix}
  }^4
  =
  \begin{gmatrix}[p]
    5 & 3 \\
    3 & 2
  \end{gmatrix}
\end{align*}
The total number of closed walks of length $4$ is the $\text{tr}(A^4)$ as shown
below:
\begin{align*}
  \text{tr}(A^4) = \text{tr}
  \rowarrowsep=-2pt
  \begin{gmatrix}[p]
    5 & 3 \\
    3 & 2
  \end{gmatrix}
  = 5 + 2 = 7
\end{align*}
The factors of the characteristic polynomial given below can be found via the
quadratic forumla to be $\phi$ and $\bar{\phi}$.
\begin{align*}
  \det(A - \lambda I) =
  \det
  \rowarrowsep=-2pt
  \begin{gmatrix}[p]
    1 - \lambda & 1 \\
    1 & - \lambda
  \end{gmatrix}
  = \lambda^2 - \lambda - 1
\end{align*}
A basis for $V_{\phi}$ is
$
\rowarrowsep=-2pt
\begin{gmatrix}[p]
  \phi \\
  1
\end{gmatrix}
$
as shown by the following row reduction:
\begin{align*}
  \rowarrowsep=-2pt
  \begin{gmatrix}[p]
    1 - \phi & 1 \\
    1 & - \phi
  \end{gmatrix}
  =
  \begin{gmatrix}[p]
    \bar{\phi} & 1 \\
    1 & - \phi
  \end{gmatrix}
  \squig
  \begin{gmatrix}[p]
    1 & 1/\bar{\phi} \\
    1 & - \phi
  \end{gmatrix}
  =
  \begin{gmatrix}[p]
    1 & - \phi \\
    1 & - \phi
  \end{gmatrix}
  \squig
  \begin{gmatrix}[p]
    1 & - \phi \\
    0 & 0
  \end{gmatrix}
\end{align*}
A basis for $V_{\bar{\phi}}$ is
$
\rowarrowsep=-2pt
\begin{gmatrix}[p]
  \bar{\phi} \\
  1
\end{gmatrix}
$
as shown by a similar row reduction:
\begin{align*}
  \rowarrowsep=-2pt
  \begin{gmatrix}[p]
    1 - \bar{\phi} & 1 \\
    1 & - \bar{\phi}
  \end{gmatrix}
  =
  \begin{gmatrix}[p]
    \phi & 1 \\
    1 & - \bar{\phi}
  \end{gmatrix}
  \squig
  \begin{gmatrix}[p]
    1 & 1/\phi \\
    1 & - \bar{\phi}
  \end{gmatrix}
  =
  \begin{gmatrix}[p]
    1 & - \bar{\phi} \\
    1 & - \bar{\phi}
  \end{gmatrix}
  \squig
  \begin{gmatrix}[p]
    1 & - \bar{\phi} \\
    0 & 0
  \end{gmatrix}
\end{align*}
\begin{align*}
  [ \: P \: | \: I_2 \: ] &=
  \begin{gmatrix}[p]
    \phi & \bar{\phi} & \BAR & 1 & 0 \\
    1    & 1          & \BAR & 0 & 1
    \rowops
    \mult{0}{\div{{\phi}}}
  \end{gmatrix}
  \\ & \squig
  \begin{gmatrix}[p]
    1 & \bar{\phi}/\phi & \BAR & 1/\phi & 0 \\
    1    & 1          & \BAR & 0 & 1
    \rowops
    \add[-1]{0}{1}
  \end{gmatrix}
  \\ & \squig
  \begin{gmatrix}[p]
    1 & \bar{\phi}/\phi     & \BAR & 1/\phi & 0 \\
    0 & 1 - \bar{\phi}/\phi & \BAR & - 1/\phi & 1
    \rowops
  \end{gmatrix}
  \\ &=
  \begin{gmatrix}[p]
    1 & \bar{\phi}/\phi     & \BAR & -\bar{\phi} & 0 \\
    0 & 1 - \bar{\phi}/\phi & \BAR & \bar{\phi} & 1
    \rowops
    \mult{1}{\cdot\phi}
  \end{gmatrix}
  \\ & \squig
  \begin{gmatrix}[p]
    1 & \bar{\phi}/\phi   & \BAR & -\bar{\phi} & 0 \\
    0 & \phi - \bar{\phi} & \BAR & \bar{\phi}\phi & \phi
    \rowops
  \end{gmatrix}
  \\ &=
  \begin{gmatrix}[p]
    1 & \bar{\phi}/\phi   & \BAR & -\bar{\phi} & 0 \\
    0 & \phi - \bar{\phi} & \BAR & -1 & \phi
    \rowops
    \mult{1}{\div{\phi - \bar{\phi}}}
  \end{gmatrix}
  \\ & \squig
  \begin{gmatrix}[p]
    1 & \bar{\phi}/\phi & \BAR & -\bar{\phi} & 0 \\
    0 & 1               & \BAR & \dfrac{-1}{\phi - \bar{\phi}} & \dfrac{\phi}{\phi - \bar{\phi}}
    \rowops
    \add[-\bar{\phi}/\phi]{1}{0}
  \end{gmatrix}
  \\ & \squig
  \begin{gmatrix}[p]
    1 & 0 & \BAR & -\bar{\phi} + \dfrac{\bar{\phi}}{\phi(\phi - \bar{\phi})}
                 & \dfrac{-\bar{\phi}\phi}{\phi(\phi - \bar{\phi})}  \\
    0 & 1 & \BAR & \dfrac{-1}{\phi - \bar{\phi}} & \dfrac{\phi}{\phi - \bar{\phi}}
  \end{gmatrix}
  \\ &=
  \rowarrowsep=-2pt
   \begin{gmatrix}[p]
    1 & 0 & \BAR & \dfrac{\phi}{\phi(\phi - \bar{\phi})}
                 & \dfrac{1}{\phi(\phi - \bar{\phi})}  \\
    0 & 1 & \BAR & \dfrac{-\phi}{\phi(\phi - \bar{\phi})} & \dfrac{\phi + 1}{\phi(\phi - \bar{\phi})}
  \end{gmatrix}
  = [ \: I_2 \: | \: P^{-1} \: ]
\end{align*}
Now we can put all these pieces together to diagonalize $A$ with respect to the
ordered basis
$\rowarrowsep=-2pt
\left \langle
\begin{gmatrix}[p]
  \phi \\
  1
\end{gmatrix}
,
\begin{gmatrix}[p]
  \bar{\phi} \\
  1
\end{gmatrix}
\right \rangle$.
\begin{align*}
  \rowarrowsep=-2pt
  \begin{gmatrix}[p]
    \phi & 0 \\
    0 & \bar{\phi}
  \end{gmatrix}
  = \dfrac{1}{\phi + 2}
  \begin{gmatrix}[p]
    \phi & 1 \\
    -\phi & \phi + 1
  \end{gmatrix}
  \begin{gmatrix}[p]
    1 & 1 \\
    1 & 0
  \end{gmatrix}
  \begin{gmatrix}[p]
    \phi & \bar{\phi} \\
    1 & 1
  \end{gmatrix}
\end{align*}
A previous theorem gave us $A^n = PD^nP^{-1}$ if $A$ is diagonalizable so the
following equality holds:
\begin{align*}
  \rowarrowsep=-2pt
  A^n = \dfrac{1}{\phi + 2}
  \begin{gmatrix}[p]
    \phi & \bar{\phi} \\
    1 & 1
  \end{gmatrix}
  {\begin{gmatrix}[p]
    \phi & 0 \\
    0 & \bar{\phi}
  \end{gmatrix}}^n
  \begin{gmatrix}[p]
    \phi & 1 \\
    -\phi & \phi + 1
  \end{gmatrix}
  = \dfrac{1}{\phi + 2}
  \begin{gmatrix}[p]
    \phi & \bar{\phi} \\
    1 & 1
  \end{gmatrix}
  \begin{gmatrix}[p]
    \phi^n & 0 \\
    0 & \bar{\phi}^n
  \end{gmatrix}
  \begin{gmatrix}[p]
    \phi & 1 \\
    -\phi & \phi + 1
  \end{gmatrix}
\end{align*}
This gives us a closed form for the number of closed walks of length $n$:
\begin{align*}
  \text{tr}(A^n) &\equal \text{tr} \left (
  \rowarrowsep=-2pt
  \begin{gmatrix}[p]
    \phi & \bar{\phi} \\
    1 & 1
  \end{gmatrix}
  \begin{gmatrix}[p]
    \phi^n & 0 \\
    0 & \bar{\phi}^n
  \end{gmatrix}
  \begin{gmatrix}[p]
    \phi & 1 \\
    -\phi & \phi + 1
  \end{gmatrix}
  \right ) \\
  & \equal \text{tr} \left (
  \rowarrowsep=-2pt
  \begin{gmatrix}[p]
    \phi^n & 0 \\
    0 & \bar{\phi}^n
  \end{gmatrix}
  \begin{gmatrix}[p]
    \phi & \bar{\phi} \\
    1 & 1
  \end{gmatrix}
  \begin{gmatrix}[p]
    \phi & 1 \\
    -\phi & \phi + 1
  \end{gmatrix}
  \right ) \\
  & \equal \text{tr} \left (
  \rowarrowsep=-2pt
  \begin{gmatrix}[p]
    \phi^n & 0 \\
    0 & \bar{\phi}^n
  \end{gmatrix}
  \right ) = \phi^n + {\bar{\phi}}^n
\end{align*}
\problem{2}

First we need to redefine $p(i, j, \ell)$ as it's an inductive definition with no
base case.
\begin{align*}
  p(i, j, 1)    &= A_{ij} \\
  p(i, j, \ell) &= \sum_{k=1}^n p(i, k, \ell) p(k, j, 1) \text{ when } \ell > 1
\end{align*}
This definition constructs all the walks of length $\ell$ in $G$ from $v_i$ to
$v_j$ by fixing some $v_k$ as the ``join'' vertex between two paths of length
$\ell - 1$ and $1$.
\begin{align*}
  w(i, j, 1)    &=
    \begin{cases}
      v_iv_j \:\: \text{if} \: v_i \: \text{connects to} \: v_j \: \text{in} \: G \\
      \varepsilon
    \end{cases} \\
  w(i, j, \ell) &=
    \bigcup_{k=1}^n \left \{ \: xv_ky \: | \: xv_k \in w(i, k, \ell - 1), \: v_ky \in w(k, j, 1) \: \right \}
\end{align*}
Note that $p(i, j, \ell) = | \: w(i, j, \ell) \: |$ as the sets being unioned
are all disjoint. \\

We proceed by induction in order to prove $(A^{\ell})_{ij} = p(i,j,\ell)$. First
note that the base case $(\ell = 1)$ is true by definition. Now assume forall
$r < \ell$ and forall $i,j \in \{1 .. n\}$ that $(A^r)_{ij} = p(i,j,r)$.
\begin{align*}
  \displaystyle
  (A^{\ell})_{ij}
  {} &\equal (A^{\ell - 1}A^1)_{ij} && \evidence{Definition of exponentiation} \\
  {} &\equal \sum_{k=1}^n(A^{\ell - 1})_{ik}(A^1)_{kj} && \evidence{Definition of matrix multiplication} \\
  {} &\equal \sum_{k=1}^n(A^{\ell - 1})_{ik} \, p(k, j, 1) && \evidence{Base definition} \\
  {} &\equal \sum_{k=1}^np(i, k, \ell - 1) \, p(k, j, 1) && \evidence{Inductive hypothesis} \\
  {} &\equal p(i, j, \ell) && \evidence{Inductive definition}
\end{align*}
Thus by the principle of induction $(A^{\ell})_{ij} = p(i,j,\ell)$ forall
$\ell$. $\qed$ \\

\problem{3}

First note that when $a = b = 1$, $p_0 = 0$, and $p_1 = 1$ the sequence $p_n$ is the
fibonnaci sequence ($0, 1, 1, 2, 3, 5, 8, \dots$).
Now we can use the results
from part 1 to find a closed form equation for the fibonnaci sequence:
\begin{align*}
  \rowarrowsep=-2pt
  \begin{gmatrix}[p]
    p_{n+1} \\
    p_{n}
  \end{gmatrix}
  =
  {\begin{gmatrix}[p]
      1 & 1 \\
      1 & 0
  \end{gmatrix}}^n
  \begin{gmatrix}[p]
    1 \\
    0
  \end{gmatrix}
  = \dfrac{1}{\phi + 2}
  \begin{gmatrix}[p]
    \phi & \bar{\phi} \\
    1 & 1
  \end{gmatrix}
  \begin{gmatrix}[p]
    \phi^n & 0 \\
    0 & \bar{\phi}^n
  \end{gmatrix}
  \begin{gmatrix}[p]
    \phi & 1 \\
    -\phi & \phi + 1
  \end{gmatrix}
  \begin{gmatrix}[p]
    1 \\
    0
  \end{gmatrix}
  = \dfrac{1}{\phi + 2}
  \begin{gmatrix}[p]
    \phi^{n+2} + \bar{\phi}^n \\
    \phi^{n + 1} + \bar{\phi}^{n - 1}
  \end{gmatrix}
\end{align*}
Therefore $p_n = \dfrac{\phi^{n + 1} + \bar{\phi}^{n - 1}}{\phi + 2}$.
\end{document}