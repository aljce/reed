\documentclass[fleqn]{article}

\usepackage[a4paper, margin=2cm]{geometry}
\usepackage{amsmath}
\usepackage{amssymb}
\usepackage{gauss}
\usepackage[inline]{enumitem}

\setlength{\parindent}{0pt}
\setlength{\mathindent}{0pt}

% Allow for Augmented Matricies
\usepackage{etoolbox}
\makeatletter
\patchcmd\g@matrix
 {\vbox\bgroup}
 {\vbox\bgroup\normalbaselines}% restore the standard baselineskip
 {}{}
\makeatother

\newcommand{\BAR}{%
  \hspace{-\arraycolsep}%
  \strut\vrule % the `\vrule` is as high and deep as a strut
  \hspace{-\arraycolsep}%
}

\newcommand{\squig}[0]{\ensuremath{\rightsquigarrow}}

\newcommand{\problem}[1]{\large\textbf{Problem #1}\normalsize}

\newcommand{\evidence}[1]{\ensuremath{(\hspace{0.2em} \text{#1} \hspace{0.2em})}}
\newcommand{\relation}[1]{\ensuremath{\hspace{0.2em} {{} #1 {}} \hspace{0.2em}}}
\newcommand{\equal}{\relation{=}}
\newcommand{\qed}{\hfill\ensuremath{\square}}

\newcommand{\idF}[1]{\ensuremath{\text{id}(#1)}}
\newcommand{\coordsF}[2]{\ensuremath{[ \: #1 \: ]_{\mathcal{#2}}}}
\newcommand{\matrixRep}[3]{\ensuremath{[ \: #1 \: ]_{\mathcal{#2}}^{\mathcal{#3}}}}

\begin{document}

\noindent\Large\textbf{Problem Set Week 7 Friday} \\
\normalsize
Alice McKean \\
\today \\

\problem{1a}
\begin{align*}
  [ \: A \: | \: I_3 \: ] =&
  \begin{gmatrix}[p]
    2 & 0 & 4 & \BAR & 1 & 0 & 0 \\
    3 & 3 & 1 & \BAR & 0 & 1 & 0 \\
    4 & 2 & 2 & \BAR & 0 & 0 & 1
    \rowops
    \mult{0}{\cdot\frac{1}{2}}
    \add[-3]{0}{1}
    \add[-4]{0}{2}
  \end{gmatrix}
  \\ \squig &
  \begin{gmatrix}[p]
    1 & 0 & 2  & \BAR & 1/2 & 0 & 0 \\
    0 & 3 & -5 & \BAR & -3/2 & 1 & 0 \\
    0 & 2 & -6 & \BAR & -2 & 0 & 1
    \rowops
    \add[-1]{2}{1}
    \add[-2]{1}{2}
  \end{gmatrix}
  \\ \squig &
  \begin{gmatrix}[p]
    1 & 0 & 2  & \BAR & 1/2  & 0 & 0 \\
    0 & 1 & 1 & \BAR  & 1/2 & 1 & -1 \\
    0 & 0 & -8 & \BAR & -3   & -2 & 3
    \rowops
    \mult{2}{\cdot-\frac{1}{8}}
    \add[-1]{2}{1}
    \add[-1]{2}{0}
  \end{gmatrix}
  \\ \squig &
  \begin{gmatrix}[p]
    1 & 0 & 2 & \BAR & 1/2  & 0 & 0 \\
    0 & 1 & 1 & \BAR  & 1/2 & 1 & -1 \\
    0 & 0 & 1 & \BAR  & 3/8  & 1/4 & -3/8
    \rowops
    \add[-1]{2}{1}
    \add[-2]{2}{0}
  \end{gmatrix}
  \\ \squig &
  \rowarrowsep=-2pt
  \begin{gmatrix}[p]
    1 & 0 & 0 & \BAR & -1/4 & -1/2 & 3/4 \\
    0 & 1 & 0 & \BAR & 1/8  & 3/4  & -5/8 \\
    0 & 0 & 1 & \BAR & 3/8  & 1/4  & -3/8
  \end{gmatrix}
  = [ \: I_3 \: | \: A^{-1} \: ]
\end{align*}
\problem{1b}

The left hand side of the following Gaussian elimination has a row of zeros. This means
the left hand side will never reduce to $I_3$ and that $B$ has no inverse.
\begin{align*}
  [ \: B \: | \: I_3 \: ] = &
  \begin{gmatrix}[p]
    4 & 1 & 1  & \BAR & 1 & 0 & 0 \\
    2 & 2 & -1 & \BAR & 0 & 1 & 0 \\
    2 & 5 & -4 & \BAR & 0 & 0 & 1
    \rowops
    \mult{0}{\cdot\frac{1}{4}}
    \add[-2]{0}{1}
    \add[-2]{0}{2}
  \end{gmatrix} 
  \\ \squig &
  \begin{gmatrix}[p]
    1 & 1/4 & 1/4  & \BAR & 1/4 & 0 & 0 \\
    0 & 3/2 & -3/2 & \BAR & -1/2 & 1 & 0 \\
    0 & 9/2 & -9/2 & \BAR & -1/2 & 0 & 1
    \rowops
    \add[-3]{1}{2}
  \end{gmatrix} 
  \\ \squig &
  \begin{gmatrix}[p]
    1 & 1/4 & 1/4  & \BAR & 1/4 & 0 & 0 \\
    0 & 3/2 & -3/2 & \BAR & -1/2 & 1 & 0 \\
    0 & 0   & 0    & \BAR & 1    & 0 & 1
  \end{gmatrix} 
\end{align*} 
\problem{2a}

Let $\langle v_1, v_2, \dots, v_n \rangle$ be the ordered basis $\mathcal{B}$ in the following equality:
\begin{align*}
  \rowarrowsep=-2pt
  \displaystyle
  \matrixRep{\text{id}}{B}{B} =
  \begin{gmatrix}[p]
    \BAR      & \BAR      & \dots & \BAR \\
    \coordsF{\idF{v_1}}{B} &
    \coordsF{\idF{v_2}}{B} & \dots &
    \coordsF{\idF{v_n}}{B} \\
    \BAR      & \BAR      & \dots & \BAR
  \end{gmatrix} 
  =
  \begin{gmatrix}[p]
    \BAR      & \BAR      & \dots & \BAR \\
    \coordsF{v_1}{B} &
    \coordsF{v_2}{B} & \dots &
    \coordsF{v_n}{B} \\
    \BAR      & \BAR      & \dots & \BAR
  \end{gmatrix} 
  =
  \begin{gmatrix}[p]
    1 & 0 & \dots & 0 \\
    0 & 1 & \dots & 0 \\
    \vdots & \vdots & \ddots & \vdots \\
    0 & 0 & \dots & 1
  \end{gmatrix} 
  = I_n
\end{align*}
\problem{2b}

Assume $f : V \to W$ is an isomorphism with two $n$ dimensional vector spaces
$V$ and $W$. Let $\mathcal{B}$ and $\mathcal{D}$ be ordered bases for $V$ and
$W$. Let $A = \matrixRep{f}{B}{D}$ then $A^{-1} = \matrixRep{f^{-1}}{D}{B}$.
This is because $\matrixRep{f^{-1}}{D}{B}$ is a left inverse and a right inverse
for $A$.
It is a left inverse for $A$ because by assumption $f \circ f^{-1} = \text{id}$.
Find the matrix representation of both sides so $\matrixRep{ f \circ f^{-1} }{D}{D} =
\matrixRep{\text{id}}{D}{D}$. Composition becomes matrix multiplication so
$\matrixRep{ f }{B}{D}\matrixRep{ f^{-1}}{D}{B}
= \matrixRep{\text{id}}{D}{D}$ and $A\matrixRep{ f^{-1}}{D}{B} = I_n$. A similar
argument shows that $\matrixRep{f^{-1}}{D}{B}$ is a right inverse of $A$. $\qed$ \\

\problem{2c}

First we find the matrix representation of $f$ with respect to the standard
bases:
\begin{align*}
  \rowarrowsep=-2pt
  \displaystyle
  \matrixRep{f}{\text{st}}{\text{st}} = 
  \begin{gmatrix}[p]
    \BAR & \BAR \\
    \coordsF{f(1 + 0z)}{\text{st}} & \coordsF{f(0 + 1z)}{\text{st}} \\
    \BAR & \BAR
  \end{gmatrix} 
  =
  \begin{gmatrix}[p]
    2 & 1 \\
    4 & 3
  \end{gmatrix} 
  = A
\end{align*}
Then we invert the matrix ($A$) to find the matrix representating $f^{-1}$:
\begin{align*}
  [ \: A \: | \: I_2 \: ] =&
  \begin{gmatrix}[p]
    2 & 1 & \BAR & 1 & 0 \\
    4 & 3 & \BAR & 0 & 1
    \rowops
    \mult{0}{\cdot\frac{1}{2}}
    \add[-4]{0}{1}
  \end{gmatrix}
  \\ \squig &
  \begin{gmatrix}[p]
    1 & 1/2 & \BAR & 1/2 & 0 \\
    0 & 1 & \BAR & -2 & 1
    \rowops
    \add[-\frac{1}{2}]{1}{0}
  \end{gmatrix}
  \\ \squig &
  \rowarrowsep=-2pt
  \begin{gmatrix}[p]
    1 & 0 & \BAR & 3/2 & -1/2 \\
    0 & 1 & \BAR & -2  & 1
    \rowops
  \end{gmatrix}
  = [ \: I_2 \: | \: A^{-1} \: ]
\end{align*}
Therefore $f(x, y) = (\frac{3}{2}x - \frac{1}{2}y) + (-2x + y)z$. \\

\problem{3a}

$(1, m)$ is on the line $L$ so the mirror image is itself and
$f(1, m) = (1, m)$. Geometrically the point $(m, -1)$ has it's sign flipped when
reflected over $L$ so $f(m, -1) = (-m, 1)$. \\

\problem{3b}

The set $\mathcal{M} = \{ \: (1, m), \: (m, -1) \: \}$ is linearly independent because
$x\cdot(1, m) + y \cdot (m, -1) = \vec{0}$ when $x = y = 0$ as shown by the following
system of equations. Note that the factor $1 + m^2 = 0$ has no real solutions.
\begin{equation*}
  \begin{cases}
    0 = x + my \\
    0 = mx - y
  \end{cases}
  \squig
  \begin{cases}
    0 = (1 + m^2) x \\
    0 = mx - y
  \end{cases}
  \squig
  \begin{cases}
    0 = x \\
    0 = y
  \end{cases}
\end{equation*}

The set $\mathcal{M}$ spans $\mathbb{R}^2$ as shown below:
\begin{equation*}
(x, y) =
\frac{x + my}{1 + m^2} \cdot (1, m) +
\frac{mx - y}{1 + m^2} \cdot (m,-1)
\end{equation*}
Therefore $\mathcal{M}$ is a basis for $\mathbb{R}^2$. $\qed$ \\

\problem{3c}
\begin{align*}
  \rowarrowsep=-2pt
  \displaystyle
  \matrixRep{\text{id}}{M}{\text{st}} =
  \begin{gmatrix}[p]
    \BAR & \BAR \\
    \coordsF{\idF{(1, m)}}{\text{st}} & \coordsF{\idF{(m, -1)}}{\text{st}} \\
    \BAR & \BAR
  \end{gmatrix} 
  =
  \begin{gmatrix}[p]
    \BAR & \BAR \\
    \coordsF{(1, m)}{\text{st}} & \coordsF{(m, -1)}{\text{st}} \\
    \BAR & \BAR
  \end{gmatrix} 
  =
  \begin{gmatrix}[p]
    1 & m \\
    m & -1
  \end{gmatrix} 
\end{align*} 
\begin{align*}
  \rowarrowsep=-2pt
  \displaystyle
  \matrixRep{f}{M}{M} =
  \begin{gmatrix}[p]
    \BAR & \BAR \\
    \coordsF{f(1, m)}{M} & \coordsF{f(m, -1)}{M} \\
    \BAR & \BAR
  \end{gmatrix} 
  =
  \begin{gmatrix}[p]
    \BAR & \BAR \\
    \coordsF{(1, m)}{M} & \coordsF{(-m, 1)}{M} \\
    \BAR & \BAR
  \end{gmatrix} 
  =
  \begin{gmatrix}[p]
    1 & 0 \\
    0 & -1
  \end{gmatrix} 
\end{align*}
\begin{align*}
  [ \: \matrixRep{\text{id}}{M}{\text{st}} \: | \: I_2 \: ] =&
  \begin{gmatrix}[p]
    1 & m  & \BAR & 1 & 0 \\
    m & -1 & \BAR & 0 & 1
    \rowops
    \add[-m]{0}{1}
  \end{gmatrix}
  \\ \squig &
  \begin{gmatrix}[p]
    1 & m  & \BAR & 1 & 0 \\
    0 & -1 - m^2 & \BAR & -m & 1
    \rowops
    \mult{1}{\cdot\frac{-1}{1 + m^2}}
  \end{gmatrix}
  \\ \squig &
  \begin{gmatrix}[p]
    1 & m & \BAR & 1 & 0 \\
    0 & 1 & \BAR & \frac{m}{1 + m^2} & \frac{-1}{1 + m^2}
    \rowops
    \add[-m]{1}{0}
  \end{gmatrix}
  \\ \squig &
  \begin{gmatrix}[p]
    1 & 0 & \BAR & 1 - \frac{m^2}{1 + m^2} & \frac{m}{1 + m^2} \\
    0 & 1 & \BAR & \frac{m}{1 + m^2} & \frac{-1}{1 + m^2}
  \end{gmatrix}
  \\ = &
  \rowarrowsep=-2pt
  \begin{gmatrix}[p]
    1 & 0 & \BAR & \frac{1}{1 + m^2} & \frac{m}{1 + m^2} \\
    0 & 1 & \BAR & \frac{m}{1 + m^2} & \frac{-1}{1 + m^2}
  \end{gmatrix}
  = [ \: I_2 \: | \: \matrixRep{\text{id}}{\text{st}}{M} \: ]
\end{align*}
\begin{align*}
  \rowarrowsep=-2pt
  \displaystyle
  \matrixRep{f}{\text{st}}{\text{st}} =
  \matrixRep{\text{id}}{M}{\text{st}} \: \matrixRep{f}{M}{M} \: \matrixRep{\text{id}}{\text{st}}{M}
  =
  \frac{1}{1 + m^2}      
  \begin{gmatrix}[p]
    1 & m \\
    m & -1
  \end{gmatrix} 
  \begin{gmatrix}[p]
    1 & 0 \\
    0 & -1
  \end{gmatrix}
  \begin{gmatrix}[p]
    1 & m \\
    m & -1
  \end{gmatrix}
  =
  \frac{1}{1 + m^2}
  \begin{gmatrix}[p]
    1 - m^2 & 2m \\
    2m & m^2 - 1
  \end{gmatrix}
\end{align*}
\begin{align*}
  f(x, y) = \left (\frac{(1 - m^2)x + 2my}{1 + m^2}, \frac{2mx + (m^2 - 1)y}{1 + m^2} \right )
\end{align*} 

\problem{3d}

When $m = 0$ the line $L$ is $y = 0$ and the equation shown above simplifies to $f(x, y) = (x, -y)$. This
function flips points over the x-axis or $y = 0$ so it makes sense. When $m = 1$ the line $L$ is
$y = 1$ and the equation shown above simlifies to $f(x, y) = (y, x)$. This
function reflects points over $y = x$ so it makes sense.
\end{document}