\documentclass{article}

\usepackage[a4paper, margin=2cm]{geometry}
\usepackage{amsmath}
\usepackage{amssymb}
\usepackage{gauss}

\setlength\parindent{0pt}

% Allow for Augmented Matricies
\usepackage{etoolbox}
\makeatletter
\patchcmd\g@matrix
 {\vbox\bgroup}
 {\vbox\bgroup\normalbaselines}% restore the standard baselineskip
 {}{}
\makeatother

\newcommand{\BAR}{%
  \hspace{-\arraycolsep}%
  \strut\vrule % the `\vrule` is as high and deep as a strut
  \hspace{-\arraycolsep}%
}

\newcommand{\rowAdd}[3]{#3 r_#1 + r_#2 \rightarrow r_#2}
\newcommand{\rowScale}[2]{#2 r_#1 \rightarrow r_#1}
\newcommand{\rowSwap}[2]{r_#1 \leftrightarrow r_#2}
\newcommand{\rowEquiv}[0]{\ensuremath{\rightsquigarrow}}
\newcommand{\problem}[1]{\large\textbf{Problem #1}\normalsize}
\newcommand{\qed}{\hfill\ensuremath{\blacksquare}}

\begin{document}

\noindent\Large\textbf{Problem Set 2} \\
\normalsize
Alice McKean \\
\today \\

\problem{1a}

I couldn't figure out how to solve this with a system of equations like in 2a so
I just used the results from 1b namely: $x = 1 + 6s \:\:\: y = 1 + s \:\:\: z = 1 + 3s$
to get:
\begin{equation*}
  \{ \: (6y - 5, y, 3y - 2) \: | \: y \in \mathbb{R} \: \}
\end{equation*}
\problem{1b}

To describe a line parametrically you need a starting point $p_0$ and a direction
vector $V$. Thus the line can be described with the following equation: $L = p_0 + sV$.
We use the given points to generate
the direction vector starting at an arbitrary point.

\begin{align*}
  \rowarrowsep=-2pt
  V =
  \begin{gmatrix}[p]
    7 \\
    2 \\
    4
  \end{gmatrix} 
  -
  \begin{gmatrix}[p]
    1 \\
    1 \\
    1
  \end{gmatrix} 
\end{align*}
So the final parametric form of the line $L$ is:
\begin{equation*}
  \rowarrowsep=-2pt
  \{
  \begin{gmatrix}[p]
    1 \\
    1 \\
    1
  \end{gmatrix} 
  +
  s
  \begin{gmatrix}[p]
    6 \\
    1 \\
    3
  \end{gmatrix} 
  \: | \: s \in \mathbb{R} \: \}
\end{equation*}


\problem{2a}

To find the plane $H$ that passes through points $(0, 1, 0)$, $(1, 4, −2)$, and
$(1, 6, 1)$ we solve for a, b, and c in the following equation: $ax + by + cz =
d$. We start by substituting the points into that equation and finding the
reduced row echelon form of its subsequent matrix.
\begin{align*}
  M_H &= 
  \begin{gmatrix}[b]
     1 &  6 &  1 & \BAR &  1 \\
     0 &  1 &  0 & \BAR &  1 \\
     1 &  4 & -2 & \BAR &  1
     \rowops
     \add[-6]{1}{0}
     \add[-4]{1}{2}
  \end{gmatrix}
  \rowEquiv                         
  \begin{gmatrix}[b]
     1 &  0 &  1 & \BAR & -5 \\
     0 &  1 &  0 & \BAR &  1 \\
     1 &  0 & -2 & \BAR & -3
     \rowops
     \add[-1]{0}{2}
     \mult{2}{\cdot-\frac{1}{3}}
  \end{gmatrix}
  \rowEquiv                         
  \begin{gmatrix}[b]
     1 &  0 &  1 & \BAR & -5 \\
     0 &  1 &  0 & \BAR &  1 \\
     0 &  0 &  1 & \BAR & -\frac{2}{3}
     \rowops
     \add[-1]{2}{0}
  \end{gmatrix}
  \\
  &\rowEquiv                         
  \begin{gmatrix}[b]
     1 &  0 &  0 & \BAR & -\frac{13}{3} \\
     0 &  1 &  0 & \BAR &  1 \\
     0 &  0 &  1 & \BAR & -\frac{2}{3}
  \end{gmatrix}
  = E_H
\end{align*}
Thus the solution set that includes those three points is:
$\{ \: (x, y, z) \in \mathbb{R}^3 \: | \: -13x + 3y - 2z = 3 \: \}$. \\

\problem{2b}

To describe a plane parametrically you can start at a point and have two vectors
in the desired directions. Once you have the directions $W, V$ and the starting
point $p_0$ the plane can be described with the following equation: $P = p_0 + sW + tV$.
We use the given points to generate the two directions starting at an arbitrary
point.
\begin{align*}
  \rowarrowsep=-2pt
  W =
  \begin{gmatrix}[p]
    1 \\
    4 \\
    -2
  \end{gmatrix} 
  -
  \begin{gmatrix}[p]
    0 \\
    1 \\
    0
  \end{gmatrix} 
  \:\:\:\:
  V =
  \begin{gmatrix}[p]
    1 \\
    6 \\
    1
  \end{gmatrix} 
  -
  \begin{gmatrix}[p]
    0 \\
    1 \\
    0
  \end{gmatrix} 
\end{align*}
So the final parametric form of the plane $H$ is:
\begin{equation*}
  \rowarrowsep=-2pt
  \{
  \begin{gmatrix}[p]
    0 \\
    1 \\
    0
  \end{gmatrix} 
  +
  s
  \begin{gmatrix}[p]
    1 \\
    3 \\
    -2
  \end{gmatrix} 
  +
  t
  \begin{gmatrix}[p]
    1 \\
    5 \\
    1
    \rowarrowsep=0pt
  \end{gmatrix} 
  \: | \: s, t \in \mathbb{R} \: \}
\end{equation*}

\newpage
\problem{3a}

The main idea behind this proof is similar to the main idea from last weeks
homework. You should use the fact that one of the coefficents is nonzero to write a
parameter in terms of the rest. For the sake of argument assume $a_n$ is nonzero
in the following equation: $a_1x_1 + a_2x_2 + \dots + a_nx_n = d$. That equation
can be rewritten as:
\begin{equation*}
  x_n = \frac{d}{a_n} - \frac{a_1}{a_n}x_1 - \frac{a_2}{a_n}x_2 - \dots - \frac{a_{n-1}}{a_n}x_{n-1}
\end{equation*}
This allows us to express any hyperplane parametrically with $n - 1$ parameters as:
\begin{equation*}
\begin{equation*}
  \rowarrowsep=-2pt
  \begin{gmatrix}[p]
    0 \\
    0 \\
    \vdots \\
    0 \\
    d / a_n
  \end{gmatrix} 
  +
  x_1
  \begin{gmatrix}[p]
    1 \\
    0 \\
    \vdots \\
    0 \\
    a_1 / a_n
  \end{gmatrix} 
  +
  x_2
  \begin{gmatrix}[p]
    0 \\
    1 \\
    \vdots \\
    0 \\
    a_2 / a_n
  \end{gmatrix} 
  + \dots + x_{n-1}
  \begin{gmatrix}[p]
    0 \\
    0 \\
    \vdots \\
    1 \\
    a_{n-1} / a_n
  \end{gmatrix} 
\end{equation*} 
The choice of $a_n$ as the nonzero coefficent was arbitrary and this construction works as
long as one of the coefficents is nonzero. This is gaurenteed by assumption. \qed \\

\problem{3b}

I would expect hyperplanes in $\mathbb{R}^n$ to be $n - 1$ dimensional. I think this because
when $n = 1$ the definition given in the problem says a hyperplane in $R^1$ is a dot ($0$ dimensional).
When $n = 2$ the definition says a hyperplane in $R^2$ is a line ($1$ dimensional). This pattern seems to continue for any $n$. \\

\problem{3c}

In general I suspect you need $n$ points to determine a hyperplane in $R^n$. I think this because I would use one point as the ``center'' and the other $n - 1$ as the directions like I did in problem 2b.
\end{document}