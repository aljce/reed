\documentclass[fleqn]{article}

\usepackage[a4paper, margin=2cm]{geometry}
\usepackage{amsmath}
\usepackage{amssymb}
\usepackage{gauss}
\usepackage[inline]{enumitem}

\setlength{\parindent}{0pt}
\setlength{\mathindent}{0pt}

% Allow for Augmented Matricies
\usepackage{etoolbox}
\makeatletter
\patchcmd\g@matrix
 {\vbox\bgroup}
 {\vbox\bgroup\normalbaselines}% restore the standard baselineskip
 {}{}
\makeatother

\newcommand{\BAR}{%
  \hspace{-\arraycolsep}%
  \strut\vrule % the `\vrule` is as high and deep as a strut
  \hspace{-\arraycolsep}%
}

\newcommand{\squig}[0]{\ensuremath{\rightsquigarrow}}

\newcommand{\problem}[1]{\large\textbf{Problem #1}\normalsize}

\newcommand{\evidence}[1]{\ensuremath{(\hspace{0.2em} \text{#1} \hspace{0.2em})}}
\newcommand{\relation}[1]{\ensuremath{\hspace{0.2em} {{} #1 {}} \hspace{0.2em}}}
\newcommand{\equal}{\relation{=}}
\newcommand{\qed}{\hfill\ensuremath{\square}}

\newcommand{\idF}[1]{\ensuremath{\text{id}(#1)}}
\newcommand{\coordsF}[2]{\ensuremath{[ \: #1 \: ]_{\mathcal{#2}}}}
\newcommand{\rankF}[1]{\ensuremath{\text{rank}(#1)}}
\newcommand{\nullityF}[1]{\ensuremath{\text{nullity}(#1)}}
\newcommand{\matrixRep}[3]{\ensuremath{[ \: #1 \: ]_{\mathcal{#2}}^{\mathcal{#3}}}}

\begin{document}

\noindent\Large\textbf{Problem Set Week 8 Tuesday} \\
\normalsize
Alice McKean \\
\today \\

\problem{1}
\begin{align*}
  \rowarrowsep=-2pt
  \begin{gmatrix}[p]
    1 & -3 \\
    0 & 1
  \end{gmatrix} 
  \begin{gmatrix}[p]
    1 & 0 \\
    0 & 1/11
  \end{gmatrix}
  \begin{gmatrix}[p]
    1 & 0 \\
    3 & 1
  \end{gmatrix}
  \begin{gmatrix}[p]
    1  &  2 &  3 &  4 \\
    -3 & -6 &  2 &  4
  \end{gmatrix}
  =
  \begin{gmatrix}[p]
    1  &  2 &  0 & -4/11 \\
    0  &  0 &  1 & 16/11
  \end{gmatrix}
\end{align*}
\begin{align*}
  \rowarrowsep=-2pt
  E_1 =
  \begin{gmatrix}[p]
    1 & 0 \\
    3 & 1
  \end{gmatrix}
  \:\:\:\: E_2 =
  \begin{gmatrix}[p]
    1 & 0 \\
    0 & 1/11
  \end{gmatrix}
  \:\:\:\: E_3 =
  \begin{gmatrix}[p]
    1 & -3 \\
    0 & 1
  \end{gmatrix} 
\end{align*}

\problem{2a}

In this proof instead of thinking of $r_i$ as elements of $F^n$ we think of
them as elements of $\mathcal{M}_{1 \times n}$.
\begin{align*}
  & d(r_1, \dots, r_i + kr_i', \dots, r_n) \\
  {} & \hspace{0.3em} = \frac{\det{((r_1, \dots, r_i + kr_i', \dots, r_n)B)}}{\det{(B)}}
     && \evidence{Definition of $d$} \\
  {} & \hspace{0.3em} = \frac{\det{(r_1B, \dots, (r_i + kr_i')B, \dots, r_nB)}}{\det{(B)}}
     && \evidence{Matrix multiplication} \\
  {} & \hspace{0.3em} = \frac{\det{(r_1B, \dots, r_iB + k(r_i'B), \dots, r_nB)}}{\det{(B)}}
     && \evidence{Distributivity of matrix multiplication} \\
  {} & \hspace{0.3em} = \frac{\det{(r_1B, \dots, r_iB, \dots, r_nB)} +
              k \det{(r_1B, \dots, r_i'B, \dots, r_nB)}}{\det{(B)}}
     &&  \evidence{Multilinearity of $\det$} \\
  {} & \hspace{0.3em} = \frac{\det{(r_1B, \dots, r_iB, \dots, r_nB)}}{\det(B)} +
              k \frac{\det{(r_1B, \dots, r_i'B, \dots, r_nB)}}{\det{(B)}}
     && \evidence{Algebra} \\
  {} & \hspace{0.3em} = d(r_1, \dots, r_i, \dots, r_n) + k d(r_1, \dots, r_i', \dots, r_n)
     && \evidence{Definition of $d$}
\end{align*}
Note that for the rows $r_i \in \mathcal{M}_{1 \times n}$ of $A$ the
rows of $AB$ are $r_iB \in \mathcal{M}_{1 \times n}$. $\qed$ \\

\problem{2b}

In this proof we think of $r_i$ as elements of $\mathcal{M}_{1 \times n}$ as we
did in (2a). Assume that $r_i = r_j$ for some $i \neq j$.
\begin{align*}
  d(r_1, \dots, r_n)
     &= \frac{\det((r_1, \dots, r_n)B)}{\det(B)}
     && \evidence{Definition of $d$} \\
  {} &= \frac{\det(r_1B, \dots, r_nB)}{\det(B)}
     && \evidence{Matrix multiplication} \\
  {} &= \frac{0}{\det(B)} = 0
     && \evidence{Alternation of $\det$}
\end{align*}
Note that by assumption $r_i = r_j$ so $r_iB = r_jB$. $\qed$ \\

\problem{2c}
\begin{align*}
  d(I_n) = \frac{\det(I_nB)}{\det(B)} = \frac{\det(B)}{\det(B)} = 1
\end{align*}
Note that all the equalities given above require $\det(B) \neq 0$ but that is satisfied
by assumption. $\qed$ \\

\problem{2d}

The above proofs show $d$ is a determinant but by a previous theorem
determinants are unique so $d(A) = \det(A)$. The previous equality
and the definition of $d$ implies $\det(AB) = \det(A)\det(B)$. $\qed$ \\

\problem{3a}

Assume $v \in \mathcal{N}(f)$ with $f(v) = 0$.
Then $v \in \mathcal{N}(g \circ f)$ because
$(g \circ f)(v) = g(f(v)) = g(0) = 0$. \\

Assume $u \in \mathcal{R}(g \circ f)$ with $(g \circ f)(v) = u$ for some $v \in
V$. Then $u \in \mathcal{R}(g)$ because $f(v)$ is an element of $W$. $\qed$ \\

\newpage

\problem{3b}

If $V$ is a subspace of $W$ then $\text{dim} V \leq \text{dim} W$. To prove this
claim assume to the contrary that $\text{dim} V > \text{dim} W$.
Let $B = \{ v_1, v_2, \dots, v_n \}$ be a basis for $V$. Note
that $B$ is a linearly independent subset of $W$ because $V \subseteq W$.
So the dimension of $W$ is at least the dimension of $V$. These facs imply
$\text{dim} V > \text{dim} W \geq \text{dim} V$.
$\rightarrow\leftarrow$ \\

The lemma given above and the previous results imply that
$\nullityF{f} \leq \nullityF{g \circ f}$
and $\rankF{g \circ f} \leq \rankF{f}$.
\begin{align*}
  \rankF{g \circ f}
     &= \text{dim} V - \nullityF{g \circ f}
     && \evidence{Dimension Theorem} \\
  {} &\leq \text{dim} V - \nullityF{f}
     && \evidence{$-\nullityF{f} \geq -\nullityF{g \circ f}$} \\
  {} &= \text{dim} V - (\text{dim} V - \rankF{f})
     && \evidence{Dimension Theorem} \\
  {} &= \rankF{f}
\end{align*} $\qed$

\problem{3c}

$\rankF{AB} = \rankF{L_{AB}} = \rankF{L_A \circ L_B} \leq \rankF{L_A} =
\rankF{A}$ \\

$\rankF{AB} = \rankF{L_{AB}} = \rankF{L_A \circ L_B} \leq \rankF{L_B} = \rankF{B}$
$\qed$ \\

\problem{3d}

First case on $\det(A) = 0$ or $\det(B) = 0$:
\begin{enumerate}[label=\roman*:]
\item Assume $\det(A) = 0$ so $A$ is not invertable and $\rankF{A} < n$.
  This fact coupled with the previous proof implies $\rankF{AB} < A$ and $AB$ is
  not invertable. Therefore $\det(AB) = 0$.
\item Assume $\det(B) = 0$ so $B$ is not invertable and $\rankF{B} < n$.
  This fact coupled with the previous proof implies $\rankF{AB} < n$ so $AB$ is
  not invertable. Therefore $\det(AB) = 0$. $\qed$
\end{enumerate}

\end{document}