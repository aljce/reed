\documentclass[20pt]{article}

\usepackage[a4paper, margin=2cm]{geometry}
\usepackage{titlesec}
\usepackage{amsmath}
\usepackage{enumitem}
\usepackage{times}
\usepackage{tipa}
\usepackage{phonrule}

\setlength\parindent{0pt}
\titlespacing*{\section}{0pt}{0.7\baselineskip}{0.7\baselineskip}
\titleformat*{\section}{\Large\bfseries}

\newcommand{\broad}[1]{/\textipa{#1}/}
\newcommand{\narrow}[1]{[ \textipa{#1} ]}
\newcommand{\english}[1]{$<$#1$>$}
\newcommand{\sk}[0]{{\kern 0.05em}}
\newcommand{\mk}[0]{{\kern 0.1em}}
\newcommand{\smallcapi}[0]{\sk\textsci\sk}
\newcommand{\openo}[0]{\sk O}

\begin{document}

\Large\textbf{PROBLEM SET 5: PHONOLOGY} \\
\normalsize
Alice McKean \\
\today

\section{Narrow Transcription of English}

\textipa{
  f\openo\*r sk\openo\*r \~{\t{3@}}\sk nd "s3v\~{@}n j\sk i\*r\sk z @"goU
  aU\*r "faD\sk\textrhookschwa\sk z b\*rAt f\openo\*r\mk T \~{A}n DI\sk s
  "k\super{h}\~{A}nP\~{@}n\~{@}nt, @ nu "neIS\~{@}n, k\super{h}\~{@}\sk n"sivd
  I\sk n "lI\sk b\textrhookschwa ti, \~{\t{3@}}\sk nd "d3R@ke\sk I\sk t@d t@ D2
  p\super{h}\*rAp\sk @"zIS\~{@}n D\sk\ae t a\textltilde {\kern 0.03em} m\~{3}n
  A\sk\*r k\super{h}\*r\sk I"eIR@d "ikw@\textltilde. naU wi A\sk\*r
  \~{I}n"geI\textdyoghlig d \~{I}\sk n 2 g\sk\*reI\sk t "sIv@\textltilde
  {\kern 0.03em} wO\*r "t3st@N "w3D\textrhookschwa {\kern 0.03em} D\ae t
  "neIS\~{@}n, O\*r "3n\sk i "neIS\~{@}n soU k\super{h}\~{@}n"sivd
  \~{\t{3@}}\sk nd soU "d3R@keI\sk t@d k\super{h}\~{\t{3@}}n lA\sk N I\sk n"du\*r.
}

\section{Gujarati Laterals}

\narrow{l \:l} are separate phonemes because the two words
\narrow{k\super{h}al} \narrow{k\super{h}a\:l} are minimal pairs. There is
evidence of neutralization because the word for pill \narrow{go\:li} and the
word for bullet \narrow{go\:li} are the same.

\section{German Fricatives}

\phonl
  {\narrow{\c{c}}}
  {\phonfeat{-front \\ -cornal}}
  {\phonfeat{-front \\ -syllabic}} \\

This rule involves assimilation because the -front feature is carry on
assimilated from the preceding vowel.

\begin{enumerate}[label=\roman*.]
\item \narrow{ho:\sk x}
\item \narrow{"h\o :\c{c}st5}
\item \narrow{"kY\c{c}\mk @n}
\item \narrow{"\c{c}e:mIS}
\end{enumerate}

\section{Japanese Voiceless Vowels}

\narrow{i \r*i} and \narrow{W \r*W} are allophones of a single phoneme. The
devoiced partner is the restricted allophone. I think this because it occurs
less often in the dataset and the following rule restricts where it can occur. \\

\phonb
  {\phonfeat{+high}}
  {\phonfeat{-voice}}
  {\phonfeat{-voice}}
  {\phonfeat{-voice}} \\

This rule is clearly devocalization assimilation as the voiceless phonemes
around the force the phoneme in the middle to devoice.

\begin{enumerate}[label=\roman*.]
\item \narrow{\^ins\r*Wtanto}
\item \narrow{o\textphi\r*isW}
\item \narrow{Cop:iNgW}
\item \narrow{p\r*ikaso}
\end{enumerate}

After adding the new data to the dataset I would change the rule to ignore long
vowels. Unfortunately there doesn't seem to be a length class in the feature
handout. In this case I made up a natural class to handle \narrow{:} called ``long''. \\

\phonb
  {\phonfeat{+high \\ -long}}
  {\phonfeat{-voice}}
  {\phonfeat{-voice}}
  {\phonfeat{-voice}} \\

  
\end{document}