\documentclass{article}

\usepackage{covington}
\usepackage{tipa}
\usepackage[a4paper, total={7in, 10in}]{geometry}
\usepackage{color}

\setlength\parindent{0pt}

\newcommand{\ipa}[1]{\textipa{#1}}
\newcommand{\krn}[0]{\hspace{0.1em}}
\newcommand{\red}[1]{{\color{red} #1}}
\newcommand{\blue}[1]{{\color{blue} #1}}
\definecolor{olive}{rgb}{0,0.6,0}
\newcommand{\green}[1]{{\color{olive} #1}}
\renewcommand{\thesubsection}{\thesection.\alph{subsection}}

\begin{document}

\textbf{PROBLEM SET 1: MORPHOLOGY} \\
Alice McKean \\
\today

\section{Samoan}

The plural forms of verbs in Samoan are formed by reduplicating the second
to last syllable of the singular verb. Syllables are -CV- or -V-.

\begin{minipage}[t]{0.5\textwidth}
  \begin{examples}
  \item[(1)]
    \gll mate
        CV-CV
    \glt `die SG.SUB'
    \glend
  \item[(3)]
    \gll alofa
        V-CV-CV
    \glt `love SG.SUB'
    \glend
  \item[(5)]
    \gll taoto
        CV-V-CV
    \glt `lie SG.SUB'
    \glend
  \end{examples}
\end{minipage}
\begin{minipage}[t]{0.5\textwidth}
  \begin{examples}
  \item[(2)]
    \gll mamate
        CV-CV-CV
    \glt `die PL.SUB'
    \glend
  \item[(4)]
    \gll alolofa
        V-CV-CV-CV
    \glt `love PL.SUB'
    \glend
  \item[(6)]
    \gll taooto
        CV-V-V-CV
    \glt `lie PL.SUB'
    \glend
  \end{examples}
\end{minipage}

\section{Cree}

\begin{tabular}{rl|ll|ll}
  \multicolumn{2}{c|}{-1 Person} & \multicolumn{2}{c|}{+0 Root} & \multicolumn{2}{c}{+1 Number} \\
  \hline
  \ipa{ni}- & `1' & \ipa{ta:s}     & `pants'        & -\ipa{\o}    & `SG' \\ 
  \ipa{ki}- & `2' & \ipa{stikwa:n} & `head'         & -\ipa{ina:n} & `PL.EXCL' \\
  \ipa{o}-  & `3' & \ipa{imis}     & `older sister' & -\ipa{inaw}  & `PL.INCL' \\
            &     &                &                & -\ipa{iwa:w} & `PL' \\
\end{tabular}

\section{Cahuilla}

\subsection{Morphemes}

The following table is the position class chart and the glosses of almost everything noun
related. The only thing missing is the infix non-concatenative morpheme
-\ipa{P}--\ipa{i} `SG.OBJ'. The glottal stop inserts before the last syllable
where syllables are defined in the same way as Samoan. Then the i is placed
at the end of the word. \\

\begin{tabular}{ll|ll}
  \multicolumn{2}{c|}{+0 Root} & \multicolumn{2}{c}{+1 Suffixes} \\
  \hline
  \ipa{hunwet}    & `bear'   & -\ipa{\o} & `SG.SUB' \\
  \ipa{pakac}     & `mouse'  & -\ipa{em} & `PL.SUB' \\
  \ipa{wikikmalj} & `bird'   & -\ipa{mi} & `PL.OBJ' \\
  \ipa{hunal}     & `badger' &           & \\
  \ipa{Pisilj}    & `coyote' &           & \\
\end{tabular}

\hfill

The following table is the position class chart and the glosses of everything verb
related. `TRS' is the gloss for transitive and `INTRS' is the gloss for
intransitive. \\

\begin{tabular}{rl|rl|ll|ll|ll}
  \multicolumn{2}{c|}{-2 Object} & \multicolumn{2}{c|}{-1 Subject} & \multicolumn{2}{c|}{+0 Root}
                                 & \multicolumn{2}{c|}{+1 Number}  & \multicolumn{2}{c }{+2 Tense} \\
  \hline
  \ipa{ne}-                   & `1SG.OBJ' & \ipa{n}-   & `1SG.SUB.TRS'   & \ipa{mamajaw} & `to help'
                              & -\ipa{qa} & `SG.SUB'   & -\ipa{P}        & `PAST' \\
  \ipa{Pe}-                   & `2SG.OBJ' & \ipa{ne}-  & `1SG.SUB.INTRS' & \ipa{kukatac} & `to talk'
                              & -\ipa{wa} & `PL.SUB'   & -\ipa{\o}       & `PRES' \\
  \{\ipa{pe}\krn, \ipa{pi}\}- & `3SG.OBJ' & \ipa{P}-   & `2SG.SUB.TRS'   & \ipa{jawici}  & `to carry'
                              &           &            &                 & \\
  \ipa{ceme}-                 & `1PL.OBJ' & \ipa{Pe}-  & `2SG.SUB.INTRS' & \ipa{kicuNi}  & `to kiss'
                              &           &            &                 & \\
  \ipa{Peme}-                 & `2PL.OBJ' & \ipa{\o}-  & `3SG.SUB'       & \ipa{haPtis}  & `to sneeze'
                              &           &            &                 & \\
  \{\ipa{me}\krn, \ipa{mi}\}- & `3PL.OBJ' & \ipa{cem}- & `1PL.SUB'       & \ipa{naqma}   & `to hear'
                              &           &            &                 & \\
  \ipa{\o}-                   & `INTRS'   & \ipa{Pem}- & `2PL.SUB'       & \ipa{te:w}    & `to see'
                              &           &            &                 & \\
                              &           & \ipa{m}-   & `3PL.SUB.TRS'   & \ipa{nac}     & `to sit'
                              &           &            &                 & \\
                              &           & \ipa{hem}- & `3PL.SUB.INTRS' & \ipa{taxmu}   & `to sing'
                              &           &            &                 & \\
                              &           &            &                 & \ipa{hiN}     & `to fly'
                              &           &            &                 & \\
                              &           &            &                 & \ipa{qwaP}    & `to eat'
                              &           &            &                 & \\
\end{tabular}

\subsection{Grammar}

Cahuilla appears to have a noun class system for the number. In the following
examples when the number agrees between the verb and noun it is colored.

\begin{examples}
  \item
    \gll \ipa{hunwet-em}   \ipa{\o-hem-nac-we-P}
        bear-\red{PL.SUB}  INTRS-3PL.SUB.INTRS-sit-\red{PL.SUB}-PAST
    \glt `The bears sat down'
    \glend
  \item
    \gll \ipa{pakac-\o}    \ipa{hunwet-mi}    \ipa{me-\o-te:w-qa-\o}
        mouse-\red{SG.SUB} bear-\blue{PL.OBJ} 3\blue{PL.OBJ}-3\red{SG.SUB}-see-\red{SG.SUB}-PRES
    \glt `The mouse sees the bears'
    \glend
\end{examples}

\par

Unlike English Cahuilla does not require a SVO typology. Instead sentences in
Cahuilla appear to be either SOV or OSV. In the following illustrative examples
subjects are colored \red{red}, objects are colored \blue{blue} and verbs are
colored \green{green}. Example (3) is clearly OSV and example (4) is clearly SOV.
This pattern holds for every sentence in the dataset.



\begin{examples}
  \item
    \gll \blue{\ipa{paka-P-c-i}}  \red{\ipa{wikikmalj-\o}}  \green{\ipa{pe-\o-naqma-qa-\o}}
         mouse-SG.OBJ             bird-SG.SUB               3SG.OBJ-3SG.SUB-hear-SG.SUB-PRES
    \glt `The mouse sees the bird'
    \glend
  \item
    \gll \red{\ipa{hunwet-\o}}    \blue{\ipa{Pisi-P-lj-i}}    \green{\ipa{pe-\o-kicuNi-qa-\o}}
         bear-SG.SUB              coyote-SG.OBJ            3SG.OBJ-3SG.SUB-kiss-SG.SUB-PRES
    \glt `The bear kisses the coyote'
    \glend
\end{examples}

\end{document}