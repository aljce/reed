\documentclass[20pt]{article}

\usepackage[a4paper, margin=2cm]{geometry}
\usepackage{titlesec}
\usepackage{amsmath}
\usepackage{amssymb}
\usepackage{enumitem}
\usepackage{times}
\usepackage{tipa}
\usepackage{phonrule}
\usepackage{covington}

\setlength\parindent{0pt}
\titlespacing*{\section}{0pt}{0.7\baselineskip}{0.7\baselineskip}
\titleformat*{\section}{\large\bfseries}

\newcommand{\broad}[1]{/\textipa{#1}/}
\newcommand{\narrow}[1]{[ \textipa{#1} ]}
\newcommand{\english}[1]{$<$#1$>$}
\newcommand{\sk}[0]{{\kern 0.05em}}
\newcommand{\mk}[0]{{\kern 0.1em}}
\newcommand{\smallcapi}[0]{\sk\textsci\sk}
\newcommand{\openo}[0]{\sk O}
\newcommand{\marked}[1]{\textsc{#1}}
\newcommand{\trans}[3]{\lq #1\rq \mk \narrow{#2} $\rightsquigarrow$ \narrow{#3}}

\begin{document}

\Large\textbf{PROBLEM SET 6: PHONOLOGY II} \\
\normalsize
Alice McKean \\
\today

\section{Rules}

\textbf{\textipa{P}-epenthesis}

\phonb{$\emptyset$}{\narrow{P}}{[$_{wd}$}{[+syllabic][-syllabic]}

Samoan marks (see below for exceptions) non CV syllables and this rule repairs
single naked vowels at the beginning of words such as \trans{ice}{aisa}{Paisa}
and \trans{apply}{apalai}{Papalai}. \\

\textbf{glottalization}

\phonb{\narrow{h}}{\narrow{P}}{}{}

The only glottal sound available in Samoan is \narrow{P} so \narrow{h} always
transforms into one. In the data set the words for \\
\trans{half}{hafa}{Pafa} and \trans{hamburger}{hamepeka}{Pamepeka}
illustrate this transformation.  \\

\textbf{stop-devoicing}
\vspace{0.1em}

\phonb
  {\phonfeat{-sonorant \\ -continuant}}
  {[ -voice ]}
  {}
  {}
\vspace{0.3em}
  
Samoan lacks voiced plosives and so devoices any voiced plosive such as
\narrow{b}, \narrow{d}, or \narrow{g}. This can been seen in the word for
\trans{gear}{gia}{kia} and in the word for \trans{deliver}{diliva}{tiliva} \\

\textbf{lateralization}
\vspace{0.2em}

\phonb
  {\phonfeat{+consonantal \\ +approximant}}
  {[ +lateral ]}
  {}
  {}
\vspace{0.3em}

All liquids lateralize and transform into \narrow{l}. This rule transforms
\trans{rummy}{\*rami}{lami} and many other words. \\

\textbf{glide-repair}
\vspace{0.2em}
  
\phonb
  {\phonfeat{-syllabic \\ -consonantal}}
  {[ +syllabic ]}
  {}
  {}
\vspace{0.3em}

Glides are transformed into their respective vowel. This occurs in words such as
\trans{walkathon}{wakafoni}{uakafoni} and
\trans{computer}{komipjuta}{komipiuta}. This rule is solid proof that Samoan
allows for VV syllables. This means that a triphthong isn't needed to syllabized
\narrow{sa.ie.ni.ti.si} and every word in the data set can by syllabized into CV
or VV. This means Samoan follows two syllable markedness rules
\marked{$*$complex onset} and \marked{nocoda}. \\
  
\textbf{th-fronting}
\vspace{0.1em}

\phonb
  {\phonfeat{-labial \\ -labiodental \\ +coronal}}
  {\phonfeat{+labial \\ +labiodental \\ -coronal}}
  {}
  {} \\

\textbf{anteriorization}
\vspace{0.2em}

\phonb
  {\phonfeat{+strident \\ -anterior \\ -voice}}
  {\phonfeat{+anterior \\ +voice}}
  {}
  {} \\
 
  
\textbf{boundary-deletion}

\phonb
  {[ -syllabic ]}
  {$\emptyset$}
  {[ -syllabic ]}
  {]$_{wd}$} \\

\textbf{boundary-insertion}

\phonb
  {$\emptyset$}
  {V}
  {[ -syllabic ]}
  {]$_{wd}$} \\
 
\textbf{syllable-repair}

\phonb
  {$\emptyset$}
  {V}
  {[ -syllabic ]}
  {[ -syllabic ]} \\
 
  
\end{document}
