\documentclass[12pt]{article}
\usepackage{setspace}
\usepackage[margin=1in]{geometry}

\title{On Doors and Sexuality}
\author{Kyle McKean}

\begin{document}
\maketitle
\setlength{\parindent}{.5in}
\doublespacing

Gilgamesh and Enkidu's relationship begins with a passionate sex scene inside a
doorway. On the surface the sex scene is just a fight but it is an extended
metaphor for sex described by the human sexual response cycle.
The human sexual response cycle follows four phases sexual excitement, plateau,
orgasm, and resolution. If all of the phases appear in the fight, it would
follow that the fight is actually a sexual interaction. This sexual intercourse
catapults their relationship from strangers to friends.
Doors in \textit{The Epic of Gilgamesh} mark a sexual interaction between
Gilgamesh and Enkidu, and this sexual interaction develops their relationship.

\par

Gilgamesh and Enkidu's fight begins with sexual excitement, the first phase of
the human sexual response cycle. Sexual excitement is characterized by kissing
and physical contact. Both of these are seen at the beginning of the fight
between Gilgamesh and Enkidu.
``Like a babe-in-arms they where [kissing his feet,]...''(II-107) Although
this was not Gilgamesh kissing Enkidu it sets the sexual scene. The beginning of
their sexual interaction can be seen in this quote, ``They seized
each other at the door...''(II-113). The word seized means to take possession of,
Gilgamesh and Enkidu are taking possession of each others bodies. Rising tension
characterizes sexual excitement and tension builds in the above quote.

\par

Plateau follows sexual excitement in the human sexual response cycle and tension
plateaus as the passage reads on. As opposed to the previous line this quote
``...in the street they joined combat...''(II-114) is devoid of passionate words like
seized. The sentence also rapidly moves the plot forward without providing any
detail about the fight itself. This follows the clique that men are only
interested in achieving orgasm. This quote repeats at line II-116 which
signifies Gilgamesh and Enkidu entering the plateau phase at different times. 

\par

Orgasm culminates Gilgamesh and Enkidu's fight with intensity and climax.
``The door-jambs shook, the walls did shudder...''(II-115) This quotes diction
flows from the definition of an orgasm. They are characterized by ``shaking''
and ``shuddering'', called involuntary muscle spasms. This line offers more than
diction, it climaxes the fight. This line is the climax because it contains
striking imagery at a level above the rest of the passage. The quote also
repeats which represents Gilgamesh and Enkidu climaxing at different times during
the sexual intercourse.

\par

Gilgamesh and Enkidu's fight ends with resolution. The physical sex ends with
``...his anger subsided, he broke off from the fight.''(P-230) Anger is a phallic
metaphor, during the resolution penises become flaccid, in other-words they
subside. Resolution is also marked by a change in position, in this case
Gilgamesh breaks off from the fight.
Colloquially Gilgamesh rolls off Enkidu. Resolution often induces a
bonding effect in the sexual partners which can be seen in this quote, ``They
kissed each other and formed a friendship.''(Y-18)

\par

Before the sex scene of Gilgamesh and Enkidu's first interaction the pair were
strangers and needed to be convinced to meet. The reader knows that Gilgamesh
is unacquainted with Enkidu because Gilgamesh's mother speaks mythically about
Enkidu. ``His strength is as mighty as a rock from the sky.'' (I-293)
Mythically describes that quote
accurately because many early cultures viewed meteors as messages from the gods.
Gilgamesh's mother describes Enkidu in this way in order to give him a
significant reason to meet him. In a similar vein the reader
knows that Enkidu is unacquainted with Gilgamesh because Shamhat describes
Gilgamesh with hyperbole. ``Where Gilgamesh is perfect in strength...'' (I-211)
Gilgamesh's mother and Shamhat built up Gilgamesh and Enkidu to tense alienating
levels before their meeting.

\par

The sexual interaction changed Gilgamesh and Enkidu's relationship from
apprehensive knowledge of the other to friendship. ``They kissed each other
and formed a friendship.'' (Y-18) The scene is also a cathartic release from
Gilgamesh's mother and Shamhat's build up of the pair. Gilgamesh's describes
Enkidu to his mother accurately and without embellishment.
``Shaggy hair hanging loose... he was born in the wild.'' (II-176) 

\par

Doors signify sexual interaction between the pair Gilgamesh and Enkidu, and that
sexual interaction drives each other from stranger to friend. That sexual
interaction follows the human sexual response cycle, allowing it to be
classified as a sexual interaction. The cathartic resolution of the human sexual
response cycle changed Gilgamesh and Enkidu's opinion of each other allowing
them to become friends. 

\end{document}