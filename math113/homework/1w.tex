 \documentclass[11pt]{article}
 \usepackage{url}
 \usepackage{hyperref}
  \usepackage{amssymb}
  \usepackage{enumerate}
\usepackage{amsmath}
\usepackage{amsthm}
\usepackage{epstopdf}
\usepackage{mathrsfs}
 \usepackage{amsfonts}
 \usepackage{color}
\usepackage{amsthm}
\usepackage{graphicx}
\usepackage{mathrsfs}
  \addtolength{\topmargin}{-0.60in}
  \addtolength{\textheight}{2\baselineskip}
  \setlength{\oddsidemargin}{-20pt}
  \setlength{\evensidemargin}{-10pt}
  \setlength{\textwidth}{6.5in}
  \setlength{\textheight}{9.5in}
  \setlength{\parindent}{0pt}
  \setlength{\leftmargini}{0pt}
\pagenumbering{gobble}

\newcommand{\coursenumber}{113} 
\newcommand{\coursename}{Discrete Structures}
\newcommand{\classroom}{LIB 204}
\newcommand{\classtime}{MWF 10-10:50}
\newcommand{\crn}{11176}
\newcommand{\quarter}{Fall 2017} 
 \pagestyle{myheadings}
  \markright{\sc MATH \coursenumber --- \quarter --- S. Chettih}

%###############################################################################

\begin{document}

%###############################################################################
%
\thispagestyle{plain}
%
\addtolength{\topmargin}{-.40in}
\centerline{
  \framebox{\parbox{4.0in}{
    \vspace{6pt}
     \begin{centering}
             \Large       \bf \coursename \\
             \normalsize  \bf MATH \coursenumber---CRN \crn \\[1pt]
             \normalsize  \bf \quarter           \\[1pt]
      \end{centering}
    \vspace{6pt}
  }}
}
%
\vspace{5pt}

%===============================================================================
From LPV:
\begin{description}
\item[Reading:] Section 1.1 and 1.2
\item[1.1.2:] What is the number of pairings in Carl's sense (when it matters who sits on which side of the board, but the boards are all alike), and in Diane's sense (when it is the other way around)?
\item[1.2.8] List all subsets of $\{a,b,c,d,e\}$ containing $a$ but not $b$.

\item[1.2.9] Define a set of which both $\{1,3,4\}$ and $\{0,3,5\}$ are subsets. Find such a set with the smallest possible number of elements.

\item[1.2.12] We form the union of a set with 5 elements and a set with 9 elements. Which of the following numbers can we get as the cardinality of the unions: 4, 6, 9, 10, 14, 20?
\end{description}
%=============================================================================

%###############################################################################

\end{document}

%###############################################################################

