\documentclass[fleqn]{article}

\usepackage[margin=2cm]{geometry}
\usepackage{amsmath}
\usepackage{amssymb}
\usepackage{gauss}
\usepackage{enumitem}
\usepackage{qcircuit}
\usepackage{algorithm}
\usepackage[noend]{algpseudocode}

\setlength{\parindent}{0pt}
\setlength{\mathindent}{0pt}

% Allow for Augmented Matricies
\usepackage{etoolbox}
\makeatletter
\patchcmd\g@matrix
 {\vbox\bgroup}
 {\vbox\bgroup\normalbaselines}% restore the standard baselineskip
 {}{}
\makeatother

\newcommand{\BAR}{%
  \hspace{-\arraycolsep}%
  \strut\vrule % the `\vrule` is as high and deep as a strut
  \hspace{-\arraycolsep}%
}

\newcommand{\squig}[0]{\ensuremath{\rightsquigarrow}}

\newcommand{\problem}[1]{{\large\textbf{Problem #1}}}
\newcommand{\lemma}[2]{\textbf{Lemma #1} #2}

\newcommand{\evidence}[1]{\ensuremath{(\hspace{0.2em} \text{#1} \hspace{0.2em})}}
\newcommand{\relation}[1]{\ensuremath{\hspace{0.2em} {{} #1 {}} \hspace{0.2em}}}
\newcommand{\equal}{\relation{=}}
\newcommand{\qed}{\hfill\ensuremath{\square}}

\newcommand{\idF}[1]{\ensuremath{\text{id}(#1)}}
\newcommand{\trF}[1]{\ensuremath{\text{tr}\left(#1\right)}}
\newcommand{\coordsF}[2]{\ensuremath{[ \: #1 \: ]_{\mathcal{#2}}}}
\newcommand{\rankF}[1]{\ensuremath{\text{rank}(#1)}}
\newcommand{\nullityF}[1]{\ensuremath{\text{nullity}(#1)}}
\newcommand{\matrixRep}[3]{\ensuremath{{\left [ \: #1 \: \right ]}_{\mathcal{#2}}^{\mathcal{#3}}}}
\newcommand{\signF}[1]{\ensuremath{\text{sign}(#1)}}
\usepackage{listofitems}
% Cycle Notation
\newcommand\cycleF[2][\:]{
  \readlist\thecycle{#2}
  #1\foreachitem\i\in\thecycle{\ifnum\icnt=1\else#1\fi\i}#1
}
\newcommand{\normF}[1]{\left\lVert#1\right\rVert}

\newcommand{\bra}[1]{\ensuremath{\langle #1 |}}
\newcommand{\ket}[1]{\ensuremath{| #1 \rangle}}
\newcommand{\innerF}[2]{\ensuremath{\langle #1 | #2 \rangle}}
\newcommand{\outerF}[2]{\ket{#1} \bra{#2}}
\newcommand{\modF}[2]{\ensuremath{#1 \, \text{mod} \, #2}}
\newcommand{\ordF}[2]{\ensuremath{\text{ord}_{#1}\left( #2 \right)}}
\newcommand{\expF}[1]{\ensuremath{\text{exp}\left(#1\right)}}

\begin{document}

\noindent\Large\textbf{Final Project} \\
\normalsize
Alice McKean \\
\today \\

In this project we will explore the Discrete Fourier Transform. DFTs are a variant of the
Continuous Fourier Transform over a finite number of samples. We define the kth
element of a DFT as follows. Note that $x_k$ is the kth sample and $\omega_{N}$
is the nth root of unity.
\begin{align*}
  X_k = \sum_{j = 0}^{N - 1} x_j \cdot \omega_N^{\, j k} \hspace{3em}
  \omega_{N}^{\, x} = \expF{\frac{- i \, \tau \, x}{N}}
\end{align*}
This definition provides an obvious algorithm to compute the DFT of an array of
samples.
\begin{algorithm}[H]
  \caption{Quadratic DFT}\label{euclid}
  \begin{algorithmic}
    \Function{DFT}{$x$}
    \State $n \leftarrow \textbf{length } x$
    \State $r \leftarrow \textbf{new array}[n]$
    \For{$k \leftarrow 0 \textbf{ to } n - 1$}
      \State $s \leftarrow 0$
      \For{$j \leftarrow 0 \textbf{ to } n - 1$}
        \State $\omega \leftarrow \expF{- i \, \tau \, j \, k / n}$
        \State $s \leftarrow s + x[j] \, \omega$
      \EndFor
      \State $r[k] \leftarrow s$
    \EndFor
    \State \textbf{return} $r$
    \EndFunction
  \end{algorithmic}
\end{algorithm}
Unfortunately the nested loops cause this algorithm to run in $O(n^2)$ time
assuming primitive mathematical operations run in $O(1)$ time. This being said the Discrete
Fourier Transform contains much structure that can be exploited to improve the
runtime. First notice that when you square the nth roots of unity you
effectively collapse half of them together:
\begin{align*}
  \omega_{N}^{\, 2 x}
  = \expF{\frac{- i \, \tau \, 2 x}{N}}
  = \expF{\frac{- i \, \tau \, x}{N / 2}}
  = \omega_{N / 2}^{\, x}
\end{align*}
This insight lights a path forward, split the samples into even and odd numbered
samples. Then you can compute two DFTs that are half the size. This idea is
formalized in the lemma below:
\begin{align*}
  X_k =& \sum_{j = 0}^{N - 1} x_j \cdot \omega_N^{\, j k} \\
      =& \sum_{j = 0}^{N / 2 - 1} x_{2 j} \cdot \omega_N^{\, 2 j k}
       + \sum_{j = 0}^{N / 2 - 1} x_{2 j + 1} \cdot \omega_N^{\, \left( 2 j + 1 \right) k} \\
      =& \left( \sum_{j = 0}^{N / 2 - 1} x_{2 j} \cdot \omega_N^{\, 2 j k} \right)
       + \omega_N^{\, k} \left( \sum_{j = 0}^{N / 2 - 1} x_{2 j + 1} \cdot \omega_N^{\, 2 j k} \right) \\
      =& \left( \sum_{j = 0}^{N / 2 - 1} x_{2 j} \cdot \omega_{N / 2}^{\, j k} \right)
       + \omega_N^{\, k} \left( \sum_{j = 0}^{N / 2 - 1} x_{2 j + 1} \cdot \omega_{N / 2}^{\, j k} \right)
\end{align*}
The subproblems at the end of the lemma above range from $k = 0$ to $k = N / 2 -
1$. We can use the periodicity of the roots of unity to compute the subproblems $k = N / 2$ to
$k = N$ for free. This periodicity is shown below:
\begin{align*}
  \omega_{N}^{j \left( k + N / 2 \right)}
  = \expF{\frac{- i \, \tau j \left( k + N / 2 \right)}{N}}
  = \expF{\frac{- i \, \tau \, j k}{N}} \expF{\frac{- i \, \tau \, j}{2}}
  = - \expF{\frac{- i \, \tau \, j k}{N}} = - \omega_{N}^{\, j k}
\end{align*}

\begin{align*}
  X_{k + N / 2}
  =& \left( \sum_{j = 0}^{N / 2 - 1} x_{2 j} \cdot \omega_{N / 2}^{\, j \left( k + N / 2 \right)} \right)
   + \omega_N^{\, k + N / 2} \left( \sum_{j = 0}^{N / 2 - 1} x_{2 j + 1} \cdot \omega_{N / 2}^{\, j \left( k + N / 2 \right)} \right) \\
  =& \left( \sum_{j = 0}^{N / 2 - 1} x_{2 j} \cdot \omega_{N / 2}^{\, j k} \right)
   - \omega_N^{\, k} \left( \sum_{j = 0}^{N / 2 - 1} x_{2 j + 1} \cdot \omega_{N / 2}^{\, j k} \right)
\end{align*}
The DFT of a single sample is just the sample which provides a base case for the
following recursive algorithm. Note that the function $SPLIT$ returns
two arrays, the even entries of the input and the odd entries of the input.
\begin{algorithm}[H]
  \caption{Fast Fourier Transform}\label{euclid}
  \begin{algorithmic}
    \Function{FFT}{$x$}
    \State $n \leftarrow \textbf{length } x$
    \If{$n \leq 1$}
      \State \textbf{return} $x$
    \EndIf
    \State $e ,\, o \leftarrow \Call{SPLIT}{x}$
    \State $fe ,\, fo \leftarrow \Call{FFT}{e} ,\, \Call{FFT}{o}$
    \State $r \leftarrow \textbf{new array}[n]$
    \For{$k \leftarrow 0 \textbf{ to } n / 2$}
      \State $\omega \leftarrow \expF{- i \, \tau \, k / n}$
      \State $r[k] \leftarrow fe[k] + \omega * fo[k]$
      \State $r[k + n / 2] \leftarrow fe[k] - \omega * fo[k]$
    \EndFor
    \State \textbf{return} $r$
    \EndFunction
  \end{algorithmic}
\end{algorithm}

This algorithm's runtime is modeled by the recurrence, $T(n) = 2 * T(n / 2) +
O(n)$. This recurrence can be solved by the master method to show a runtime of
$O(n \, \text{lg}(n))$. Unfortunately this algorithm requires the input length
to be a power of 2. This problem can be solved by padding the input with zeros
until the length reaches a power of 2.

\end{document}

