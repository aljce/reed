\documentclass[fleqn]{article}

\usepackage[margin=2cm]{geometry}
\usepackage{amsmath}
\usepackage{amssymb}
\usepackage{gauss}
\usepackage{enumitem}
\usepackage{qcircuit}

\setlength{\parindent}{0pt}
\setlength{\mathindent}{0pt}

% Allow for Augmented Matricies
\usepackage{etoolbox}
\makeatletter
\patchcmd\g@matrix
 {\vbox\bgroup}
 {\vbox\bgroup\normalbaselines}% restore the standard baselineskip
 {}{}
\makeatother

\newcommand{\BAR}{%
  \hspace{-\arraycolsep}%
  \strut\vrule % the `\vrule` is as high and deep as a strut
  \hspace{-\arraycolsep}%
}

\newcommand{\squig}[0]{\ensuremath{\rightsquigarrow}}

\newcommand{\problem}[1]{{\large\textbf{Problem #1}}}
\newcommand{\lemma}[2]{\textbf{Lemma #1} #2}

\newcommand{\evidence}[1]{\ensuremath{(\hspace{0.2em} \text{#1} \hspace{0.2em})}}
\newcommand{\relation}[1]{\ensuremath{\hspace{0.2em} {{} #1 {}} \hspace{0.2em}}}
\newcommand{\equal}{\relation{=}}
\newcommand{\qed}{\hfill\ensuremath{\square}}

\newcommand{\idF}[1]{\ensuremath{\text{id}(#1)}}
\newcommand{\trF}[1]{\ensuremath{\text{tr}\left(#1\right)}}
\newcommand{\coordsF}[2]{\ensuremath{[ \: #1 \: ]_{\mathcal{#2}}}}
\newcommand{\rankF}[1]{\ensuremath{\text{rank}(#1)}}
\newcommand{\nullityF}[1]{\ensuremath{\text{nullity}(#1)}}
\newcommand{\matrixRep}[3]{\ensuremath{{\left [ \: #1 \: \right ]}_{\mathcal{#2}}^{\mathcal{#3}}}}
\newcommand{\signF}[1]{\ensuremath{\text{sign}(#1)}}
\usepackage{listofitems}
% Cycle Notation
\newcommand\cycleF[2][\:]{
  \readlist\thecycle{#2}
  #1\foreachitem\i\in\thecycle{\ifnum\icnt=1\else#1\fi\i}#1
}
\newcommand{\normF}[1]{\left\lVert#1\right\rVert}

\newcommand{\bra}[1]{\ensuremath{\langle #1 |}}
\newcommand{\ket}[1]{\ensuremath{| #1 \rangle}}
\newcommand{\innerF}[2]{\ensuremath{\langle #1 | #2 \rangle}}
\newcommand{\outerF}[2]{\ket{#1} \bra{#2}}

\begin{document}

\noindent\Large\textbf{Problem Set 5} \\
\normalsize
Alice McKean \\
\today \\

\problem{1} \\
The probability that the given algorithm succeeds is shown below:
\vspace{-0.5em}
\begin{align*}
  P(M_1 \, | \, \psi_1) &= \innerF{0}{0}\innerF{0}{0} = 1 \\
  P(M_2 \, | \, \psi_2) &= \innerF{+}{1}\innerF{1}{+} = 1/2 \\
  P(\text{success}) &= P(\psi_1 | M_1) P(M_1) + P(\psi_2 | M_2) P(M_2) \\
  &= \frac{P(M_1 | \psi_1) P(\psi_1) P(M_1)}{P(M_1)} + \frac{P(M_2 | \psi_2)P(\psi_2)P(M_2)}{P(M_2)} \\
  &= P(M_1 | \psi_1) P(\psi_1) + P(M_2 | \psi_2) P(\psi_2) \\
  &= 1 * 1/2 + 1/2 * 1/2 = 3/4
\end{align*}

\problem{2} \\
My algorithm is to measure with the haddamard basis $\{ \ket{+} , \ket{-} \}$.
If you see a $\ket{+}$ choose $\psi_3$ and if you see a $\ket{-}$ choose $\psi_2$.
The reasoning behind this is
$\bra{\psi_3}M_{+}\ket{\psi_3}$ is very close to $1$ and similarly for $\bra{\psi_2}M_{-}\ket{\psi_2}$.
\begin{align*}
  P(M_+ | \psi_2) &= \innerF{\psi_2}{+}\innerF{+}{\psi_2} \approx 0.066 \\ 
  P(M_- | \psi_2) &= \innerF{\psi_2}{-}\innerF{-}{\psi_2} \approx 0.933 \\ 
  P(M_+ | \psi_3) &= \innerF{\psi_3}{+}\innerF{+}{\psi_3} \approx 0.933 \\ 
  P(M_- | \psi_3) &= \innerF{\psi_3}{-}\innerF{-}{\psi_3} \approx 0.066 \\ 
  P(\text{success}) &= P(G_1 | M_{+})P(M_{+} | \psi_1)P(\psi_1) + P(G_1 | M_{-})P(M_{-} | \psi_1)P(\psi_1) \\
                    &+ P(G_2 | M_{+})P(M_{+} | \psi_2)P(\psi_2) + P(G_2 | M_{-})P(M_{-} | \psi_2)P(\psi_2) \\
                    &+ P(G_3 | M_{+})P(M_{+} | \psi_3)P(\psi_3) + P(G_3 | M_{-})P(M_{-} | \psi_3)P(\psi_3) \\
                    &\approx 2/3 
\end{align*}

\problem{2.64} \\
Any subspace formed by a linearly independent set of vectors
$\{ \, \ket{\psi_i} \, , \, \ket{\psi_2} \, , \, \dots \, , \, \ket{\psi_m} \, \}$ 
has a orthonormal basis
$\{ \, \ket{\phi_i} \, , \, \ket{\phi_2} \, , \, \dots \, , \, \ket{\phi_m} \, \}$ 
via Gram-Schmidt.
The first $1 \leq i \leq m$ measurements are defined by
$E_i = \outerF{\phi_i}{\phi_i}$ with the last measurement defined by
$E_{m + 1} = I - \sum_i E_i$. This definition trivially satisfies the
completeness condition. The first $1 \leq i \leq m$ measurements are
also clearly positive operators. Therefore if the measurement $E_{m + 1}$
is a positive operator the measurements form a POVM. Note that
$\sum_i E_i$ is a projection. The following equality proves $E_{m + 1}$ is
a projection and therefore a positive operator:
\begin{align*}
  E_{m + 1}^2
  = \left( I - \sum_i E_i \right) \left( I - \sum_i E_i \right)
  = I I - I \left( \sum_i E_i \right) - \left( \sum_i E_i \right) I + \sum_i E_i \sum_i E_i
  = I - \sum_i E_i = E_{m + 1}
\end{align*}
Note that
$\bra{\psi_i}E_j\ket{\psi_i}
= \innerF{\psi_i}{\phi_j}\innerF{\phi_j}{\psi_i} = \innerF{\psi_i}{\phi_j} {\innerF{\psi_i}{\phi_j}}^*
\geq 0$ with equality if and only if the vectors are orthogonal ($i \neq j$).
Therefore $\bra{\psi_i}E_j\ket{\psi_i} = 0$ for $i \neq j$ and
$\bra{\psi_i}E_i\ket{\psi_i} > 0$ so if you measure $E_i$ for $1 \leq i \leq m$
you know the state was in $\ket{\psi_i}$ with certainty. If you measure
$E_{m + 1}$ you unfortunately get no information. \\ 

\problem{2.71} \\
The following inequality is true because $0 \leq p_i \leq 1$ forall $i$:
\begin{align*}
  \trF{\rho^2}
  = \trF{\left( \sum_i p_i \, \outerF{\psi_i}{\psi_i} \right)^2}
  = \trF{\sum_i p_i^2 \, \outerF{\psi_i}{\psi_i}}
  = \sum_i p_i^2 \leq \sum_i p_i = \trF{\sum_i p_i \, \outerF{\psi_i}{\psi_i}}
  = \trF{\rho} = 1
\end{align*}
To prove the forward implication, assume $\trF{\rho^2} = 1$ so
$\sum_i p_i^2 = \sum_i p_i = 1$ with $0 \leq p_i \leq 1$.
The proof proceeds from the following two mutually exclusive cases:
\begin{enumerate}[label=\Roman*.]
  \item Assume there exists some $p_j$ such that $0 < p_j < 1$: A consequence of
    this assumption is that $p_j^2 < p_j$ so $1 = \sum_i p_i^2 < \sum_i p_i = 1$
    a contradiction.
  \item Otherwise for any $p_i$ either $p_i = 0$ or $p_i = 1$:
    As $\sum_i p_i = 1$ there exists some $p_j = 1$ with all other $p_i = 0$.
    This shows the state can be written as a pure state $\{ p_j , \, \ket{\psi_j} \}$
\end{enumerate}

The reverse implication is easier, assume $\rho$ is a pure state
$\rho = \{ 1 , \ket{\psi} \}$:
\begin{align*}
  \trF{\rho^2} = \trF{\left( 1 \outerF{\psi}{\psi} \right)^2} = \trF{1 \outerF{\psi}{\psi}} = 1
\end{align*}

\end{document}

