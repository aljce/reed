\documentclass[fleqn]{article}

\usepackage[margin=2cm]{geometry}
\usepackage{amsmath}
\usepackage{amssymb}
\usepackage{gauss}
\usepackage[inline]{enumitem}
\usepackage{qcircuit}

\setlength{\parindent}{0pt}
\setlength{\mathindent}{0pt}

% Allow for Augmented Matricies
\usepackage{etoolbox}
\makeatletter
\patchcmd\g@matrix
 {\vbox\bgroup}
 {\vbox\bgroup\normalbaselines}% restore the standard baselineskip
 {}{}
\makeatother

\newcommand{\BAR}{%
  \hspace{-\arraycolsep}%
  \strut\vrule % the `\vrule` is as high and deep as a strut
  \hspace{-\arraycolsep}%
}

\newcommand{\squig}[0]{\ensuremath{\rightsquigarrow}}

\newcommand{\problem}[1]{{\large\textbf{Problem #1}}}
\newcommand{\lemma}[2]{\textbf{Lemma #1} #2}

\newcommand{\evidence}[1]{\ensuremath{(\hspace{0.2em} \text{#1} \hspace{0.2em})}}
\newcommand{\relation}[1]{\ensuremath{\hspace{0.2em} {{} #1 {}} \hspace{0.2em}}}
\newcommand{\equal}{\relation{=}}
\newcommand{\qed}{\hfill\ensuremath{\square}}

\newcommand{\idF}[1]{\ensuremath{\text{id}(#1)}}
\newcommand{\coordsF}[2]{\ensuremath{[ \: #1 \: ]_{\mathcal{#2}}}}
\newcommand{\rankF}[1]{\ensuremath{\text{rank}(#1)}}
\newcommand{\nullityF}[1]{\ensuremath{\text{nullity}(#1)}}
\newcommand{\matrixRep}[3]{\ensuremath{{\left [ \: #1 \: \right ]}_{\mathcal{#2}}^{\mathcal{#3}}}}
\newcommand{\signF}[1]{\ensuremath{\text{sign}(#1)}}
\usepackage{listofitems}
% Cycle Notation
\newcommand\cycleF[2][\:]{
  \readlist\thecycle{#2}
  #1\foreachitem\i\in\thecycle{\ifnum\icnt=1\else#1\fi\i}#1
}
\newcommand{\normF}[1]{\left\lVert#1\right\rVert}

\newcommand{\bra}[1]{\ensuremath{\langle #1 |}}
\newcommand{\ket}[1]{\ensuremath{| #1 \rangle}}
\newcommand{\innerF}[2]{\ensuremath{\langle #1 | #2 \rangle}}
\newcommand{\outerF}[2]{\ket{#1} \bra{#2}}

\begin{document}

\noindent\Large\textbf{Problem Set 3} \\
\normalsize
Alice McKean \\
\today \\

\problem{1}

\lemma{1.1}{Any operator $A$ can be written as $A = B + iC$ for some hermitian
  $B$ and $C$
} \\
Proof: We can find $B$ and $C$ with the following expression
\begin{align*}
  A = \frac{A + A^{\dagger}}{2} + \frac{-A^{\dagger} + A}{2}
    = \frac{A + A^{\dagger}}{2} + i\left( \frac{iA^{\dagger} - iA}{2} \right)
\end{align*}
The following equalities show that the matrices are hermitian:
\begin{align*}
  \left( \frac{A + A^{\dagger}}{2} \right)^{\dagger}
  &= \frac{\left( A + A^{\dagger} \right)^{\dagger}}{2}
  = \frac{A^{\dagger} + A}{2}
  = \frac{A + A^{\dagger}}{2} \\
  \left( \frac{iA^{\dagger} - iA}{2} \right)^{\dagger}
  &= \frac{\left( iA^{\dagger} - iA \right)^{\dagger}}{2}
  = \frac{-iA + iA^{\dagger}}{2}
  = \frac{iA^{\dagger} - iA}{2}
\end{align*}

\lemma{1.2}{For any hermitian operator $A$ the expression
  $\bra{v}A\ket{v}$ is real for any $\ket{v}$
} \\
Proof: Every hermitian operator $A$ is normal so it has a spectral decomposition
with real eigenvalues
\begin{align*}
  \bra{v}A\ket{v} = \bra{v}\left( \sum_{a} \lambda_{a} \outerF{a}{a} \right)\ket{v}
  = \sum_{a} \lambda_{a} \innerF{v}{a} \innerF{a}{v}
  = \sum_{a} \lambda_{a} \innerF{v}{a} \innerF{v}{a}^*
\end{align*}

Full proof: Assume $A$ is a positive operator
\begin{align*}
 0 &\leq \bra{v}A\ket{v} \\
   &= \bra{v}B + iC\ket{v} \\
   &= \bra{v}B\ket{v} + i\bra{v}C\ket{v}
\end{align*}
So by Lemma 1.2 $\bra{v}B\ket{v}$ and $\bra{v}C\ket{v}$ are real numbers which
means that $\bra{v}C\ket{v} = 0$.
\begin{align*}
  ( v , Av ) &= ( v, (B + iC)v ) \\
             &= ( v, Bv ) + i( v , Cv ) \\
             &= ( v, Bv ) \\
             &= ( v, Bv ) - i( v, Cv ) \\
             &= ( Bv, v ) - i( Cv, v ) \\
             &= ( Bv, v ) + ( iCv, v ) \\
             &= ( (B + iC)v, v ) \\
             &= ( Av, v )
\end{align*}

\problem{2}

\lemma{2.1}{For any vector $\ket{v} \in V$ if
  $\forall \ket{w} \in V. \, \innerF{w}{v} = 0$ then $\ket{v} = 0$
} \\
Proof: Let $\ket{w} = \ket{v}$ with $\innerF{v}{v} = 0$ so by axiom 3 of the
inner product $\ket{v} = 0$. \\

\lemma{2.2}{For any linear operator $A$ if
  $\forall \ket{w}, \ket{v} \in V. \, \bra{w}A\ket{v} = 0$
  then $A = 0$
} \\
Proof: Lemma 2.1 shows that $\forall \ket{v}. \, A\ket{v} = 0 = 0\ket{v}$ apply
function extensionality to get $A = 0$. \\

\lemma{2.3}{For any linear operator $A$ if
  $\forall \ket{v} \in V. \, \bra{v}A\ket{v} = 0$
  then $A = 0$
} \\
Proof: Let $\ket{p}, \ket{q}$ be arbitrary vectors in $V$:
\vspace{-0.5em}
\begin{align*}
  0 =& 0 + i * 0 \\
    =& \bra{p + q}A\ket{p + q} + i * \bra{p + iq}A\ket{p + iq} \\
    =& \bra{p}A\ket{p} + \bra{p}A\ket{q} + \bra{q}A\ket{p} + \bra{q}A\ket{q} + i * \bra{p + iq}A\ket{p + iq} \\
    =& \bra{p}A\ket{q} + \bra{q}A\ket{p} + i * \bra{p + iq}A\ket{p + iq} \\
    =& \bra{p}A\ket{q} + \bra{q}A\ket{p} +
       i * \left( \, \bra{p}A\ket{p} + \bra{p}A\ket{iq} + \bra{iq}A\ket{p} + \bra{iq}A\ket{iq} \, \right) \\
    =& \bra{p}A\ket{q} + \bra{q}A\ket{p} + i * \left( \, \bra{p}A\ket{iq} + \bra{iq}A\ket{p} \, \right) \\
    =& \bra{p}A\ket{q} + \bra{q}A\ket{p} + i * \left( \, i * \bra{p}A\ket{q} - i * \bra{q}A\ket{p} \, \right) \\
    =& \bra{p}A\ket{q} + \bra{q}A\ket{p} - \bra{p}A\ket{q} + \bra{q}A\ket{p} \\
    =& 2 * \bra{q}A\ket{p}
\end{align*}
The equality shown above proves that
$\forall \ket{p}, \ket{q}. \, \bra{p}A\ket{q} = 0$
so by Lemma 2.2 $A = 0$. \\
Note that this lemma is not true for real inner product
spaces as shown by the following counter example:
\begin{align*}
  \rowarrowsep=-2pt
  \begin{gmatrix}[p]
    v_1 & v_2
  \end{gmatrix}
  \begin{gmatrix}[p]
    0 & 1 \\
    -1 & 0
  \end{gmatrix}
  \begin{gmatrix}[p]
    v_1 \\
    v_2
  \end{gmatrix}
  =
  \begin{gmatrix}[p]
    v_1 & v_2
  \end{gmatrix}
  \begin{gmatrix}[p]
    v_2 \\
    - v_1
  \end{gmatrix}
  = v_1 * v_2 - v_1 * v_2 = 0
\end{align*}
\lemma{2.4}{For any positive operator $M$ note that
  $\forall \ket{v}. \, \bra{v}M^{\dagger}\ket{v} = \bra{v}M\ket{v}$
} \\
Proof: The last line is only true because $M$ is a positive operator
\vspace{-0.5em}
\begin{align*}
  \bra{v}M^{\dagger}\ket{v} =& { \left( { \bra{v}M^{\dagger}\ket{v} }^{\dagger} \right) }^{\dagger} \\
  =& { \bra{v}{ \left( M^{\dagger} \right) }^{\dagger}\ket{v} }^{\dagger} \\
  =& { \bra{v}M\ket{v} }^{\dagger} \\
  =& { \bra{v}M\ket{v} }^* \\
  =& \bra{v}M\ket{v}
\end{align*}
Full proof: Assume $M$ is a positive operator and let $A = M - M^{\dagger}$ note
that
$\bra{v}M - M^{\dagger}\ket{v}
  = \bra{v}M\ket{v} - \bra{v}M^{\dagger}\ket{v}
  = \bra{v}M\ket{v} - \bra{v}M\ket{v} = 0$
so by Lemma 2.3 the equalities $A = 0$ and $M = M^{\dagger}$ hold. \qed \\

\problem{4.17}

The gate is $(I \otimes H)C(Z)(I \otimes H)$ with the following diagram: \\
$
\Qcircuit @C=1em @R=1.5em {
    & \qw      & \ctrl{1} & \qw      & \qw \\
    & \gate{H} & \gate{Z} & \gate{H} & \qw
}$ \\

When the top qubit is $\ket{0}$ then the gate is the same as $C(X)$ trivially. \\
When the top qubit is $\ket{1}$ then the gate is the same as $C(X)$ because
$HZH = X$.

\end{document}

