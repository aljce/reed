\documentclass[fleqn]{article}

\usepackage[a4paper]{geometry}
\usepackage{amsmath}
\usepackage{amssymb}
\usepackage{gauss}
\usepackage[inline]{enumitem}

\setlength{\parindent}{0pt}
\setlength{\mathindent}{0pt}

% Allow for Augmented Matricies
\usepackage{etoolbox}
\makeatletter
\patchcmd\g@matrix
 {\vbox\bgroup}
 {\vbox\bgroup\normalbaselines}% restore the standard baselineskip
 {}{}
\makeatother

\newcommand{\BAR}{%
  \hspace{-\arraycolsep}%
  \strut\vrule % the `\vrule` is as high and deep as a strut
  \hspace{-\arraycolsep}%
}

\newcommand{\squig}[0]{\ensuremath{\rightsquigarrow}}

\newcommand{\problem}[1]{\large\textbf{Problem #1}\normalsize}

\newcommand{\evidence}[1]{\ensuremath{(\hspace{0.2em} \text{#1} \hspace{0.2em})}}
\newcommand{\relation}[1]{\ensuremath{\hspace{0.2em} {{} #1 {}} \hspace{0.2em}}}
\newcommand{\equal}{\relation{=}}
\newcommand{\qed}{\hfill\ensuremath{\square}}

\newcommand{\idF}[1]{\ensuremath{\text{id}(#1)}}
\newcommand{\coordsF}[2]{\ensuremath{[ \: #1 \: ]_{\mathcal{#2}}}}
\newcommand{\rankF}[1]{\ensuremath{\text{rank}(#1)}}
\newcommand{\nullityF}[1]{\ensuremath{\text{nullity}(#1)}}
\newcommand{\matrixRep}[3]{\ensuremath{{\left [ \: #1 \: \right ]}_{\mathcal{#2}}^{\mathcal{#3}}}}
\newcommand{\signF}[1]{\ensuremath{\text{sign}(#1)}}
\usepackage{listofitems}
% Cycle Notation
\newcommand\cycleF[2][\:]{
  \readlist\thecycle{#2}
  #1\foreachitem\i\in\thecycle{\ifnum\icnt=1\else#1\fi\i}#1
}
\newcommand{\normF}[1]{\left\lVert#1\right\rVert}

\newcommand{\bra}[1]{\ensuremath{\langle #1 |}}
\newcommand{\ket}[1]{\ensuremath{| #1 \rangle}}
\newcommand{\innerF}[2]{\ensuremath{\langle #1 | #2 \rangle}}
\newcommand{\outerF}[2]{\ket{#1} \bra{#2}}

\begin{document}

\noindent\Large\textbf{Problem Set 2} \\
\normalsize
Alice McKean \\
\today \\

\problem{2.22} \\
Assume $\lambda_1$ and $\lambda_2$ are eigenvalues for eigenvectors $v$ and $w$
with $\lambda_1 \neq \lambda_2$: \\
$\lambda_1\innerF{v}{w} = \innerF{Av}{w} = \innerF{v}{A^{\dagger}w} =
\innerF{v}{Aw} = \lambda_2\innerF{v}{w}$ so
$(\lambda_1 - \lambda_2)\innerF{v}{w} = 0$ and $\innerF{v}{w} = 0$. \\

\problem{2.23} \\
Every projector is normal so it has a spectral decomposition:
\begin{align*}
  \sum_{a} \lambda_{a} \outerF{a}{a} =& \left( \sum_{a} \lambda_{a} \outerF{a}{a} \right)
                                       \left( \sum_{b} \lambda_{b} \outerF{b}{b} \right) \\
  =& \sum_{a} \sum_{b} \lambda_{a} \lambda_{b} \ket{a} \innerF{a}{b} \bra{b} \\
  =& \sum_{a} \lambda_{a}^2 \innerF{a}{a}
\end{align*}
So for every eigenvalue
$\lambda_{a}^2 - \lambda_{a} = 0 = \lambda_{a} (\lambda_{a} - 1)$
and $\lambda_{a} = 0 , 1$. \\

\problem{2.25} \\
For any operator $A$:
\vspace{-0.7em}
\begin{align*}
  ( \ket{v} , A^{\dagger} A \ket{v} ) = ( A\ket{v} , A\ket{v} ) \geq 0
\end{align*}

\problem{2.28}
\begin{align*}
  (A \otimes B)^* =
  \rowarrowsep=-2pt
  { \begin{gmatrix}[p]
      A_{11} B & A_{12} B & \dots & A_{1n} B \\
      A_{21} B & A_{22} B & \dots & A_{2n} B \\
      \vdots  & \vdots   & \ddots & \vdots \\
      A_{m1} B & A_{m2} B & \dots & A_{mn} B
    \end{gmatrix}
  }^*
  =
  \begin{gmatrix}[p]
    A_{11}^* B^* & A_{12}^* B^* & \dots & A_{1n}^* B^* \\
    A_{21}^* B^* & A_{22}^* B^* & \dots & A_{2n}^* B^* \\
    \vdots      & \vdots      & \ddots & \vdots \\
    A_{m1}^* B^* & A_{m2}^* B^* & \dots & A_{mn}^* B^*
  \end{gmatrix}
  = A^* \otimes B^*
\end{align*}
\begin{align*}
  (A \otimes B)^T =
  \rowarrowsep=-2pt
  { \begin{gmatrix}[p]
      A_{11} B & A_{12} B & \dots & A_{1n} B \\
      A_{21} B & A_{22} B & \dots & A_{2n} B \\
      \vdots  & \vdots   & \ddots & \vdots \\
      A_{m1} B & A_{m2} B & \dots & A_{mn} B
    \end{gmatrix}
  }^T
  =
  \begin{gmatrix}[p]
    A_{11} B^T & A_{21} B^T & \dots & A_{n1} B^T \\
    A_{12} B^T & A_{22} B^T & \dots & A_{n2} B^T \\
    \vdots    & \vdots     & \ddots & \vdots \\
    A_{1m} B^T & A_{2m} B^T & \dots & A_{nm} B^T
  \end{gmatrix}
  = A^T \otimes B^T
\end{align*}



The analogous proof for the adjoint is just a simple composition of the above
proofs for the conjugate and the transpose. \\

\problem{2.29} \\
For any two unitary operators $A$ and $B$:
\vspace{-0.5em}
\begin{align*}
  (A \otimes B)^{\dagger} (A \otimes B) =& (A^{\dagger} \otimes B^{\dagger}) (A \otimes B) \\
                                       =& A^{\dagger}A \otimes B^{\dagger} B \\
                                       =& I \otimes I = I
\end{align*}

\problem{2.34} \\
First we diagonalize the matrix:
\begin{align*}
  \rowarrowsep=-2pt
  \begin{gmatrix}[p]
    4 & 3 \\
    3 & 4
  \end{gmatrix}
  =
  \frac{1}{\sqrt{2}}
  \begin{gmatrix}[p]
    1 & 1 \\
    1 & -1
  \end{gmatrix}
  \begin{gmatrix}[p]
    7 & 0 \\
    0 & 1
  \end{gmatrix}
  \frac{1}{\sqrt{2}}
  \begin{gmatrix}[p]
    1 & 1 \\
    1 & -1
  \end{gmatrix}
\end{align*}
This allows us to compute the square root and logarithm of the matrix:
\begin{align*}
  \rowarrowsep=-2pt
  \sqrt{
    \begin{gmatrix}[p]
      4 & 3 \\
      3 & 4
    \end{gmatrix}
  }
  =&
  \rowarrowsep=-2pt
  \frac{1}{\sqrt{2}}
  \begin{gmatrix}[p]
    1 & 1 \\
    1 & -1
  \end{gmatrix}
  \begin{gmatrix}[p]
    \sqrt{7} & 0 \\
    0 & \sqrt{1}
  \end{gmatrix}
  \frac{1}{\sqrt{2}}
  \begin{gmatrix}[p]
    1 & 1 \\
    1 & -1
  \end{gmatrix}
  =
  \frac{1}{2}
  \begin{gmatrix}[p]
    1 + \sqrt{7} & \sqrt{7} - 1 \\
    \sqrt{7} - 1 & 1 + \sqrt{7}
  \end{gmatrix} \\
  \rowarrowsep=-2pt
  \ln{
  \begin{gmatrix}[p]
    4 & 3 \\
    3 & 4
  \end{gmatrix}
  }
  =&
  \rowarrowsep=-2pt
  \frac{1}{\sqrt{2}}
  \begin{gmatrix}[p]
    1 & 1 \\
    1 & -1
  \end{gmatrix}
  \begin{gmatrix}[p]
    \ln{7} & 0 \\
    0 & \ln{1}
  \end{gmatrix}
  \frac{1}{\sqrt{2}}
  \begin{gmatrix}[p]
    1 & 1 \\
    1 & -1
  \end{gmatrix}
  =
  \frac{1}{2}
  \begin{gmatrix}[p]
    \ln{7} & \ln{7} \\
    \ln{7} & \ln{7}
  \end{gmatrix}
\end{align*}

\problem{2.37}
\begin{align*}
  \displaystyle
  \text{tr}(AB)
  & \equal \sum_{i=1}^{n} (AB)_{ii} \\
  {} & \equal \sum_{i=1}^{n} \sum_{k=1}^{n} A_{ik}B_{ki} \\
  {} & \equal \sum_{i=1}^{n} \sum_{k=1}^{n} B_{ki}A_{ik} \\
  {} & \equal \sum_{k=1}^{n} \sum_{i=1}^{n} B_{ki}A_{ik} \\
  {} & \equal \sum_{i=1}^{n} \sum_{k=1}^{n} B_{ik}A_{ki} \\
  {} & \equal \sum_{i=1}^{n} (BA)_{ii} \\
  {} & \equal \text{tr}(BA)
\end{align*}

\end{document}

