\documentclass[fleqn]{article}

\usepackage[margin=2cm]{geometry}
\usepackage{amsmath}
\usepackage{amssymb}
\usepackage{gauss}
\usepackage[inline]{enumitem}
\usepackage{qcircuit}

\setlength{\parindent}{0pt}
\setlength{\mathindent}{0pt}

% Allow for Augmented Matricies
\usepackage{etoolbox}
\makeatletter
\patchcmd\g@matrix
 {\vbox\bgroup}
 {\vbox\bgroup\normalbaselines}% restore the standard baselineskip
 {}{}
\makeatother

\newcommand{\BAR}{%
  \hspace{-\arraycolsep}%
  \strut\vrule % the `\vrule` is as high and deep as a strut
  \hspace{-\arraycolsep}%
}

\newcommand{\squig}[0]{\ensuremath{\rightsquigarrow}}

\newcommand{\problem}[1]{{\large\textbf{Problem #1}}}
\newcommand{\lemma}[2]{\textbf{Lemma #1} #2}

\newcommand{\evidence}[1]{\ensuremath{(\hspace{0.2em} \text{#1} \hspace{0.2em})}}
\newcommand{\relation}[1]{\ensuremath{\hspace{0.2em} {{} #1 {}} \hspace{0.2em}}}
\newcommand{\equal}{\relation{=}}
\newcommand{\qed}{\hfill\ensuremath{\square}}

\newcommand{\idF}[1]{\ensuremath{\text{id}(#1)}}
\newcommand{\coordsF}[2]{\ensuremath{[ \: #1 \: ]_{\mathcal{#2}}}}
\newcommand{\rankF}[1]{\ensuremath{\text{rank}(#1)}}
\newcommand{\nullityF}[1]{\ensuremath{\text{nullity}(#1)}}
\newcommand{\matrixRep}[3]{\ensuremath{{\left [ \: #1 \: \right ]}_{\mathcal{#2}}^{\mathcal{#3}}}}
\newcommand{\signF}[1]{\ensuremath{\text{sign}(#1)}}
\usepackage{listofitems}
% Cycle Notation
\newcommand\cycleF[2][\:]{
  \readlist\thecycle{#2}
  #1\foreachitem\i\in\thecycle{\ifnum\icnt=1\else#1\fi\i}#1
}
\newcommand{\normF}[1]{\left\lVert#1\right\rVert}

\newcommand{\bra}[1]{\ensuremath{\langle #1 |}}
\newcommand{\ket}[1]{\ensuremath{| #1 \rangle}}
\newcommand{\innerF}[2]{\ensuremath{\langle #1 | #2 \rangle}}
\newcommand{\outerF}[2]{\ket{#1} \bra{#2}}

\begin{document}

\noindent\Large\textbf{Problem Set 3} \\
\normalsize
Alice McKean \\
\today \\

\problem{2.48} \\
Any positive operator $P$ is already in polar form because $I$ is unitary and
$P = IP$.

By similar reasoning any unitary operator $U$ is already in polar form because
$I$ is positive and $U = UI$. 

Every hermitian matrix has a polar decomposition of the form:
$A = U\sqrt{A^{\dagger}A} = U\sqrt{A^2} = U \normF{A}$ \\

\problem{2.49} \\
Note that for normal operators
$J = \sqrt{A^{\dagger} A} = \sqrt{A A^{\dagger}} = K$
This fact implies $A = UJ = KU = JU$ so $J$ and $U$ are
simultaneously diagonalizable.
\begin{align*}
  A = \sum_i u_i \outerF{i}{i} \sum_j \normF{a_j} \outerF{j}{j} = \sum_i u_i \normF{a_i} \outerF{i}{i}
\end{align*}

\problem{2.50}
\begin{align*}
  \rowarrowsep=-2pt
  \begin{gmatrix}[p]
    1 & 0 \\
    1 & 1
  \end{gmatrix}
  =
  \frac{1}{\sqrt{5}}
  \begin{gmatrix}[p]
    3 & -1 \\
    2 & 1
  \end{gmatrix}
  \frac{1}{\sqrt{5}}
  \begin{gmatrix}[p]
    2 & 1 \\
    1 & 3
  \end{gmatrix}
  = U K =
  \frac{1}{\sqrt{5}}
  \begin{gmatrix}[p]
    1 & 1 \\
    -1 & 4
  \end{gmatrix}
  \frac{1}{\sqrt{5}}
  \begin{gmatrix}[p]
    3 & -1 \\
    2 & 1
  \end{gmatrix}
  = L U
\end{align*}

\problem{2.59} \\
The average in state $\ket{0}$:
\begin{align*}
  \bra{0}X\ket{0} = \bra{0} \left( \, \outerF{0}{1} + \outerF{1}{0} \,  \right) \ket{0}
                  = \innerF{0}{0} \innerF{1}{0} + \innerF{0}{1} \innerF{0}{0}
                  = 0
\end{align*}
The standard deviation in state $\ket{0}$:
\begin{align*}
  \sqrt{\bra{0} X^2 \ket{0} - { \bra{0} X \ket{0} }^2 } = \sqrt{\innerF{0}{0} - 0} = 1
\end{align*}

\problem{2.66} \\
Note that $X_1Z_2 = X \otimes Z$ so the average in the bell state:
\begin{align*}
  \left(\frac{\ket{00} + \ket{11}}{\sqrt{2}}\right)^{\dagger}(X \otimes Z)\left(\frac{\ket{00} + \ket{11}}{\sqrt{2}}\right)
  = \left(\frac{\ket{00} + \ket{11}}{\sqrt{2}}\right)^{\dagger} \left( \frac{-\ket{01} + \ket{10}}{\sqrt{2}} \right)
  = 0
\end{align*}

\problem{2.69} \\
Note that $\{ x + y = 0, \, x - y = 0 \}$ implies $\{ x = 0, \, y = 0 \}$. \\
The bell basis is linearly independent:
\begin{align*}
  0 &= a \left( \frac{\ket{00} + \ket{11}}{\sqrt{2}} \right) +
       b \left( \frac{\ket{00} - \ket{11}}{\sqrt{2}} \right) +
       c \left( \frac{\ket{10} + \ket{01}}{\sqrt{2}} \right) +
       d \left( \frac{\ket{01} - \ket{10}}{\sqrt{2}} \right) \\
    &= \frac{a\ket{00} + a\ket{11} + b\ket{00} - b\ket{11} +
             c\ket{10} + c\ket{01} + d\ket{01} - d\ket{10}}{\sqrt{2}} \\
    &= \frac{(a + b) \ket{00} + (a - b) \ket{11} + (c + d) \ket{01} + (c - d) \ket{10}}{\sqrt{2}}
\end{align*}
So $\{ a + b = 0, \, a - b = 0 \}$, $\{ c + d = 0, \, c - d = 0 \}$,
and $a = b = c = d = 0$. \\

The equation above also proves that the bell basis spans $C^{\otimes2}$:  \\
\vspace{-1em}
\begin{align*}
  a\ket{00} + b\ket{01} + c\ket{10} + d\ket{11}
  &= \left( \frac{a + d}{2} \right) \left( \frac{\ket{00} + \ket{11}}{\sqrt{2}} \right) +
     \left( \frac{a - d}{2} \right) \left( \frac{\ket{00} - \ket{11}}{\sqrt{2}} \right) \\
  &+ \, \left( \frac{b + c}{2} \right) \left( \frac{\ket{10} + \ket{01}}{\sqrt{2}} \right) +
     \left( \frac{b - c}{2} \right) \left( \frac{\ket{01} - \ket{10}}{\sqrt{2}} \right)
\end{align*}
The bell states are all clearly normalized and orthogonal so they form an
orthonormal basis for $C^{\otimes2}$. \\

\problem{2.70} \\
Assume $x$ and $y$ are real and that $a$, $b$, $c$, $d$ form an orthonormal basis:
\begin{align*}
  \left(x\ket{ab} + y\ket{cd}\right)^{\dagger}\left( E \otimes I \right)\left(x\ket{ab} + y\ket{cd}\right)
  &= \left(x\ket{a} \otimes x\ket{b} + y\ket{c} \otimes y\ket{d}\right)^{\dagger}
     \left( E \otimes I \right)
     \left(x\ket{a} \otimes x\ket{b} + y\ket{c} \otimes y\ket{d}\right) \\
  &= \left(x\bra{a} \otimes x\bra{b} + y\bra{c} \otimes y\bra{d}\right)
     \left( E \otimes I \right)
     \left(x\ket{a} \otimes x\ket{b} + y\ket{c} \otimes y\ket{d}\right) \\
  &= \left(x\bra{a} \otimes x\bra{b} + y\bra{c} \otimes y\bra{d}\right)
     \left(xE\ket{a} \otimes x\ket{b} + yE\ket{c} \otimes y\ket{d}\right) \\
  &= x^2\bra{a}E\ket{a} \otimes x^2\innerF{b}{b} + xy\bra{a}E\ket{c} \otimes xy\innerF{b}{d} \\
  &+ xy\bra{c}E\ket{a} \otimes xy\innerF{d}{b} + y^2\bra{c}E\ket{c} \otimes y^2\innerF{d}{d} \\
  &= x^4\bra{a}E\ket{a}\innerF{b}{b} + \left( xy \right)^2\bra{a}E\ket{c}\innerF{b}{d}
   + \left( xy \right)^2\bra{c}E\ket{a}\innerF{d}{b} + y^4\bra{c}E\ket{c}\innerF{d}{d} \\
  &= x^4\bra{a}E\ket{a} + y^4\bra{c}E\ket{c}
\end{align*}
The above equality shows that $E \otimes I$ measured on any bell basis returns
the same value because $x = y = \pm 1 / \sqrt{2}$ and if $a = 1$ then $c = 0$ or
if $a = 0$ then $c = 1$ for the bell states. \\
This result also shows that Eve can't learn any information from Alice's qubit
because she cant perform any measurement that extracts information. Any
measurement returns the same value.
\end{document}

