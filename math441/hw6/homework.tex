\documentclass[fleqn]{article}

\usepackage[margin=2cm]{geometry}
\usepackage{amsmath}
\usepackage{amssymb}
\usepackage{gauss}
\usepackage{enumitem}
\usepackage{qcircuit}

\setlength{\parindent}{0pt}
\setlength{\mathindent}{0pt}

% Allow for Augmented Matricies
\usepackage{etoolbox}
\makeatletter
\patchcmd\g@matrix
 {\vbox\bgroup}
 {\vbox\bgroup\normalbaselines}% restore the standard baselineskip
 {}{}
\makeatother

\newcommand{\BAR}{%
  \hspace{-\arraycolsep}%
  \strut\vrule % the `\vrule` is as high and deep as a strut
  \hspace{-\arraycolsep}%
}

\newcommand{\squig}[0]{\ensuremath{\rightsquigarrow}}

\newcommand{\problem}[1]{{\large\textbf{Problem #1}}}
\newcommand{\lemma}[2]{\textbf{Lemma #1} #2}

\newcommand{\evidence}[1]{\ensuremath{(\hspace{0.2em} \text{#1} \hspace{0.2em})}}
\newcommand{\relation}[1]{\ensuremath{\hspace{0.2em} {{} #1 {}} \hspace{0.2em}}}
\newcommand{\equal}{\relation{=}}
\newcommand{\qed}{\hfill\ensuremath{\square}}

\newcommand{\idF}[1]{\ensuremath{\text{id}(#1)}}
\newcommand{\trF}[1]{\ensuremath{\text{tr}\left(#1\right)}}
\newcommand{\coordsF}[2]{\ensuremath{[ \: #1 \: ]_{\mathcal{#2}}}}
\newcommand{\rankF}[1]{\ensuremath{\text{rank}(#1)}}
\newcommand{\nullityF}[1]{\ensuremath{\text{nullity}(#1)}}
\newcommand{\matrixRep}[3]{\ensuremath{{\left [ \: #1 \: \right ]}_{\mathcal{#2}}^{\mathcal{#3}}}}
\newcommand{\signF}[1]{\ensuremath{\text{sign}(#1)}}
\usepackage{listofitems}
% Cycle Notation
\newcommand\cycleF[2][\:]{
  \readlist\thecycle{#2}
  #1\foreachitem\i\in\thecycle{\ifnum\icnt=1\else#1\fi\i}#1
}
\newcommand{\normF}[1]{\left\lVert#1\right\rVert}

\newcommand{\bra}[1]{\ensuremath{\langle #1 |}}
\newcommand{\ket}[1]{\ensuremath{| #1 \rangle}}
\newcommand{\innerF}[2]{\ensuremath{\langle #1 | #2 \rangle}}
\newcommand{\outerF}[2]{\ket{#1} \bra{#2}}
\newcommand{\modF}[2]{\ensuremath{#1 \, \text{mod} \, #2}}
\newcommand{\ordF}[2]{\ensuremath{\text{ord}_{#1}\left( #2 \right)}}

\begin{document}

\noindent\Large\textbf{Problem Set 6} \\
\normalsize
Alice McKean \\
\today \\

\problem{5.4}
\vspace{-0.5em}
\begin{align*}
  \Qcircuit @C=1em @R=.7em {
  & \qw                      & \ctrl{1} & \qw                      & \ctrl{1} & \gate{R_{k + 1}} & \qw \\
  & \gate{R_{k + 1}^{\dagger}} & \targ    & \gate{R_{k + 1}^{\dagger}} & \targ    & \gate{R_k}      & \qw
                                                                                                      }
\end{align*}
The key idea in this construction is that
$R_{k}R_{k + 1}^{\dagger}R_{k + 1}^{\dagger} = I$. This is because the first two
gates do two half rotations then the last gate reverses those rotations. Notice
that $R_k\ket{0} = \ket{0}$ and $R_k\ket{1} = e^{2 \pi i / 2^k}\ket{1}$ the
eigenvalues of $R_k$. The first equality shows the circuit is correct for the
basis elements $\ket{00}, \ket{01}$.
\begin{align*}
  R_{k} X R_{k + 1}^{\dagger} X R_{k + 1}^{\dagger}\ket{0}
  &=  R_{k} X R_{k + 1}^{\dagger} X\ket{0} \\
  &= R_{k} X R_{k + 1}^{\dagger}\ket{1} \\
  &= e^{-2 \pi i / 2^{k + 1}} R_{k} X \ket{1} \\
  &= e^{-2 \pi i / 2^{k + 1}} R_{k} \ket{0} \\
  &= e^{-2 \pi i / 2^{k + 1}} \ket{0} \\
  R_{k} X R_{k + 1}^{\dagger} X R_{k + 1}^{\dagger}\ket{1}
  &= e^{-2 \pi i / 2^{k + 1}} R_{k} X R_{k + 1}^{\dagger} X\ket{1} \\
  &= e^{-2 \pi i / 2^{k + 1}} R_{k} X R_{k + 1}^{\dagger}\ket{0} \\
  &= e^{-2 \pi i / 2^{k + 1}} R_{k} X \ket{0} \\
  &= e^{-2 \pi i / 2^{k + 1}} R_{k} \ket{1} \\
  &= e^{-2 \pi i / 2^{k + 1}} e^{2 \pi i / 2^k} \ket{1} \\
  &= e^{2 \pi i / 2^{k + 1}} \ket{1}
\end{align*}
Using the equalities shown above we can calculate the action of the gate on the
basis element $\ket{10}$:
\begin{align*}
  G(\ket{10}) = e^{2 \pi i / 2^{k + 1}}\ket{1} \otimes e^{-2 \pi i / 2^{k + 1}}\ket{0} = \ket{10}
\end{align*}
Using the equalities shown above we can calculate the action of the gate on the
basis element $\ket{11}$:
\begin{align*}
  G(\ket{11}) = e^{2 \pi i / 2^{k + 1}}\ket{1} \otimes e^{2 \pi i / 2^{k + 1}} \ket{1}
  = e^{2 \pi i / 2^{k}} \ket{11}
\end{align*}
The gate defined above has the same action on each basis element as
$C(R_{k})$ so the two gates are equal. \\

\problem{5.7} \\
If you write $\ket{j}$ in binary the circuit's correctness becomes quite clear.
\begin{align*}
  U^j = U^{2^{t - 1} j_{t - 1} + 2^{t - 2} j_{t - 2} + \dots + 2^0 j_0}
  = U^{2^{t - 1} j_{t - 1}} U^{2^{t - 2} j_{t - 2}} \dots U^{2^0 j_0}
\end{align*}
For each set bit $t$ into the binary expansion the circuit performs the gate $U$, $2^t$ times. \\

\problem{5.9} \\
Note the following action of $U$ on $\ket{\psi}$ when $\ket{\psi}$ is written in the
eigenvalues of $U$:
\begin{align*}
  \left( \outerF{\psi_{+1}}{\psi_{+1}} - \outerF{\psi_{-1}}{\psi_{-1}} \right) 
  \left( a \ket{\psi_{+1}} + b \ket{\psi_{-1}} \right)
  = a \ket{\psi_{+1}} - b \ket{\psi_{-1}}
\end{align*}
Therefore the operator $U$ doesn't effect the probabilities of collapse into
either of the eigenvectors.

Construct a state
$\ket{0}\ket{\psi} = \ket{0}\left( a \ket{\psi_{+1}} + b \ket{\psi_{-1}} \right)$
then perform a quantum phase estimation on the state which outputs
$a \ket{0} \ket{\psi_{+1}} + b \ket{1/2} \ket{\psi_{-1}}$.
Then measure the first register. If you measure
a $0$ the state is in $\ket{\psi_{+1}}$ if you measure a $1/2$ the state is in $\psi_{-1}$.
This gate and the gate found in $4.34$ both only require two qubits. This gate
only requires two qubits because you only need one bit of precision to determine
a $0$ from a $1/2$ in binary.

\problem{5.10} \\
The following calculations show that $6$ is the first positive integer such that
$\modF{5^n}{21} = 1$ therefore $\ordF{5}{21} = 6$.
\begin{align*}
  \modF{5^1}{21} = 5 \,\,\,\,\,
  \modF{5^2}{21} = 4 \,\,\,\,\,
  \modF{5^3}{21} = 20 \,\,\,\,\,
  \modF{5^4}{21} = 16 \,\,\,\,\,
  \modF{5^5}{21} = 17 \,\,\,\,\,
  \modF{5^6}{21} = 1 \,\,\,\,\,
\end{align*}

\problem{5.13} \\
Note that $\delta_{(t - k)0} = 1$ if and only if $t = k$:
\begin{align*}
  \frac{1}{\sqrt{r}} \sum_{s = 0}^{r - 1} e^{2 \pi i s k / r}\ket{u_s}
  &= \frac{1}{\sqrt{r}} \sum_{s = 0}^{r - 1} e^{2 \pi i s k / r}
     \frac{1}{\sqrt{r}} \sum_{t = 0}^{r - 1} e^{-2 \pi i s t / r} \ket{\modF{x^t}{N}} \\
  &= \frac{1}{r} \sum_{s = 0}^{r - 1} \sum_{t = 0}^{r - 1} e^{2 \pi i s (k - t) / r}\ket{\modF{x^t}{N}} \\
  &= \frac{1}{r} \sum_{t = 0}^{r - 1} \ket{\modF{x^t}{N}} \sum_{s = 0}^{r - 1} e^{2 \pi i s (k - t) / r} \\
  &= \frac{1}{r} \sum_{t = 0}^{r - 1} \ket{\modF{x^t}{N}} \sum_{s = 0}^{r - 1} e^{-2 \pi i s (t - k) / r} \\
  &= \frac{1}{r} \sum_{t = 0}^{r - 1} \ket{\modF{x^t}{N}} r \delta_{(t - k)0}\\
  &= \ket{\modF{x^k}{N}}
\end{align*}
\problem{5.14} \\
If we can prove that $U^j\ket{1} = \ket{\modF{x^j}{N}}$ for any $j$ the exercise
is solved. What follows is a proof by induction on $j$. When $j = 0$ the
equality is satisfied $U^0\ket{1} = \ket{1} = \ket{\modF{1}{N}} =
\ket{\modF{x^0}{N}}$. Next assume that $U^j\ket{1} = \ket{\modF{x^j}{N}}$ to
show that the equality is satisfied for $1 + j$:
\begin{align*}
  U^{1 + j}\ket{1} = U U^j \ket{1} = U \ket{\modF{x^j}{N}} = \ket{\modF{x\left( \modF{x^j}{N} \right)}{N}}
  = \ket{\modF{x x^{j}}{N}}= \ket{\modF{x^{1 + j}}{N}}
\end{align*}

\end{document}

