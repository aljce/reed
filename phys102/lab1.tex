\documentclass{article}

\usepackage[total={6in, 8in}]{geometry}
\usepackage[utf8]{inputenc}
\usepackage[english]{babel}

\usepackage{float}
\usepackage{amsmath}
\usepackage{amssymb}
\usepackage{apacite}
\usepackage{siunitx}
\usepackage{graphicx}
\graphicspath{ {images/} }

\newcommand{\graph}[2]{
  \begin{figure}[H]
    \caption{#1}
    \includegraphics[scale=0.3]{#2.png}
  \end{figure}
}
\newcommand{\dataStudio}[0]{\textit{Data Studio} }
\newcommand{\workshop}[0]{\textit{Science Workshop}}
\newcommand{\pError}[0]{\ensuremath{\% \text{Error}}}
\newcommand{\tableFrac}[2]{\ensuremath{#1/#2}}

\title{Fourier Transforms}
\author{Kyle McKean \\ \normalsize Mattie O'Kelley, Grace Martin, Elena McKnight}
\date{\normalsize FLO1 Box 22}

\begin{document}
\maketitle
\begin{abstract}
  The focus of this experiment was to investigate the Fourier transforms of
  different wave sources. We observed the Fourier transform of tuning forks,
  boomwackers and ukuleles and compared the resulting frequencies to the expected
  frequencies. We found that the Fourier transform accurately predicted the
  expected frequencies with $1.95\%$, $0.340\%$, and $3.40\%$ respective error.
  Then we generated sine and square waves with a PASCO power amplifier and
  recorded their Fourier Transforms. The sine wave frequency had $0.0\%$ error and
  the square wave frequency had $1.6\%$ error.
\end{abstract}
\section{Introduction}
The Fourier transform is a method of transforming any periodic function into a
(potentially infinite) trigonometric series \cite{wolfram:1}. A discrete Fourier
Transform computes the frequencies of a trigonometric series given a collection
of equally spaced samples. There exists an algorithm for computing the discrete
Fourier transform called the Fast Fourier Transform (FFT) \cite{cooley:1}. The
output of this algorithm is frequencies and amplitudes associated with n sine
waves that when added together approximate the original signal. We used this FFT
algorithm to investigate various wave forms.

\begin{figure}[h]
  \centering
  \caption{Fourier Series Examples}
  \begin{tabular}{|c|c|c|}
    \hline
    Wave-form & Frequency          & Relative Amplitude \\
    \hline
    Sine     & $f$                & 1 \\
    Square   & $(2n - 1) \cdot f$ & $\tableFrac{1}{(2n - 1)}$ \\
    \hline
  \end{tabular}
  \label{fig:fourier-series}
\end{figure}

After applying a Fourier transform to the above wave-forms their respective
frequencies and amplitudes are as shown. In these cases $f$ is the original frequency
of the wave. It should be noted that the Fourier transform of a sine wave is
just the sine wave unchanged.

\section{Experimental Design and Procedure}
\begin{figure}[h!]
  \centering
  \caption{Sketch of Experimental Design}
  \vspace{0.3em}
  \includegraphics[scale=0.5]{sketch.png} 
\end{figure}
First we investigated the Fourier transforms of tuning forks, boomwackers, and
ukuleles. We attached a microphone to a ring stand with a clamp. Then we
connected that microphone to the \workshop. After this we
opened \dataStudio and ensured that the voltage probe was sampling at
$\num{10}\si{kHz}$. Next we opened a FFT window in \dataStudio and set the
sample rate to $\num{20000}\si{Hz}$. Then we decreased the bin size in the FFT
window to its lowest possible value. With the configuration over we began
recording data. First we struck a tuning fork with a rubber mallet. Then we
struck a D boomwacker against the table. Finally we played a D
chord on a ukulele. After each of these steps we saved the FFT graph.

Next we investigated the Fourier transforms of generated waves. First we exited
\dataStudio. Then we plugged the voltage sensor into the \workshop and attached
its probes to the output of a PASCO power amplifier. Next we reopened
\dataStudio and modified the appropriate ports to initialize the voltage sensor and
power amplifier. We then set the voltage sensor sensitivity to $1x$ and set the
power amplifier gain to $1x$. Then we opened a FFT window and initialized it as
we did before. Now that the configuration is completed we generated the waves.
First we generated a sine wave with a $\num{640}\si{Hz}$ frequency. Then we
generated a square wave with a $\num{640}\si{Hz}$ frequency. In both cases we
saved the resulting FFT graph.
\section{Results and Analysis}
\begin{equation*}
  \pError = \left| \frac{O - E}{E} \right| \cdot 100
\end{equation*}
In the following tables percent error is calculated with the above equation. 
\begin{center}
  \begin{tabular}{ |c|c|c|c| }
    \multicolumn{4}{c}{Audio Source Frequency} \\
    \hline
    {} & Tuning Fork & Boomwacker & Ukulele \\
    \hline
    Expected Frequency $(\si{Hz})$ & $256$  & $293.7$ (D) & $293.7$ (D) \\
    Observed Frequency $(\si{Hz})$ & $261$  & $295$       & $304$ \\
    $\pError$                      & $1.95$ & $0.340$     & $3.40$ \\
    \hline
  \end{tabular}
\end{center}
The above observed frequencies were all quite close to the expected. Our tuning
fork was rated to generate a sine wave with a frequency $\num{256}{\si{Hz}}$ so
it makes sense to observe a value close to that. With the boomwacker and ukulele
we ensured we were playing as close to a D as possible. We ensured this by using
a guitar tuning tool. A D chord has a frequency of $\num{293.7}\si{Hz}$. With
this in mind it makes sense that in both instruments the strongest frequency
observed was around $\num{293.7}\si{Hz}$.
\begin{center}
  \begin{tabular}{|c|c|c|c|c|}
    \multicolumn{5}{c}{Square Wave Frequency} \\
    \hline
    {} & wave 1 & wave 2 & wave 3 & wave 4 \\
    \hline
    Expected Frequency $(\si{Hz})$
       & $640$ & $1920$ & $3200$ & $4480$ \\
    Observed Frequency $(\si{Hz})$
       & $630$ & $2000$ & $3200$ & $4500$ \\
    $\pError$
       & $1.6$ & $4.2$  & $0.0$  & $0.4$  \\
    \hline
  \end{tabular}
\end{center}
The expected frequency was calculated using figure \ref{fig:fourier-series}
where $f = \num{640}\si{Hz}$. The observed frequency followed a clear pattern in
frequency spacing and that pattern followed the expected pattern with $1.6\%$
average error.
\begin{center}
  \begin{tabular}{|c|c|c|c|c|}
    \multicolumn{5}{c}{Square Wave Relative Amplitude} \\
    \hline
    {} & wave 1 & wave 2 & wave 3 & wave 4 \\
    \hline
    Expected Relative Amplitude
       & $1$   & $\tableFrac{1}{3}$ & $\tableFrac{1}{5}$ & $\tableFrac{1}{7}$ \\
    Observed Relative Amplitude
       & $1$   & $0.31$ & $0.16$ & $0.09$ \\
    $\pError$
       & $0$   & $7$    & $20$   & $37$  \\
    \hline
  \end{tabular}
\end{center}
The relative amplitudes of the generated square wave deviated significantly from
the expected with a $16\%$ average error.
\section{Conclusions}
The only data-set with a significant percent error was the relative amplitude
data-set. The error increases as the relative amplitude decreases.
This error could be attributed to human error, noise in the signal,
and/or errors in the FFT algorithm. All of these could be causes because all these
problem get worse as the relative amplitude decreases. As the relative
amplitudes decrease it becomes harder for a human to determine which spike in
amplitude is not noise. There are many ways to generate electric noise. One
example is coupled noise in which a radio source induces a current a piece of
wire \cite{motchenbacher:1}. These being said the most likely explanation in
the for the high error is Round off error in the FFT. This is because as the
amplitude gets smaller the round off error compounds \cite{ramos:1}.
\bibliographystyle{apacite}
\bibliography{citation.bib}
\end{document}