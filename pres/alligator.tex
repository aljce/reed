\documentclass[usenames,dvipsnames]{beamer}
\usepackage{amsmath}
\usepackage{mathtools}
\usepackage{graphicx}
\usepackage{enumitem}
\usetheme{metropolis}
\usefonttheme[onlymath]{serif}
\title{Alligator Eggs!}
\date{13 November 2018}
\author{Alice McKean}
\institute{Reed Student Colloquium}

\newcommand{\textbs}[1]{{\sffamily\fontseries{sbc}\selectfont #1}}

\newcommand{\mathbs}[1]{\ensuremath{\text{\textbs{#1}}}}
\renewcommand{\mathtt}[1]{\ensuremath{\texttt{#1}}}

\newcommand{\mrs}[1]{\ensuremath{\mathnormal{#1}}} % Reset font to normal
\newcommand{\mbf}[1]{\ensuremath{\mathbf{#1}}}     % Boldface
\newcommand{\mbs}[1]{\ensuremath{\mathbs{#1}}}     % Bold + sans-serif
\newcommand{\mbb}[1]{\ensuremath{\mathbb{#1}}}     % Blackboard bold
\newcommand{\mtt}[1]{\ensuremath{\mathtt{#1}}}     % Teletype
\newcommand{\mrm}[1]{\ensuremath{\mathrm{#1}}}     % Serif ("roman")
\newcommand{\msf}[1]{\ensuremath{\mathsf{#1}}}     % Sans-serif
\newcommand{\msc}[1]{\ensuremath{\mathsc{#1}}}     % Small-caps
\newcommand{\mcl}[1]{\ensuremath{\mathcal{#1}}}    % Calligraphic
\newcommand{\msr}[1]{\ensuremath{\mathscr{#1}}}    % Script
\newcommand{\mfr}[1]{\ensuremath{\mathfrak{#1}}}   % Fraktur

\newcommand{\pic}[2]{\includegraphics[width=#1\columnwidth]{#2}}
\newcommand{\hungry}[0]{\textbf{hungry}}

\graphicspath{
  {resources/}
}

\begin{document}

\maketitle

\section{Pieces}
\begin{frame}{Hungry Alligators}
  \begin{figure}
    \pic{0.9}{pieces/hungry_alligators.png}
  \end{figure}
  Hungry alligators are \hungry. They'll eat anything that's in front of them! But
  they are also \textbf{responsible} alligators, so they guard their families.
\end{frame}

\begin{frame}{Old Alligators}
  \begin{figure}
    \pic{0.8}{pieces/old_alligators.png}
  \end{figure}
  Old alligators are not \hungry. They've eaten enough.
  All they do is guard their families.
\end{frame} 

\begin{frame}{Eggs}
  \begin{figure}
    \pic{0.8}{pieces/egg.png}
  \end{figure}
  Eggs \textbf{hatch} into new families of alligators and must match with their parents color!
\end{frame} 

\section{Families}
\begin{frame}{Small Family}
  \begin{figure}
    \pic{0.4}{family/small.png}
  \end{figure}
  A green alligator is guarding her green egg.
\end{frame}

\begin{frame}{Big Family}
  \begin{figure}
    \pic{0.4}{family/big.png}
  \end{figure}
  A green alligator and red alligator are guarding a green egg and a red egg.
  Or, you could say that the green alligator is guarding the red alligator,
  and the red alligator is guarding the eggs.
\end{frame}

\begin{frame}{Huge Family}
  \begin{columns}
    \begin{column}{0.4\textwidth}
      \begin{figure}
        \pic{1}{family/huge.png}
      \end{figure}
    \end{column}
    \begin{column}{0.6\textwidth}
      \begin{itemize}
      \item<1|only@1>
        We've got yellow, green, red alligators guarding this family. They are
        guarding three things: a green egg, an old alligator, and a red egg. The old
        alligator is guarding a yellow egg and a green egg.

      \item<2|only@2>
        Notice that eggs only use the colors of the alligators guarding them. You
        can't have a blue egg without there being a blue alligator around to guard it.
      \end{itemize}
    \end{column}
  \end{columns}
\end{frame}

\section{An Example}
\begin{frame}{Example}
  \begin{columns}
    \begin{column}{0.4\textwidth}
      \begin{tikzpicture}
        \foreach \Value in {1,2,3}
          \node<\Value> (img\Value) {\pic{1.2}{example/\Value.png}};
        \foreach \Value in {4,5,6,7,8,9,10}
          \node<\Value> (img\Value) {\pic{1}{example/\Value.png}};

      \end{tikzpicture}
    \end{column}
    \begin{column}{0.6\textwidth}
      \begin{itemize}
      \item<1|only@1>
        That green alligator sure is hungry. And there's a yellow family, right
        in front of her mouth, and it looks tasty!      

      \item<2|only@2>
        The top left alligator eats the first family to the right of it.

      \item<3|only@3>
        The top left alligator eats the first family to the right of it.

      \item<4|only@4>
        Unfortunately, the green alligator's eyes were bigger than her stomach.
        She ate too much!

      \item<5|only@5>
        And so she dies, and goes off to alligator heaven. But, the story
        doesn't end there. Because the green alligator died, the green egg
        starts to hatch...

      \item<6|only@6>
        Amazingly, it hatches into exactly what the green alligator ate.
        It's the miracle of life!

      \item<6|only@6>
        Now, we're down to one family. We have a red alligator guarding a yellow
        alligator and a red egg, and the yellow alligator is guarding her yellow egg.

      \item<7|only@7>
        But that yellow alligator sure is hungry, and there's a tasty red egg in
        front of her. Here we go again...

      \item<8|only@8>
        Poor alligator. Even a single egg is too much for her stomach!

      \item<9|only@9>
        The yellow alligator dies... but once again, the yellow egg starts to hatch...

      \item<10|only@10>
        And it hatches into exactly what the yellow alligator ate!

      \item<10|only@10>
        Now, there's nothing for anyone to eat, so we stop.
      \end{itemize}
    \end{column}
  \end{columns}
\end{frame}
\section{Rules}

\begin{frame}{Eating Rule}
  \begin{itemize}
  \item 
    That was an example of the first rule of the game the \\ \textbf{Eating Rule}.
  \item
    The eating rule says that if there are some families side-by-side...
  \end{itemize}
\end{frame}

\begin{frame}{Eating Rule}
  \begin{tikzpicture}
    \foreach \Value in {1,2,3}
      \node<\Value> (img\Value) {\pic{0.9}{eating/\Value.png}};
  \end{tikzpicture}
  \begin{itemize}
    \item<1|only@1>
      The eating rule says that if there are some families side-by-side...

    \item<2|only@2>
      The top left alligator eats the first family to the right of it.

    \item<3|only@3>
      Then, that alligator dies. But if she was guarding any eggs of the same color,
      each of those eggs hatches into what she ate.
  \end{itemize}
\end{frame}

\begin{frame}{Color Rule}

  \begin{tikzpicture}
    \foreach \Value in {1,2,3,4,5,6,7}
      \node<\Value> (img\Value) {\pic{0.9}{color/\Value.png}};
  \end{tikzpicture}
  \begin{itemize}
    \item<1|only@1>
      Continuing with the example for the eating rule, the orange alligator eats the yellow
      family.

    \item<2|only@2>
      The color rule says that if an alligator is about to eat a family, and
      there's a color that appears in both families, we need to change that
      color in one family to something else.

    \item<3|only@3>
      In the previous slide, green and red appear in both the first and second
      families. So, in the second family, we switch all of the greens to cyan,
      and all of the reds to blue. This is the \textbf{Color Rule}.

    \item<4|only@4>
      Now that they don't share any colors, we can eat!

  \end{itemize}
\end{frame}

\begin{frame}{Old Age Rule}
  \begin{tikzpicture}
    \foreach \Value in {1,2,3,4,5}
      \node<\Value> (img\Value) {\pic{0.9}{old/\Value.png}};
  \end{tikzpicture}
  \begin{itemize}
    \item<1|only@1>
      The top-left alligator here is not \hungry. She's not going to eat
      anything. All she cares about is her family. So, how does she die?
    \item<1|only@1>
      She dies when she's only guarding one thing. Right now, she's guarding
      both a green family and a red family. 
      
    \item<2|only@2>
      They need her to take care of them. But that green alligator is hungry.
      The green alligator eats the red family...
      
    \item<3|only@3>
      Now, the old alligator is only guarding a single family. That family can
      take care of itself. She's no longer needed. So, she grows old and dies.
      
    \item<4|only@4>
      That is the \textbf{Old Age Rule}.
      When an old alligator is just guarding a single thing, it dies.
  \end{itemize}
\end{frame}

\section{Puzzles}

\section{Real Math?}

\end{document}