\documentclass{article}

\title{Math 382: Homework 1}
\author{Kyle McKean}

\usepackage[total={7in, 9in}]{geometry}
\usepackage{algpseudocode}

\usepackage{amsmath}
\usepackage{mathtools}
\DeclarePairedDelimiter{\ceil}{\lceil}{\rceil}

\newenvironment{To}{\ensuremath{\:\: \textbf{to} \:\:}}

\begin{document}

\maketitle

\begin{enumerate}

\item
  \begin{enumerate}
  \item $\{(1, 2), (1, 3), (1,4), (1,5), (2,3), (2,4), (2,5)\}$
  \item The number of inversions represents the amount of swaps in insertion sort.
  \item
    $|I(A)|$ is minimized when the array A is in sorted order as the constraint
    $A[j] > A[i]$ is never satisifed. This means $|I(A)| = 0$. \\
    $|I(A)|$ is maximized when the array A is in reverse sorted order as the constraint
    $A[j] > A[i]$ is always satisifed.
    This means $|I(A)| = \sum_{i = 1}^{n - 1}{n - k} = \frac{(n - 1)n}{2}$
  \item
    \textit{Neighbor swap} sorting algorithms remove inversions. If the array is in sorted order
    it will have to do no swaps (answer c). This means the lower bound for number of swaps is $O(1)$.
  \item
    \begin{algorithmic}[0]
      \Function{MERGE-SORT-HELPER}{$A, B, l, r$}
        \State $inv \gets 0$
        \If{$r - l > 1$}
          \State $m \gets \ceil[\big]{\frac{l + r}{2}}$
          \State $inv \gets inv + \text{MERGE-SORT-HELPER}(A, B, l, m)$
          \State $inv \gets inv + \text{MERGE-SORT-HELPER}(A, B, m, r)$
          \State $inv \gets inv + \text{MERGE}(A, B, l, m, r)$
          \State $\text{COPY-BACK}(A, B)$
        \EndIf
        \State \Return $inv$
      \EndFunction
      \\
      \Function{MERGE}{$A, B, l, m, r$}
        \State $inv, \: i, \: j \gets 0, \: l, \: r$
        \For{$k \gets l \To r - 1$}
          \If{$j \geq r$}
            \State $B[k] \gets A[i]$
            \State $i \gets i + 1$
          \ElsIf{$i > m$}
            \State $B[k] \gets A[j]$
            \State $j \gets j + 1$
          \Else
            \If{$A[i] < A[j]$}
              \State $B[k] \gets A[i]$
              \State $i \gets i + 1$
            \Else
              \State $B[k] \gets A[j]$
              \State $j \gets j + 1$
              \State $inv \gets inv + 1$
            \EndIf
          \EndIf
        \EndFor
        \State \Return $inv$
      \EndFunction
    \end{algorithmic}
  \end{enumerate}

\item
  \begin{enumerate}
    \item
      \begin{algorithmic}[0]
        \Function{SELECTION-SORT}{$A$}
          \State $n \gets \mathsf{len}(A)$
          \For{$i \gets 1 \To n - 1$}
            \State $j, \: key \gets n$
            \State $min \gets A[n]$
            \While{$j > i$}
              \If{$A[j] < min$}
                \State $key \gets j$
                \State $min \gets A[j]$
              \EndIf
              \State $j \gets j + 1$
            \EndWhile
            \State $\textbf{swap} \: A[i] \Leftrightarrow A[key]$
          \EndFor
        \EndFunction
      \end{algorithmic}
    \item
      All elements before the index i in selection sort are in sorted order. This is the loop invariant.
    \item
      At the end of the first n - 1 iterations there is only one possible smallest value the last value.
      This means it does not need to be sorted and that there is no reason to run the nth pass. 
    \item
      The best case running time for selection sort is $\Theta(n^2)$.
      The worst case running time for selection sort is $\Theta(n^2)$.
  \end{enumerate}

\item
  \begin{algorithmic}[0]
      \Function{MERGE-SORT}{$A$}
        \State $n \gets \mathsf{len}(A)$
        \State $B \gets \textbf{array[n]}$
        \For{$i \gets 1 \To \ceil[\big]{\mathsf{lg} \: n}$}
          \For{$j \gets 0 \To n - 2i \:\: \textbf{step} \:\: 2i$}
            \State $\text{MERGE}(A, \: B, \: j, \: j + i, \: j + 2i)$
          \EndFor
          \State $\text{MERGE}(A, \: B, \: n - 2i, n - i, n)$
          \Comment This ensures \text{MERGE-SORT} works on non $2^p$ length arrays
          \State $\text{COPY-BACK}(A, B)$
        \EndFor
      \EndFunction
  \end{algorithmic}
\end{enumerate}

\end{document}
