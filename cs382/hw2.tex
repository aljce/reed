\documentclass{article}

\usepackage[total={7in, 9in}]{geometry}
\usepackage[utf8]{inputenc}
\usepackage[english]{babel}

\usepackage{amsmath}
\usepackage{amssymb}

\usepackage{amsthm}
\newtheorem{lemma}{Lemma}
\theoremstyle{definition}
\newtheorem*{answer}{Answer}

\usepackage{mathtools}
\DeclarePairedDelimiter\ceil{\lceil}{\rceil}
\DeclarePairedDelimiter\floor{\lfloor}{\rfloor}

\newcommand{\evidence}[1]{\ensuremath{\hspace{3em} (\hspace{0.2em} \text{#1} \hspace{0.2em})}}
\newcommand{\asymptotic}[3]{\ensuremath{#2 = #1(#3)}}
\newcommand{\bigO}[2]{\asymptotic{\mathcal{O}}{#1}{#2}}
\newcommand{\littleO}[2]{\asymptotic{o}{#1}{#2}}
\newcommand{\bigTheta}[2]{\asymptotic{\Theta}{#1}{#2}}

\newcommand{\relation}[1]{\ensuremath{\hspace{0.2em} {{} #1 {}} \hspace{0.2em}}}
\newcommand{\equal}{\relation{=}}
\newcommand{\lesseq}{\relation{\le}}
\newcommand{\parens}[1]{\left(#1\right)}
\newcommand{\lglg}[1]{\ensuremath{\lg{\parens{\lg{#1}}}}}
\newcommand{\brackets}[1]{\ensuremath{[ \hspace{0.2em} #1 \hspace{0.2em} ]}}
\newcommand{\quantify}[2]{\ensuremath{\forall #1 \in \mathbb{#2}. \hspace{0.2em}}}
\newcommand{\contra}{\ensuremath{\Rightarrow\!\Leftarrow}}
\newcommand{\maxSet}[1]{\ensuremath{\mathsf{max}\{#1\}}}

\begin{document}
\noindent
\LARGE\textbf{Math 382: Homework 2} \\
\large Kyle McKean

\section*{Problem 1}

\begin{lemma}
  \begin{equation*}
    \phi(n) := R(n) \le 2n
  \end{equation*}
\end{lemma}
\begin{proof}
  \hfill
  \begin{itemize}
  \item
    \textit{Base Case}: $\phi(1)$
    \begin{align*}
      R(1) &\le 2 \times 1 \\
      1    &\le 2
    \end{align*}
  \item
    \textit{Inductive Step}: assume $\phi(k)$ for all $k \le n$ to prove $\phi(n + 1)$
    \begin{alignat*}{4}
      & R(n + 1) & \equal  & R\parens{\floor*{\frac{n + 1}{2}}} + (n + 1) & \evidence{Definition of $R$} \\
      & {}       & \lesseq & 2\floor*{\frac{n + 1}{2}} + n + 1            & \evidence{Inductive Hypothesis} \\
      & {}       & \equal  & n + 1 + n + 1                                & {} \\
      & {}       & \equal  & 2 (n + 1)                                    & \qedhere
    \end{alignat*}
  \end{itemize}
\end{proof}
\begin{answer}[A]
  Lemma 1 tells us $\bigO{R(n)}{n}$ where $m = 1$ and $c = 2$
\end{answer}

\begin{lemma}
  \begin{equation*}
    \phi(n) := Q(n) \le 2\lglg{n}
  \end{equation*}
\end{lemma}
\begin{proof}
  \hfill
  \begin{itemize}
  \item
    \textit{Base Case}: $\phi(4)$
    \begin{align*}
      Q(4) &\le 2\lglg{4} \\
      2 &\le 2
    \end{align*}
  \item
    \textit{Inductive Step}: assume $\phi(k)$ for all $k \le n$ to prove $\phi(n + 1)$
    \begin{alignat*}{4}
      & Q(n + 1) & \equal & Q\left(\floor*{\sqrt{n + 1}}\right) + 1 & \evidence{Definition of $Q$} \\
      & {}       & \lesseq & 2\lglg{\floor*{\sqrt{n + 1}}} + 1       & \evidence{Inductive Hypothesis} \\
      & {}       & \lesseq & 2\lglg{\sqrt{n + 1}} + 1                & \evidence{$\forall x. \hspace{0.2em} \floor*{x} \le x$} \\
      & {}       & \equal  & 2\lg{\parens{\frac{1}{2}\lg{\parens{n + 1}}}} + 1     & {} \\
      & {}       & \equal  & 2\parens{\lg{1} - \lg{2} + \lglg{\parens{n + 1}}} + 1 & {} \\
      & {}       & \equal  & {-2} + 2\lglg{\parens{n + 1}} + 1                     & {} \\
      & {}       & \equal  & 2\lglg{\parens{n + 1}} - 1                            & {} \\
      & {}       & \lesseq & 2\lglg{\parens{n + 1}}                                & \qedhere
    \end{alignat*}
  \end{itemize}
\end{proof}
\begin{answer}[B]
  Lemma 2 tells us $\bigO{Q(n)}{\lglg{n}}$ where $m = 4$ and $c = 2$
\end{answer}

\section*{Problem 2}

\begin{lemma}
  \begin{equation*}
    \quantify{n, i}{N} 4 \le n \wedge i \le n \Rightarrow n \le (n - i + 1)
  \end{equation*} 
\end{lemma} 

\begin{lemma}
  \begin{equation*}
    \quantify{n}{N} 1 \le n \Rightarrow \lg{\parens{n!}} \le n \lg{n}
  \end{equation*} 
\end{lemma}
\noindent
To solve the above proposition we observe the equation expands to pairwise less than or equals.
\begin{equation*}
  \lg{1} + \lg{2} + \dots + \lg{n} \le \lg{n} + \lg{n} + \dots + \lg{n}
\end{equation*} 
\begin{proof}
      
  \begin{alignat*}{4}
    & \lg{n!} & \equal  & \lg{\prod_{i=1}^{n}{i}}  & \evidence{definition of factorial} \\
    & {}      & \equal  & \sum_{i=1}^{n}{\lg{i}}   & \evidence{log product rule} \\
    & {}      & \lesseq & \sum_{i=1}^{n}{\lg{n}}   & \evidence{$i \le n$} \\  
    & {}      & \equal  & n \lg{n}                & \qedhere
  \end{alignat*}
\end{proof}

\begin{lemma}
  \begin{equation*}
    \quantify{n}{N} 4 \le n \Rightarrow \frac{n}{2} \lg{n} \le \lg{\parens{n!}}
  \end{equation*} 
\end{lemma} 
\noindent
To solve the next proposition we rearrange the term right of the relation.
Then we observe two of the terms on the right add to more than one on the left.
The following is an example when $n = 6$.
\begin{equation*}
  \brackets{\lg{6}} + \brackets{\lg{6}} + \brackets{\lg{6}} \le
  \brackets{\lg{1} + \lg{6}} + \brackets{\lg{2} + \lg{5}} + \brackets{\lg{3} + \lg{4}}
\end{equation*}

\begin{proof}
  \hfill
  \begin{alignat*}{4}
    & \frac{n}{2}\lg{n} & \equal  & \sum_{i=1}^{n/2}{\lg{n}}            & \evidence{definition of factorial} \\
    & {}                & \lesseq & \sum_{i=1}^{n/2}{\lg{i(n - i + 1)}} & \evidence{lemma 3} \\
    & {}                & \equal  & \sum_{i=1}^{n/2}{\lg{i} + \lg{(n - i + 1)}} & {} \\
    & {}                & \equal  & \sum_{i=1}^{n/2}{\lg{i}} + \sum_{i=1}^{n/2}{\lg{(n - i + 1)}} & {} \\
    & {}                & \equal  & \sum_{i=1}^{n/2}{\lg{i}} + \sum_{i=n/2 + 1}^{n}{\lg{i}} & \evidence{readjust bounds} \\
    & {}                & \equal  & \sum_{i=1}^{n}{\lg{i}}   & {} \\
    & {}                & \equal  & \lg{\prod_{i=1}^{n}{i}}  & \evidence{log product rule} \\
    & {}                & \equal  & \lg{n!}                 & \qedhere
  \end{alignat*}
\end{proof}

\begin{answer}
  \hfill
  \begin{proof}
    \vspace{1em}
    \begin{equation*}
      0 \le \frac{1}{2} \times n \lg{n} \le \lg{n!} \le 1 \times n\lg{n} \evidence{By Lemma 4 $\&$ 5}
    \end{equation*}
    So $\bigTheta{\lg{n!}}{n\lg{n}}$ where $c_1 = \frac{1}{2}$, $c_2 = 1$, and $m = 4$ $\qedhere$ 
  \end{proof}
\end{answer}

\section*{Problem 3}

\section*{Problem 4}

\begin{lemma}
  \begin{equation*}
    \quantify{a, b}{R} 1 \le a \wedge 1 \le b \Rightarrow a + b \le 2ab
  \end{equation*}
\end{lemma}

\begin{answer}[A]
  \hfill
  \begin{proof}
    $\quantify{n}{N} m \le n$ where m is the lowest q required to satsify $\bigO{f(q)}{g(q)}$
    \begin{alignat*}{4}
      & 0  & \lesseq & \lg{f(n)}                   & {} \\
      & {} & \lesseq & \lg{\parens{c \times g(n)}} & \evidence{$f(n) \le  c \times g(n)$ for some $c$ } \\
      & {} & \equal  & \lg{c} + \lg{\parens{g(n)}} & {} \\
      & {} & \equal  & (2\lg{c} \times \lg{g(n)}   & \evidence{By the above lemma where $a = \lg{c}$ and $b = \lg{\parens{g(n)}}$}
  \end{alignat*}
  So $\bigO{\lg{f(n)}}{\lg{g(n)}}$ where $c_1 = 2\lg{c}$ and $m_1 = m$ \qedhere
  \end{proof}
  
\end{answer}

\begin{answer}[B]
  $\bigO{2n^2}{n^2}$ trivally true so assume $\bigO{2^{2n^2}}{2^{n^2}}$
  \begin{alignat*}{4}
      & 2^{2n^2}             & \lesseq & c \times 2^{n^2}                    & {} \\
      & {\parens{2^{n^2}}}^2 & \lesseq & c \times 2^{n^2}                    & \evidence{Divide by $2^{n^2}$} \\
      & 2^{n^2}              & \lesseq & c                                  & {}
  \end{alignat*}
  We now see our assumption is incorrect because:
  $\nexists c \in \mathbb{R}. \hspace{0.2em} \quantify{n}{N} 2^{n^2} \le c \hfill \contra$
\end{answer}

\begin{answer}[C]
  
  \hfill
  \begin{proof}
    $\quantify{n}{N} m \le n$ where m is the lowest q required to satsify $\bigO{f(q)}{g(q)}$
    \begin{alignat*}{4}
      & 0  & \lesseq & 1 \times f(n)          & {} \\
      & {} & \lesseq & f(n) + g(n)   & \evidence{$0 \le g(n)$} \\
      & {} & \lesseq & 2 \times f(n) & \evidence{$g(n) \le f(n)$}
    \end{alignat*}
    So $\bigTheta{f(n)}{g(n)}$ where $c_1 = 1$, $c_2 = 2$ and $m_1 = m$ \qedhere
  \end{proof}
\end{answer}

\section*{Problem 5}

\begin{answer}[A]
  \hfill
  \begin{proof}
    $\quantify{n}{N}$
    \begin{alignat*}{4}
      & 0  & \lesseq & a_i n^{i}                & \evidence{$0 < a_i$ and $0 \le n^{i}$} \\
      & {} & \lesseq & \sum_{i=0}^{d}{a_i n^{i}} & {}
    \end{alignat*}
    So $P(n)$ is asymptotically non-negative forall $n$ \qedhere
  \end{proof}
\end{answer}

\begin{answer}[B]
  \hfill
  \begin{proof}
    Let $a_{max} = \maxSet{a_1, a_2, ..., a_n}$
    \begin{alignat*}{4}
      & 0  & \lesseq & \sum_{i=0}^{d}{a_i n^{i}} & \evidence{$P(n)$ is asymptotically non-negative} \\
      & {} & \lesseq & \sum_{i=0}^{d}{a_{max} n^{k}} & \evidence{$a_{i} \le a_{max}$ and $n^{i} \le n^{d} \le n^{k}$} \\
      & {} & \equal  & (d \cdot a_{max}) \cdot n^{k} & {}
    \end{alignat*}
    So $\bigO{P(n)}{n^{k}}$ where $c = d \cdot a_{max}$ and $m = 0$ \qedhere
  \end{proof}
\end{answer}

\begin{answer}[C]
  \hfill
  \begin{proof}
    \hfill \\
    Let $a_{max} = \maxSet{a_1, a_2, ..., a_n}$ \\
    $\quantify{n}{N} \maxSet{a_{max}, d} \le n^{\frac{1}{2}}$ 
    \begin{alignat*}{4}
      & 0  & \lesseq & \sum_{i=0}^{d}{a_i n^{i}}             & \evidence{$P(n)$ is asymptotically non-negative} \\
      & {} & \lesseq & \sum_{i=0}^{d}{n^{\frac{1}{2}} n^{k-1}} & \evidence{$a_i \le n^{\frac{1}{2}}$ and $n^{i} \le n^{d} \le n^{k-1} $} \\
      & {} & \lesseq & n^{\frac{1}{2}} n^{\frac{1}{2}} n^{k-1}  & \evidence{$d \le n^{\frac{1}{2}}$} \\
      & {} & \lesseq & c \cdot n^{k}                        & \evidence{multiply by c}
    \end{alignat*}
    So $\littleO{P(n)}{n^{k}}$ where $m = {\maxSet{a_{max},d}}^2$ \qedhere
  \end{proof}
\end{answer}


\end{document}

